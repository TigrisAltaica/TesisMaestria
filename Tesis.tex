\documentclass[a4paper,10pt]{report}
\usepackage{amsmath}
\usepackage{amssymb}
\usepackage{braket}
\usepackage{fancyvrb}
\usepackage{graphicx}
\usepackage{color}
\usepackage{colortbl}

\usepackage[spanish]{babel}
\selectlanguage{spanish}
\usepackage[utf8]{inputenc}

\DefineVerbatimEnvironment{code}{Verbatim}{fontsize=\small}
\DefineVerbatimEnvironment{example}{Verbatim}{fontsize=\small}


% Title Page
\title{Tesis de Maestría}
\author{Pablo Yanes}


\begin{document}
\maketitle

\begin{abstract}

\end{abstract}

\chapter{Sistemas Cuánticos Abiertos y Ecuaciones Maestras}

\section{Introducción}

Modelar la pérdida de energía en un sistema clásico como un oscilador armónico es un proceso directo \cite{CarmichaelQO} que consiste en modificar el Hamiltoniano usual:

\begin{equation}
H = \frac{p^2}{2m} + \frac{1}{2} m\omega^2 q^2
\end{equation}

Al agregarle un termino dependiente de la velocidad $-\gamma p$, lo cual lleva a la conocida ecuación:

\begin{equation}
\ddot{q} + \gamma \dot{q} + \omega^2 q = 0
\end{equation}

Sin embargo, este enfoque no es igualmente exitosos al tratar con sistemas cuánticos. En el procedimiento anterior se llega a las ecuaciones lineales:

\begin{equation}
\dot{q} = \frac{p}{m} \qquad \dot{p} = -\gamma p -m\omega ^2 q
\end{equation}

Como estás son lineales deben de seguir siendo validas al cambiar $p$ y $q$ por operadores, pero si se piensa en la evolución temporal del conmutador, tomando ahora a $p$ y $q$ como operadores:

\begin{align*}
 \dfrac{d}{dt}[q,p] =& \dot{q}p + q\dot{p} - \dot{p}q p-\dot{q}\\  
 =& -\gamma[q,p]
\end{align*}

Por lo que:

\begin{equation}
[q(t),p(t)] = e^{-\gamma t}[q(0),p(0)] = i\hbar e^{-\gamma t}
\end{equation}

El enfoque que en el caso clásico lleva a la solución correcta en el caso cuántico lleva a que la relación de conmutación canónica decaiga con el tiempo. Se busca otra forma de proceder.

\section{Enfoque de Sitema y Baño}

Este enfoque depende de asumir que el sistema no pierde energía por sí mismo, en realidad las perdidas de energía se deben un acoplamiento a un sistema mucho más grande, al cual llamamos baño. Normalmente las propiedades del baño no son de interés, más allá de propiedades básicas como temperatura. Se asume que el acoplamiento lleva a pérdidas de energía en el sistema de estudio pero que este sistema no es lo suficientemente grande como para alterar las propiedades del baño. Esto lleva a un Hamiltoniano de la forma:

\begin{equation} \label{BathSysH}
H = H_{Sis} + H_{B} + H_{Int}
\end{equation}

Un Hamiltoniano total formado por la suma del Hamiltoniano del sistema original sin péridas de energía, un Hamiltoniano para el baño y un Hamiltoniano de acoplamiento entre los otros dos.  Dado que solo es de interés conocer las propiedades del sistema, esto sugiere el uso de matrices densidad.

\subsection{Matriz Densidad}

Si un sistema se encuentra en un estado $\Ket{\Psi}$, su matriz densidad\cite{BlumDM}, usualmente denotada como $\rho$ está dada por:

\begin{equation}
\rho = \Ket{\Psi}\Bra{\Psi}
\end{equation}

La matriz contiene toda la información del sistema y tiene varias propiedades de interés. Si se desea conocer el valor de expectación de un operador, este es:

\begin{equation}
<A> = Tr[A\rho]
\end{equation}

Su evolución temporal se comporta de acuerdo con la ecuación de Liouville:

\begin{equation}
\dfrac{\partial \rho}{\partial t} = \frac{1}{i\hbar}[H,\rho]
\end{equation}

Finalmente, y de especial utilidad, existe una forma de extraer información de un subsistema de la matriz densidad. Si se conoce la matriz densidad de un sistema compuesto formado por dos sistemas $A$ y $B$ tal que el sistema compuesto se encuentra en un espacio formado por el producto directo de los dos subsistemas:

\begin{equation}
H_{AB} = H_A \otimes H_B
\end{equation}

Entonces es posible encontrar la matriz densidad correspondiente a uno de los dos subespacios al tomar la traza sobre el otro:

\begin{equation}
\rho_A = Tr_B [\rho_{AB}]
\end{equation}

Esto muestra la utilidad de este formalismo para trabajar con Hamiltonianos del tipo \ref{BathSysH}, ya que si se encuentra una solución basta con tomar la traza sobre los estados del baño para encontrar una solución para el sistema con pérdidas de energía.

\section{Deducción de Ecuacion Maestra Mediante Operadores de Proyección}

Se busca obtener la ecuación correspondiente a un sistema sujeto a interacción con un baño térmico:

\begin{equation}
\hat{H} = \hat{H}_{Sis}+\hat{H}_{Int}+\hat{H}_B
\end{equation}

Lo cual implica una ecuación de evolución temporal para $\rho$ de la forma:

\begin{equation}\label{LMeq}
\dot{\rho}_{total} = (L_{Sis}+L_{Int}+L_B)\rho
\end{equation}

Donde:

\begin{equation}\label{defL}
L\rho(t) = \frac{i}{\hbar}[\hat{H},\rho(t_0)]
\end{equation}

Se busca una solución para el operador densidad únicamente del sistema, por lo que se busca una ecuación para el operador total después de tomar la traza sobre las variables del baño:

\begin{equation}
\rho(t) \equiv Tr_B\{\rho_{total}(t)\}
\end{equation}

Se define un operador de proyección \cite{ZollerQN} que opere sobre $ \rho $:

\begin{equation}
P\rho_t = Tr_B\{\rho_{t}(t)\}\otimes \rho_B
\end{equation}

Y un operador:

\begin{equation}
Q = \mathbf{1}-P
\end{equation}

Tal que:

\begin{align}
\rho_t =& P\rho_t + Q\rho_t \\
\equiv & v(t)+u(t)
\end{align}

Se emplean las siguientes propiedades:

\begin{enumerate}
\item $ PL_{sis} = L_{sis}P$\qquad Ya que $ P $ y $ L_{sis} $ operan en espacios distintos.

\item $ PL_{B} = L_{B}P = 0 $\qquad Por conservación de la probabilidad.

\item $PL_{int}P=0$ \qquad Se asume que la interacción no tiene términos diagonales en las variables del baño

\item $P^2 = P \quad Q^2 = Q$ \qquad $P$ y $Q$ son proyectores
\end{enumerate}

Por convieniencia se trabaja con la transformada de Laplace [citar arfken] de las proyecciones $v(t)$ y $u(t)$:

\begin{equation}
\tilde{v}(t) = \int_0^\infty e^{-st}v(t)dt \qquad \tilde{u}(t) = \int_0^\infty e^{-st}u(t)dt
\end{equation}

Dado que las dos proyecciones se encuentran es espacios distintos, se puede sustituir $\rho$ por ellas en \eqref{LMeq} y resolver por separado para cada proyección. En el caso de $v(t)$ se tiene, antes de utilizar la transformada de Laplace, al aplicar el proyector $P$ a \eqref{LMeq}:

\begin{align*}
P\dot{p}(t)_t =& P(L_{Sis}+L_{Int}+L_B)\rho \\
 \dot{v}(t) =& (PL_{Sis}+PL_{Int}+PL_B)\rho \\
 =&(L_{Sis}P+PL_{Int}+\underbrace{PL_B}_0)\rho \\
 =& L_{sis}P\rho_t + PL_{int}\rho_t \\
 =& L_{sis}v + \underbrace{PL_{int}(P}_0+Q)\rho_t \\
 =& L_{sis}v + PL_{int}Q\rho_t \\
 =& L_{sis}v + PL_{int}u 
\end{align*}

De manera análoga se obtiene una ecuación para $u$ al proyectar con $Q$:

\begin{align*}
Q\dot{p}(t)_t =& Q(L_{Sis}+L_{Int}+L_B)\rho \\
 \dot{u}(t) =& (QL_{Sis}+QL_{Int}+QL_B)\rho \\
 =&(L_{Sis}Q+QL_{Int}+L_BQ)\rho \\
 =& L_{sis}Q\rho_t + L_{B}Q\rho_t + QL_{int}\rho_t \\
 =& L_{sis}u + L_{B}u + QL_{int}(Q+P)\rho_t  \\
 =& L_{sis}u + L_{B}u + QL_{int}v+QL_{int}u\rho_t \\
 =& (L_{Sis}+L_B+QL_{Int})u +QL_{Int}v 
\end{align*}

Al aplicar la transformada de Laplace a la ecuación para $v$ se obtiene:

\begin{equation}\label{eqv}
s\tilde{v}(s)-\tilde{v}(0) = L_{sis}\tilde{v}(s) + PL_{int}\tilde{u}(s)
\end{equation}

Y en el caso de $u$:

\begin{equation}\label{equ}
s\tilde{u}(s)-\tilde{u}(0) = (L_{Sis}+L_B+QL_{Int})\tilde{u}(s) + QL_{Int}\tilde{v}(s)
\end{equation}

Se resuelve \eqref{equ} en términos de $v$:

\begin{equation}
\tilde{u}(s)=\frac{QL_{Int}\tilde{v}(s)+\tilde{u}(0)}{s-\underbrace{(L_{Sis}+L_B+QL_{Int})}_\alpha}
\end{equation}

Y se sustituye el resultado en \eqref{eqv}:

\begin{equation}
s\tilde{v}(s)-(\tilde{v}(0)+PL_{int}\alpha\tilde{u}(0))=(L_{sis}+PL_{int}\alpha QL_{int})\tilde{v}(s)
\end{equation}

En este punto se toma la aproximación de aclopamiento débil:

\begin{equation}
L_{int} \rightarrow \Xi L_{int}
\end{equation}

Los términos asociados a $\tilde{u}0$ son únicamente correciones a la condición inicial, por lo que se desprecian. Tomando esto en cuenta, se obtiene:

\begin{equation}
s\tilde{v}(s)-\tilde{v}(0) = (L_{sis}+\Xi^2\frac{PL_{int}QL_{int}}{s-L_{sis}-L_{B}})s\tilde{v}(s)
\end{equation}

Y al tomar la transformada inversa de Laplace:

\begin{equation}\label{eqProyectores}
\dot{v}(t)=L_{sis}v(t) + \Xi^2 PL_{int} \int_{0}^{\infty}d\tau e^{(L_{sis}+L_{B})\tau}QL_{int}v(t-\tau)
\end{equation}

Se busca obtener una ecuación en términos de los operadores del Hamiltoniano \cite{ZollerQN},se procede con el segundo término de la ecuación \eqref{eqProyectores}:

\begin{align*}
QL_{int}v=&(1-P)L_{int}v\\
=& (1-P)L_{int}P\rho \\
=& L_{int}P\rho \\
=& L_{int}v
\end{align*}

Con esto el segundo término de la ecuación \eqref{eqProyectores}, y tomando en cuenta la definición de los términos $L$ dada en \eqref{defL} y el lema de Baker-Campbell-Housedorff \cite{SakuraiQM}

\begin{align*}
=& \Xi^2 PL_{int} \int_{0}^{\infty}d\tau e^{(L_{sis}+L_{B})\tau}L_{int}v(t-\tau)\\
=& \Xi^2 PL_{int} \int_{0}^{\infty}d\tau e^{\frac{-i}{\hbar}H_B\tau}e^{\frac{-i}{\hbar}H_s\tau}L_{int}v(t-\tau)e^{\frac{i}{\hbar}H_B\tau}e^{\frac{i}{\hbar}H_s\tau}\\
=&\Xi^2 PL_{int} \int_{0}^{\infty}d\tau L_{int}v(t-\tau) \\
=&\Xi^2 PL_{int} \int_{0}^{\infty}d\tau[H_{int}^*,P\rho(t-\tau)] \\
=&\Xi^2 P[H_{int},\int_{0}^{\infty}d\tau[H_{int}^*,tr_B(\rho(t-\tau))\otimes\rho_B]]
\end{align*}

En este punto se realiza la aproximación de Markov \cite{ZollerQN}, al hacer la sustitución:

\begin{equation}
t-\tau \rightarrow t
\end{equation}

Y se traza toda la ecuación sobre las variables del baño. Con esto se llega a la ecuación:


\chapter{Ecuaciones Maestras: Atomo de dos Niveles, Oscilador Armónico y la Base de Decaimiento}

Se busca poder resolver ecuaciones como las encontradas en el capitulo anterior, estás en general se escriben de la forma:

\begin{equation}\label{MasterEq}
\dot{\rho} = \frac{1}{i\hbar}[H,\rho] + L\rho
\end{equation}

Donde $\rho$ es la matriz densidad del sistema, $H$ es el Hamiltoniano que modela las partes del sistema donde no hay pérdidas de energía y $L$ es el operador de Lindblat, el cual corresponde a la parte abierta del sistema. En la mecánica cuántica cerrada, el análisis se limíta al operador Hamiltoniano el cual es un operador auto adjunto. En ese caso, es suficiente resolver la ecuación de Schröedinger estacionaria:

\begin{equation}
H\Psi = E\Psi
\end{equation}

cuya solución formal es:

\begin{equation}
\Psi(t) = e^{\frac{-iHt}{\hbar}}\Psi(0)
\end{equation}

La función $\Psi(0)$ puede expresarse como una  combinación lineal de las funciones propias de $H$. Al hacer esto, la ecuación anterior toma la forma:

\begin{equation}
\Psi(t) = \sum_j c_j e^{\frac{-iE_jt}{\hbar}}\phi_j(0)
\end{equation}

Los coeficientes $c_j$ son las proyecciones de la función $\Psi(0)$ sobre las funciones propias de $H$

Se busca una solución equivalente para la ecuación \eqref{MasterEq}. Es posible transformar dicha ecuación al cuadro de interacción, donde solo importa la parte abierta del problema. En este cuadro, se busca una solución del tipo:

\begin{equation}
\rho(t) = \sum_{\lambda} c_{\lambda} e^{\lambda t}\rho_{\lambda} 
\end{equation}

Los coeficientes ahora son respecto a las funciones propias de $L$. Es importante notar que $L$ no es necesariamente un operador autoadjunto, por lo que no se puede garantizar que sus funciones propias formen una base completa del espacio ni que sus valores propios sean reales. En el caso de este tipo de operadores existen dos tipos de vectores propios\cite{BarnettSD}, izquierdos y derechos:

\begin{align*}
L\rho =& \lambda \rho \\
\rho L =& \lambda \check{\rho}
\end{align*}

Los estados que se obtienen al aplicar el superoperador por la derecha se conocen también como estados duales, se denotan por $\check{\rho}$. También es importante notar que $\Psi$ es un vector de estado que pertenece al espacio de Hilbert mientras que $rho$ es un operador que pertenece al espacio de Liouville. Por esto, se dice que $L$ es un superoperador, los vectores que se obtienen al aplicar por la izquierda y por la derecha no son simplemente complejos conjugados uno del otro, se conocen como duales y son ortogonales los unos a otros bajo un producto definido mediante la traza\cite{EnglertDB} y sus eigenvalores son iguales.

\begin{equation}
Tr[\check{\rho}_\lambda \rho_{\lambda'}] = \delta_{\lambda'\lambda}
\end{equation}

Esto se puede emplear para obtener las constantes de la expansión:

\begin{equation}
\rho(0) = \sum_{\lambda} \check{c} \rho_{\lambda}
\end{equation}

Donde:

\begin{equation}
\check{c}_\lambda = Tr[\check{\rho}_\lambda \rho(0)]
\end{equation}

Es importante notar que para realizar una expansion es estados de este tipo es necesario poder trazar sobre los estados duales. Por esto, estos estados se estudiarán de manera más cuidadosa en una sección posterior.

\section{Átomo de dos Niveles}

Antes de intentar resolver sistemas más complejos, es ilustrativo tratar el caso relativamente sencillo del átomo de niveles. A este sistema le corresponde el operador de Linblad:

\begin{align}\label{TwoLevel}
 L_{\sigma}\rho = 
 &-\frac{B}{2}(1-s)[\sigma_+\sigma_-\rho + \rho\sigma_+\sigma_--2\sigma_-\rho\sigma_+ ]\nonumber\\
 &-\frac{B}{2}s[\sigma_-\sigma_+\rho + \rho\sigma_-\sigma_+-2\sigma_+\rho\sigma_- ]\nonumber\\
 &-\frac{2C-B}{4}[\rho - \sigma_z\rho\sigma_z]
\end{align}

Es convenitente utilizar notación de Dirac en este tratamiento. Todos los operadores involucrados pueden verse como matrices de dos por dos. $\sigma_z$ es la matriz de Pauli correspondiente al eje z y:

\begin{equation}
\sigma_{\pm} = \sigma_x \pm \sigma_y = (\sigma_{\mp}) ^\dagger
\end{equation}

Un método \cite{BarnettSD} para encontrar los vectores y valores propios de $L$ consiste en proponer una forma adecuada para los vectores, aplicar $L$ a la propuesta y resolver las ecuaciones resultantes. En este caso se propone:

\begin{equation}
\rho_\sigma = \alpha\Ket{+}\Bra{+}+\beta\Ket{+}\Bra{-}+\gamma\Ket{-}\Bra{+}+\delta\Ket{-}\Bra{-}
\end{equation}

Ya que se conoce la forma en la cual las matrcies $\sigma$ operan sobre estos elementos:

\begin{align*}
&&\sigma_{+}\Bra{-} =& \Bra{+} \quad \sigma_{+}\Bra{+} = 0 \\
&&\sigma_{-}\Bra{+} =& \Bra{-} \quad \sigma_{-}\Bra{-} = 0 \\
&&\sigma_{z}\Bra{+} =& \Bra{+} \quad \sigma_{z}\Bra{-} = -\Bra{-}
\end{align*}

Al sustituir $\rho_{\sigma}$ en \eqref{TwoLevel} se obtiene, para el primer término de $L$:

\begin{equation}
-\frac{B}{2}(1-s)[2\alpha\Ket{+}\Bra{+}+\beta\Ket{+}\Bra{-}+\gamma\Ket{-}\Bra{+}-2\alpha\Ket{-}\Bra{-}]
\end{equation}

El segundo término es:

\begin{equation}
-\frac{B}{2}(s)[2\delta\Ket{+}\Bra{+}+\beta\Ket{+}\Bra{-}+\gamma\Ket{-}\Bra{+}-2\delta\Ket{-}\Bra{-}]
\end{equation}

Y el tercero:

\begin{equation}
-\frac{2C-B}{4}[2\beta\Ket{+}\Bra{-}+2\gamma\Ket{-}\Bra{+}]
\end{equation}

Esto se inserta en:

\begin{equation}
L\rho = \lambda\rho
\end{equation}

Y dado que cada uno de los cuatro componentes de $\rho$ son linealmente independientes, esto genera cuatro ecuaciones distintas:

\begin{align}
B(s(\alpha+\delta)-\alpha)\Ket{+}\Bra{+} =& \lambda\alpha\Ket{+}\Bra{+}\\
B(-s(\alpha+\delta)+\alpha)\Ket{-}\Bra{-} =& \lambda\delta\Ket{-}\Bra{-}\\
-\beta C \Ket{+}\Bra{-} =& \lambda \beta \Ket{+}\Bra{-} \\
-\gamma C \Ket{-}\Bra{+} =& \lambda \gamma \Ket{-}\Bra{+}
\end{align}

De las cuales se obtienen los cuatro vectores propios:

\begin{align}
L_\sigma \sigma_0 =& 0 \\
L_\sigma \sigma_z =& -B \sigma_z \\
L_\sigma \sigma_\pm =& -C \sigma_\pm
\end{align}

donde:

\begin{equation}
\sigma_0 = \frac{1}{2}[I + (2s-1)\sigma_z]
\end{equation}

Las constantes en \eqref{MasterEq} son todas positivas por lo que todos los valores propios son no positivos. Esto es importante, ya que al hacer tender el tiempo a infinito solo el término que corresponde al valor propio 0 sobrevive, los demás términos decaen exponencialmente con el tiempo. Esto lleva a identificar al término con valor propio 0 como el estado estacionario del sistema. Los estados duales se obtienen de la misma forma y resultan ser\cite{EnglertDB}:

\begin{align}
\check{\sigma_0} L_\sigma  =& 0 \\
\check{\sigma_z} L_\sigma =& -B \check{\sigma_z} \\
\check{\sigma_{\pm}} L_\sigma =& -C \check{\sigma_{\pm}}
\end{align}

Donde:

\begin{align*}
\check{\sigma_0} =& 1\\
\check{\sigma_z} =& \frac{1}{2}[\sigma_z-(2s-1)]\\
\check{\sigma_{\pm}} =& \frac{1}{4}\sigma_{\mp}
\end{align*}

Se puede verificar rápidamente la condición de ortogonalidad, por ejemplo utilizando $\check{\sigma_0}$:

\begin{align*}
Tr[\sigma_0\check{\sigma_0}] =& Tr[\frac{1}{2}[I + (2s-1)\sigma_z]]\\
=& \frac{1}{2} Tr[I] + \frac{2s-1}{2}Tr[\sigma_z]\\
=& 1
\end{align*}

Donde se utiliza el hecho de que las matrices de Pauli tienen traza cero. Con esta misma propiedad se puede verificar que los productos restantes con $\check{\sigma_0}$ se anulan.

\section{Oscilador Armónico: Primer Ansatz}

Se busca modelar interacciones que involucran un capo electromagnético. Este se modela simplemente como una suma infinita de modos de oscilador armónico. Este sistema en particular corresponde a un operador de Lindblat de la forma \cite{EnglertDB}

\begin{align}\label{EMField}
L_a \rho =& - \frac{A}{2}(\nu + 1)[a^\dagger a\rho + \rho a^\dagger a -2a\rho a^\dagger] \nonumber \\
 &- \frac{A}{2}(\nu)[ aa^\dagger\rho + \rho  aa^\dagger -2a^\dagger\rho a]
\end{align}

Esto modela un campo electromagnético en una cavidad con acoplamiento a un reservorio térmico con un número $\nu$ promedio de fotones térmicos. $A,\nu \geq 0$. La elección de un ansatz para este sistema es mucho más compleja, ya que ahora la base estados sobre la cual actúan los operadores que forman $L$ es infinita.  En base a lo visto en la sección anterior, se propone como ansatz una base que acople estados de distinta energía:

\begin{equation}
 \rho^l_n = \sum_{n=0}^\infty Q_n^l\Ket{n}\Bra{n+l}
\end{equation} 

El objetivo de nuevo es sustituir el ansatz en \eqref{EMField} para obtener un sistema de ecuaciones para los valores y vectores propios. Se trabaja primero con el primer término de \eqref{EMField}, donde se ignora el coeficiente exterior por brevedad:

\begin{align*}
=&[a^\dagger a\sum_{n=0}^\infty Q_n^l\Ket{n}\Bra{n+l} + \sum_{n=0}^\infty Q_n^l\Ket{n}\Bra{n+l} a^\dagger a -2a\sum_{n=0}^\infty Q_n^l\Ket{n}\Bra{n+l} a^\dagger]\\
=&[\sum_{n=0}^\infty Q_n^la^\dagger a\Ket{n}\Bra{n+l} + \sum_{n=0}^\infty Q_n^l\Ket{n}\Bra{n+l} a^\dagger a -2\sum_{n=0}^\infty Q_n^la\Ket{n}\Bra{n+l} a^\dagger]\\
=&[\sum_{n=0}^\infty Q_n^ln\Ket{n}\Bra{n+l} + \sum_{n=0}^\infty Q_n^l\Ket{n}\Bra{n+l} (n+l) -2\sum_{n=0}^\infty Q_n^l\sqrt{n}\Ket{n-1}\Bra{n+l-1}\sqrt{n+l}]\\
=&[\sum_{n=0}^\infty Q_n^l(2n+l)\Ket{n}\Bra{n+l} -2\sum_{n=0}^\infty\sqrt{n(n+l)} Q_n^l\Ket{n-1}\Bra{n+l-1}]\\
\end{align*}

El segundo término actúa de una forma completamente análoga y se obtniene:

\begin{equation*}
=[\sum_{n=0}^\infty Q_n^l(2n+l+2)\Ket{n}\Bra{n+l} -2\sum_{n=0}^\infty\sqrt{(n+1)(n+l+1)} Q_n^l\Ket{n+1}\Bra{n+l+1}]
\end{equation*}

Como $p^l_n$  es un vector propio los términos anteriores deben de poder igualarse a un valor propio por $\rho^l_n$:

\begin{align}
\lambda_n^l \rho_n^l=& \sum_{n=0}^\infty Q_n^l \{ -\frac{A}{2}(\nu+1)[(2n+l)\Ket{n}\Bra{n+l} -2\sqrt{n(n+l)} Q_n^l\Ket{n-1}\Bra{n+l-1}] \nonumber \\
-&\frac{A}{2}(\nu)[(2n+l+2)\Ket{n}\Bra{n+l}-2\sqrt{(n+1)(n+l+1)} Q_n^l\Ket{n+1}\Bra{n+l+1}]\}
\end{align}

Si en el segundo término se recorre el índice de suma hacia arriba por uno y en el cuarto hacía abajo por uno, se puede utilizar independencia lineal respecto al elemento $\Ket{n}\Bra{n+l}$ para obtener una ecuación para cada elemento de la suma. Se obtiene que los coeficientes deben cumplir con:

\begin{align}\label{RecurrenciaStengholm}
A&(\nu+1) Q_{n+1}^l \sqrt{(n+1)(n+l+1) }+ A(\nu) Q_{n-1}^l\sqrt{n(n+l)}\\ \nonumber
=&[\lambda_n^l+A\nu(2n+l+2)+\frac{A}{2}(2n+l)]Q_{n}^l
\end{align}

El operador de dual Linblat para este sistema es:

\begin{align}
L_a \rho =& - \frac{A}{2}(\nu + 1)[a^\dagger a\rho + \rho a^\dagger a -2a^\dagger\rho a] \nonumber \\
 &- \frac{A}{2}(\nu)[ aa^\dagger\rho + \rho  aa^\dagger -2a\rho a^\dagger]
\end{align}

Lo cual sugiere el mismo ansatz que para el caso de \ref{EMField}, y en efecto se obtiene una relación extremadamente similar, únicamente se intercambian los términos donde el operador densidad se coloca enmedio. No he logrado resolver estas relaciones de recurrencia de forma anlítica para condiciones generales. Sin embargo, en el caso de temperatura cero, se tiene que $\nu = 0$ y \ref{RecurrenciaStengholm} se simplifica considerablemente:

\begin{equation}
AQ_{n+1}^l\sqrt{(n+1)(n+l+1)}=[\lambda_n^l + \frac{A}{2}(2n+l)]Q_n^l
\end{equation}

Lo cual nos permite resolver para una de las dos $Q$:

\begin{equation}
Q_{n+1}^l = \frac{\lambda_n^l + \frac{A}{2}(2n+l)}{A\sqrt{(n+1)(n+l+1)}}Q_n^l
\end{equation}

Lo cual tiene un número infinito de soluciones. En \cite{BarnettSD} se corta la suma con el valor de $\lambda$ y se llega a una solución para A, que ahora también depende de un nuevo entero $m$:

\begin{equation}
A_m^l = \sum_{n=0}^m (-1)^m \frac{m!}{(m-n)!}\sqrt{\frac{l!}{n!(n+l)!}}\Ket{n}\Bra{n+l}
\end{equation}

Al cual le corresponde a un valor para $\lambda$ de:

\begin{equation}
\lambda_l^m = -(2m+l)
\end{equation}

Utilizando la condición de ortogonalidad con los estados duales se llega a que estos son \cite{BarnettSD}:

\begin{equation}
B_m^l = \sum_{n=m}^\infty \sqrt{\frac{n!(n+l)!}{l!}}\frac{1}{(n-m)!}\Ket{n}\Bra{n+l}
\end{equation}

Sin embargo esto solo es valido en el caso de temperatura cero. Para poder tratar el caso general se requiere de otro enfoque.

\section{Oscilador Armónico: Base de Decaimiento}




\begin{equation}\label{Englert1993}
\rho_\lambda (a,a^\dagger) = :f(aa^\dagger):a^l
\end{equation}

Se puede en $a^l$ como

\begin{equation}
\sum_{n=0}^\infty C_n^l\Ket{n}\Bra{n+l}
\end{equation}

Es entonces posible relacionar este ansatz con el de Englert, lo cual lleva a su solución para $\rho_n^l$

\begin{align}\label{DefDB}
&a^{\dagger l}\frac{(-1)^n}{(\nu+1)^{l+1}}:L_n^l[\frac{a^\dagger a}{\nu+1}]e^{-[\frac{a^\dagger a}{\nu+1}]}:\quad l \geq 0 \\
&\frac{(-1)^n}{(\nu+1)^{|l|+1}}:L_n^{|l|}[\frac{a^\dagger a}{\nu+1}]e^{-[\frac{a^\dagger a}{\nu+1}]}:a^{|l|}\quad l \leq 0
\end{align}

con autovalores

\begin{equation}
\lambda_n^l = -A[n + \frac{|l|}{2}]
\end{equation}

con las condiciones

\begin{equation}
n=0,1,2...,\qquad l = 0,\pm 1, \pm 2,... 
\end{equation}

A continuación se muestra de forma explicita como calcular estos elementos. 

\subsection{Estado Base}

Este es el estado con valor propio 0, lo cual corresponde a $n=0, \qquad l=0$. En este caso, la solución tiene la forma

\begin{equation}
\frac{1}{(\nu+1)}:L_0^0[\frac{a^\dagger a}{\nu+1}]e^{-[\frac{a^\dagger a}{\nu+1}]}:
\end{equation}

ya que $L_0^0 = 1$ \cite{ArfkenMM} si se desarrolla la exponencial en una serie infinita se tiene, aplicando el ordenamiento normal:

\begin{equation}
\sum_{n=0}^\infty \frac{-(a^\dagger)^n (a)^n}{n!(\nu+1)^n}
\end{equation}

si se sustituye esto en la expresión para el estado base, junto con la relación de cerradura para la base de Foch se llega a \cite{EnglertDB}

\begin{equation}
\rho_0^0 = \frac{1}{1+\nu} [\frac{\nu}{1+\nu}]^{a^\dagger a}
\end{equation}

El cual corresponde a un estado térmico. El estado base es el estado estacionario del sistema.

\section{Estados Duales de la base de decaimiento}

La base de decaimiento \cite{EnglertDB} definida en \ref{DefDB} corresponde a eigenvalores:

\begin{equation}
 -A[n+\frac{|l|}{2}]
\end{equation}

Lo cual corresponde a una doble degeneración $\pm l$. Antes de poder realizar la expansión de estados arbitrarios en esta base es necesario conocer los eigenestados del operador dual, los cuales son aquellos que cumplen la condición\cite{EnglertDB}:

\begin{equation}
\bar{\rho}_\gamma L = \gamma\bar{\rho}_\gamma
\end{equation}

Es decir, los eigenestados \textit{izquierdos} del operador de Lindblat. Los estados duales cumplen con:

\begin{equation}
Tr[\bar{\rho}_{\gamma}\rho_{\gamma'}] = \delta_{\gamma\gamma'}
\end{equation}

Donde la barra denota al estado dual y $\gamma$ denota algún eigenvalor particular del problema. Bajo este producto es posible entonces resolver las ecuaciones para los coeficientes de una expansión de un estado particular en estados de la base de decaimiento:

\begin{equation}
\Psi = \sum_{\gamma}\bar{c}_\gamma \rho_\gamma
\end{equation}

Entonces:

\begin{equation}
\bar{c}_\gamma = Tr[\Psi\bar{\rho}_{\gamma}]
\end{equation}

Con esto la evolución temporal del sistema queda determinada:

\begin{equation}
\Psi(t) = \sum_\gamma \bar{c}_\gamma e^{\frac{-i}{\hbar} \gamma t} \rho_{\gamma}
\end{equation}

\subsection{Estados Duales: Forma Explícita}

Ya que es necesario tomar trazas sobre los estados duales, es conveniente encontrar tanto su forma explícita en términos de operadores de creación y aniquilación, así como una relación de recurrencia que permita relacionar un estado a los estados de menor número $n$. Los polinomios asociados de Laguerre, en general, son de la forma\cite{ArfkenMM}

\begin{equation}\label{DefLaguerre}
L_n^k(x) = \sum_{m=0}^n (-1)^m \frac{(n+k)!}{(n-m)!(k+m)!m!} x^m
\end{equation}

Recordando que en este caso la variable $x$ corresponde a $\frac{a^\dagger a}{\nu}$, y que los estados se encuentran multiplicados por un coeficiente:

\begin{equation}
(\frac{-\nu}{1+\nu})^n \frac{n!}{(n+k)!}
\end{equation}

Se llega a la expresión general, la cual toma en cuenta el ordenamiento normal y los operadores de aniquilación a la derecha del mismo:

\begin{equation}\label{DefDual}
\sum_{m=0}^n \frac{(-\nu)^{n-m}}{(\nu+1)^n} \frac{n!}{(n-m)!(m+k)!m!} a^{\dagger m}a^{m+k}
\end{equation}

Basandose en \ref{DefDual}, se puede ver que para cualquier valor de $n$ y $k \\geq 0$ el primer término de la suma, es decir el correspondiente a $m=0$ es:

\begin{equation}
(\frac{-\nu}{1+\nu})^n \frac{a^k}{k!}
\end{equation} 

El último término de la suma, el correspondiente a $m=n$, es:

\begin{equation}
\frac{1}{(\nu+1)^n} \frac{1}{(n+k)!} a^{\dagger n}a^{n+k}
\end{equation}

Si se desea una representación de estos estados en la base de Foch, se puede recurrir a:

\begin{equation}
I = \sum^\infty _n \Ket{n}\Bra{n}
\end{equation}

Y aplicar todos los operadores desde la derecha. En el caso del primer término ya antes mencionado, se obtiene:

\begin{equation}
 (\frac{-\nu}{1+\nu})^n \sqrt{\frac{n!}{(n-k)!}}\sum^\infty _{l=k} \Ket{l-k}\Bra{l}
\end{equation}

Los estados con número de ocupación menor a $k$ se anulan por la definición del operadore de aniquilación. En el caso del último término de la suma, se obtiene:

\begin{equation}
(\frac{1}{1+\nu})^n \frac{n!}{(n-k)!}\sum^\infty _{l=n+k} \Ket{l-k}\Bra{l}
\end{equation}

Puede verse que ambos términos involucran la misma diferencia en el número de ocupación del ket y del bra, pero la suma correspondiente al último término no cubre tantos estados, pues ahora todo estado con número menor a $n+k$ es anulado.

\subsection{Relación de Recurrencia para Estados Duales}

A fin de obtener una relación de recurrencia para los estados duales, es posible basarse en una de las muchas relaciones de recurrencia de los polinomios asociados de Laguerre \cite{ArfkenMM}:

\begin{equation}
L_n^{k+1}(x) = \frac{n-x}{n}L_{n-1}^{k+1}(x) + \frac{n+k}{n}L_{n-1}^{k}(x)
\end{equation}

Si se sustituye esto en la expresión para el estado dual correspondiente se tiene, recordando que se utiliza $x$ para la variable únicamente por simplicida:

\begin{align}
\hat{\rho}_n ^{k+1} =& [\frac{-\nu}{1+\nu}]^n \frac{n!}{(n+k+1)!} :L_n^{k+1}:a^{k+1}\\
=&[\frac{-\nu}{1+\nu}]^n\frac{n!}{(n+k+1)!} :\frac{n-x}{n}L_{n-1}^{k+1}(x) + \frac{n+k}{n}L_{n-1}^{k}(x):a^{k+1}\\
=&[\frac{-\nu}{1+\nu}]^n\frac{n!}{(n+k+1)!}[:L_{n-1}^{k+1}(x)+L_{n-1}^{k}(x):+\frac{1}{n}:-xL_{n-1}^{k+1}(x)+L_{n-1}^{k}(x):]a^{k+1}\\
=&[(\frac{-\nu}{1+\nu})\frac{1}{(n+k+1)}+ \frac{(n+k)}{-\nu(\nu+1)}a^\dagger \bullet a]\hat{\rho}_{n-1} ^{k+1} + [(\frac{-\nu}{1+\nu}) (n+1)(n+k+1)(n+k)]\hat{\rho}_{n-1} ^{k}
\end{align}

Donde $\bullet$ indica que el estado dual va entre el operador de creación y el de aniquilación.


\chapter{Solución de la Ecuación Maestra Mediante Estados de Floquet}

Para resolver el problema correspondiente a un oscilador armónico amortiguado con frequencia dependiente del tiempo, se utiliza la teoría de Floquet \cite{WardFT}y se busca una forma de la ecuación maestra (eq capitulo anterior) expresada mediante operadores de Floquet, que se definirán más adelante.

\section{Teoría de Floquet}

Sea un problema del tipo:

\begin{equation}
x' = A(t)x
\end{equation}

Donde la función $A(t)$ es periódica con periodicidad $\tau$. En este caso el teorema de Floquet dice que la solución no necesariamnte es periódica pero debe tener la forma:

\begin{equation}\label{FloquetForm}
x(t)=e^{\mu t}p(t)
\end{equation}

Se considera que $x(t)$ es un vector de $n$ dimensiones y que $A(t)$ es una matriz de $n\times n$. Existen $n$ valores distintos de $\mu$ tales que:

\begin{equation}
e^{\mu_1 }...e^{\mu_n}=e^{\int_0^\tau tr[A(t)]}
\end{equation}

Los valores de $\mu$ se conocen como exponentes característicos. Es importante notar que la función $p(t)$ es periódica con período $\tau$. Los coeficientes $\mu$ son, en general complejos.

\section{Estados de Floquet en Mecánica Cuántica}

Se busca estudiar Hamiltonianos del tipo:

\begin{equation}
H(t)=H(t+\tau)
\end{equation}

El hecho de que el Hamiltoniano sea simétrico respecto a (ciertas) traslaciones en el tiempo, permite el uso del formalismo de Floquet \cite{HanngiDQS}. Se asume que la dependencia temporal puede ser vista como una perturbación sobre un Hamiltoniano original:

\begin{equation}
H(x,t)=H_0(x)+V(x,t) \qquad V(x,t)=V(x,t+\tau)
\end{equation} 

También se asume que el Hamiltoniano no perturbado poseé un conjunto completo de eigenfuciones $\{\phi_n\}$ con eigenvalores correspondientes $E_n$. En este caso, la ecuación de Schr\"{o}dinger tiene la forma:

\begin{equation}\label{SchrodingerEQ}
-i\hbar\dot{\Psi}(x,t) = H(x,t)\Psi(x,t)
\end{equation}

El problema admite una solución del tipo visto en la sección anterior:

\begin{equation}
\Psi_n(x,t) = e^{(\frac{-i}{\hbar}\mu_nt)}\Phi_n(x,t)
\end{equation}

Como se mencionó en la sección anterior, $\mu$ en general es un número complejo, lo cual puede llevar a soluciones inestables. En este caso $\Phi_n(x,t)$ es la función que contiene la periodícidad en el tiempo. Sustituir la solución en la ecuación \ref{SchrodingerEQ} genera una ecuación para las funciones periódicas:

\begin{equation}
H(x,t)\Phi_n(x,t)=\Xi_n\Phi_n(x,t)
\end{equation}

\section{Oscilador Armónico Dependinte del Tiempo: Solución Mediante Formalismo de Floquet}

En el caso clásico \cite{HanngiFM} se tiene, para un oscilador armónico unidimensional con frecuencia dependiente del tiempo y el cual experimenta una fuerza disipativa dependiente de la velocidad, que la posición cumple:

\begin{equation}
\ddot{x}+\gamma\dot{x}+\frac{k(t)}{m}x=0
\end{equation}

Se asume que la función $k(t)$ tiene es periódica con periódo $T$ Si se utiliza la sustitución $x=ye^{-\frac{\gamma t}{2}}$, se llega a la ecuación:

\begin{equation}
\ddot{y} +(\frac{k(t)}{m}-\frac{\gamma^2}{4})y=0
\end{equation}

El teorema de Floquet para ecuaciones de segundo orden con coeficientes del tiempo (ver Hanngi para referencia) asegura que esta ecuación tiene dos soluciones:


\begin{equation}
\Xi_1(t) = e^{i\mu t}\phi(t), \quad \Xi_2(t)=\Xi_1(-t)
\end{equation}

Recordando que la función $\phi$ debe tener la misma periodicidad que $k(t)$. Dado que la función cumple con esta condición, es posible realizar una expansión de Fourier \cite{ArfkenMM} de la misma:

\begin{equation}
\phi(t) = \sum_{-\infty}^\infty c_n e^{in\omega t}
\end{equation}

Para fijar los coeficientes se normaliza el Wronskiano tal que:

\begin{equation}
W = \dot{\Xi}_1(t)\Xi_2(t)-\Xi_1(t)\dot{\Xi}_2(t) = 2i
\end{equation}

Lo cual genera la regla de suma:

\begin{equation}
\sum_{-\infty}^\infty c_n^2(\mu + n\omega) = 1
\end{equation}

\section{Caso Cuántico}

En el caso de un Hamiltoniano con dependencia temporal como la vista anteriormente, existe un conjunto completo de soluciones \cite{BarnettSD}:

\begin{equation}
\Ket{\Psi_\alpha (t)} = e^{-i\mu_\alpha t}\Ket{\phi_\alpha t}, \qquad \Ket{\phi_\alpha (t)}=\Ket{\phi_\alpha (t+\tau)}
\end{equation}

Estas soluciones tienen la forma explícita (FALTA REFERENCIA):

\begin{equation}
\Psi_\alpha (x,t) = (\frac{\sqrt{m/\pi\hbar}}{2^\alpha n!E_1^0(t)})^{\frac{1}{2}}(\frac{E_1^0(t)}{E_2^0(t)})^\frac{\alpha}{2}H_\alpha(x\sqrt{\frac{m}{\hbar E_1^0(t) E_2^0(t)}})e^{(ix^2\frac{E_1^0(t)}{2E_2^0(t)})}
\end{equation}

Sin embargo, estas soluciones se comportan de manera análoga a los estados de la base de Foch bajo la acción de los operadores de Floquet:

\begin{align*}\label{FloquetOperators}
\Gamma(t) \Ket{\Psi_\alpha (x,t)} =& \sqrt{\alpha}\Ket{\Psi_{\alpha-1} (x,t)} \\
\Gamma^\dagger(t) \Ket{\Psi_\alpha (x,t)} =& \sqrt{\alpha+1}\Ket{\Psi_{\alpha+1} (x,t)}
\end{align*}

Es importante notar que estos operadores dependen explicitamente del tiempo, en términos de los operadores de momento y posición, pueden expresarse de manera análoga a los operadores de creación y aniquilación del oscilador armónico:

\begin{equation}
\Gamma(t) = \frac{1}{2i}(\hat{x}\dot{E}_1^0(t)\sqrt{\frac{2m}{\hbar}}-\hat{p}E_1^0(t)\sqrt{\frac{\hbar}{2m}})
\end{equation}

Es conveniente entender el origen de estos operadores. Si se comienza con un Hamiltoniano con frecuencia dependiente del tiempo, se puede seguir el procedimiento de \cite{BrownPT}:

\begin{equation}\label{TDHO}
H = \frac{1}{2m}p^2 + \frac{1}{2}k(t)q^2
\end{equation}

Este lleva a la ecuación de movimiento:

\begin{equation}
m\ddot{q}(t) + k(t)q(t) = 0
\end{equation}

Para el operador $q(t)$. Lo que se busca es una transformación unitaria que lleve este problema al problema usual del oscilador armónico en mecánica cuántica. Es decir, se busca $U$ tal que, trabajando en el cuadro de Heisenberg \cite{SakuraiQM}

\begin{align}
\tilde{q}(t) =& U^{-1}(t)q(t)U(t)\\
\tilde{p}(t) =& U^{-1}(t)p(t)U(t)
\end{align}

Y donde entonces el nuevo Hamiltoniano queda dado por:

\begin{equation}
\tilde{H} = H + U^{-1}i\dot{U}
\end{equation}

Para la transformación se elije:

\begin{equation}
U = e^{-i\chi(t)q^2(t)}
\end{equation}

Donde:

\begin{equation}
\chi(t) = \frac{m}{4}(\frac{\dot{f}}{f}+\frac{\dot{f^*}}{f^*})
\end{equation}

Las funciones $f$ son las soluciones al problema clásico correspondiente al Hamiltoniano \ref{TDHO}, que tiene dos soluciones linealmente independientes, pero una es la compleja conjugada de la otra. Bajo esta transformación:

\begin{align}
\tilde{q}(t)=&q(t)\\
\tilde{p}(t)=&p(t)-2\chi(t)q(t)
\end{align}

Utilizando esto se puede escribir el Hamiltoniano en las nuevas coordenas tomando en cuenta que $\ddot{f}= -k(t)f$ y el Wronskiano:

\begin{equation}
 H = \frac{1}{2m}\tilde{p}^2 + \frac{\chi(t)}{m}(\{\tilde{q},\tilde{p}\}) + \frac{mW^2}{|f|^2}k(t)\tilde{q}^2
\end{equation}

Para eliminar el término cruzado se utiliza una segunda transformación:

\begin{equation}
U_2(t)=e^{\frac{i}{4}(\{\tilde{q},\tilde{p}\})ln|f|^2}
\end{equation}

Esto es una transformación de escala que deja como variables finales:

\begin{align}
Q=&U_2^{-1}\tilde{q}U_2 =\frac{1}{|f|}q(t)\\
P=&U_2^{-1}\tilde{p}U_2 = |f|(p-2\chi q) 
\end{align}

En estas variables, el Hamiltoniano es:

\begin{equation}\label{QTDHO}
\tilde{H} = \frac{1}{|f(t)|^2}(\frac{1}{2m}P^2(t)+\frac{1}{2}mW^2Q^2(t))
\end{equation}

Este Hamiltoniano es, salvo por un coeficiente general dependiente del tiempo, el Hamiltoniano usual de oscilador armónico y se puede resolver por medio de operadores de creación y aniquilación. Estos operadores son los operadores de Floquet ya mencionados.

\section{Ecuación Maestra Mejorada: Solución con Base de Decaimiento}

Se busca obtener una mejor ecuación maestra al tomar en cuenta la dependencia temporal del Hamiltoniano antes de realizar la aproximación de Markov \cite{HanngiDQS}. Como el Hamiltoniano es explícitamente dependiente del tiempo, no pueden usarse los operadores usuales de creación y aniquilación, es aquí donde entra la teoría de floquet de la sección anterior. Utilizando dichos operadores, los elementos de matríz del operador de posición son:

\begin{align}
X_{\alpha \beta}(t) =& e^{i(\mu_{\alpha}-\mu_{\beta})}\Bra{\phi_{\alpha}(t)}x\Ket{\phi_{\beta}(t)}\\
=& \sum{k} e^{i\Delta_{\alpha\beta k}t} X_{\alpha\beta k}
\end{align}

Con:

\begin{align*}
X_{\alpha\beta k} =& \frac{1}{T} \int_0^T dt e^{-ik\Omega t}\Bra{\phi_{\alpha}(t)}x\Ket{\phi_{\beta}(t)} \\
\Delta_{\alpha \beta k} =& \mu_{\alpha}-\mu_{\beta} + k\Omega
\end{align*}

Aprovechando la periodicidad de los estados de Floquet para hacer un desarrollo en series de Fourier. Con esto, es posible expresar los elementos de matriz de la forma:

\begin{equation}
X_{\alpha\beta k} = \sqrt{\frac{\hbar}{2m}}(\sqrt{\beta}c_{-k} \delta_{\alpha,\beta-1}+\sqrt{\alpha}c_k\delta_{\alpha,\beta+1})
\end{equation}

Si se regresa ahora a la ecuacion maestra en el límite de acoplamiento débil, asumiendo que el baño es Ohmico, es decir de la forma $I(\omega) = m\gamma \omega$:

\begin{equation}
\dot{\rho} = -\frac{i}{\hbar}[H(t)_{sis},\rho] + \frac{1}{\pi \hbar} \int_{-\infty}^\infty d\omega I(\omega)n_{th}(\omega)\int_0^\infty e^{i\omega t}[\tilde{x}(t-\tau,t)\rho,x] + C.H.
\end{equation}

Aquí $C.H.$ indica el conjugado Hermitiano de la parte disipativa y $n(\omega)$ es la ocupación térmica de los osciladores del baño. Utilizando que los estados de Floquet son la solución a la parte no disipativa, se llega a la ecuación en representación de Floquet:

\begin{align*}
\dot{\rho}_{\alpha \beta} =& \frac{1}{\pi \hbar} \int_{-\infty}^\infty d\omega I(\omega)n_{th}(\omega) \int_0 ^{\infty} d\tau e^{i\omega \tau} \\
 \times & \sum_{\alpha' \beta '} [X_{\alpha \alpha'}(t-\tau)\rho_{\alpha' \beta'}X^* _{\beta \beta'}(t)\\
 -& X^* _{\alpha' \alpha}(t) X_{\alpha' \beta'}(t-\tau)\rho_{\beta' \beta}] + C.H.
\end{align*}

Y se llega a:

\begin{align*}
\dot{\rho}_{\alpha \beta} =& \frac{1}{\hbar} \sum_{\alpha' \beta'}\sum_{k k'}[-I(\Delta_{\alpha' \beta' k'}) n_{th} (\Delta_{\alpha' \beta' k'})\\
\times & e^{i(\Delta_{\alpha' \beta' k'}-\Delta_{\alpha \alpha' k})t} X^*_{\alpha' \alpha k}X_{\alpha' \beta' k'}\rho_{\beta \beta'} + I(\Delta_{\alpha \alpha' k})\\
\times & n_{th}(\Delta_{\alpha \alpha' k}) e^{i(\Delta_{\alpha \alpha' k}-\Delta_{\beta \beta' k})}X_{\alpha \alpha' k} \rho_{\alpha' \beta'}X^*_{\beta \beta' k'}]\\
+& C. H.  
\end{align*}

La función $\Delta_{\alpha \beta k}$ contiene las cuasienergías del sistema sin pérdidas de energía \cite{HanngiDQS}. Se utiliza la aproximación de onda rotante FALTA REFERENCIA., sin embargo en este sistema no se desprecian todos los términos, ya que $(\alpha-\beta,k)=(\alpha'-\beta',k')$ es suficiente para asegurar que $\Delta_{\alpha \beta k}=\Delta_{\alpha' \beta' k'}$, de esta forma se obtiene la ecuación:

\begin{align*}
\dot{\rho}_{\alpha \beta} = \frac{\gamma}{2}&(N+1)[2\sqrt{(\alpha+1)(\beta + 1)}\rho_{\alpha+1,\beta+1} - (\alpha + \beta)\rho_{\alpha \beta}]\\
+& N[2\sqrt{\alpha \beta}\rho_{\alpha-1,\beta-1} - (\alpha + \beta + 2)\rho_{\alpha \beta}]
\end{align*}

Utilizando las propiedades de los operadores definidos en \ref{FloquetOperators} se obtiene la ecuación en términos de operadores:

\begin{align*}
\dot{\rho} = -&\frac{i}{\hbar}[H_S (t),\rho] + \frac{\gamma}{2}(N+1)(2\Gamma\rho\Gamma^\dagger - \Gamma^\dagger\Gamma \rho - \rho \Gamma^\dagger \Gamma )\\
 +& N(2\Gamma^\dagger \rho \Gamma - \Gamma \Gamma^\dagger \rho - \rho \Gamma\Gamma^\dagger)
\end{align*}

Formalmente, la parte disipativa de la ecuación tiene la misma forma que la ecuación que genera la base \ref{DefDB}, así que transformando al cuadro de interacción, donde se toma el Hamiltoniano del sistema como \ref{QTDHO}, el cual puede expresarse como\cite{BrownPT}

\begin{equation}
H_S = \frac{W}{|f(t)|^2}(\Gamma^\dagger \Gamma + \frac{1}{2})
\end{equation}

Dado que los operadores de Floquet conmutan con este Hamiltoniano, se puede realizar la transformación al cuadro de interacción con el operador unitario \cite{SakuraiQM}:

\begin{equation}
 e^{\frac{i}{\hbar}\frac{W}{|f(t)|^2}t(\Gamma^\dagger \Gamma + \frac{1}{2})}
\end{equation}

Lo que lleva a la ecuación:

\begin{equation}
\dot{\rho}=\frac{\gamma}{2}(N+1)(2\Gamma\rho\Gamma^\dagger - \Gamma^\dagger\Gamma \rho - \rho \Gamma^\dagger \Gamma )
 + N(2\Gamma^\dagger \rho \Gamma - \Gamma \Gamma^\dagger \rho - \rho \Gamma\Gamma^\dagger)
\end{equation}

La cual queda formalmente resuelta mediante la base de decaimiento.


\chapter{Enfriamiento Laser}

Se piensa en un sistema que consiste de una cavidad optomecánica, como por ejemplo una cavidad de Fabry-Perot en la cual uno de los dos espejos en los extremos se encuentra a su vez acoplado a un resorte. Este acoplamiento permite la utilización de medios ópticos para controlar el objeto mecánico, las cuales pueden llevar a enfriamiento laser del objeto mecánico\cite{NooshiLC}. En particular, se desea analizar el sistema cuando la frecuencia del oscilador armónico depende del tiempo. Se asume que el oscilador se encuentra acoplado a un único modo forzado de la cavidad el cual tiene frecuencia $\omega_{cav}$. Se asume que el marco de referencia rota con la frecuencia de la bomba. Se modela el sistema mediante el siguiente Hamiltoniano\cite{BarberisLC}:

\begin{equation}
H(t) = H_{cav} + H_{mec}(t) + H_{rad} + B
\end{equation}

En donde:

\begin{align}
H_{cav} =& -\hbar \delta a^\dagger a\\
H_{mec}(t) =& \frac{p^2}{2m} + \frac{1}{2}m \nu (t) x^2\\
H_{rad} =& -\hbar g a^\dagger a x\\
B =& \hbar\frac{\Omega}{2}(a^\dagger + a)
\end{align}

En este caso, $\delta = w_{bomba} - w_{cav}$ representa la diferencia de frecuencias entre la bomba de fotones y la cavidad y $\hbar g$ representa la fuerza de radiación que un fotón aplica sobre el oscilador mecánico sin modulación. Por \ref{QTDHO}, se modela al oscilador mecánico utilizando operadores de floquet:

\begin{equation}
H_{mec}(t) = \frac{W}{|f(t)|^2}(\Gamma^\dagger \Gamma + \frac{1}{2})
\end{equation}

Recordando la definición de los operadores de Floquet \ref{FloquetOperators}, se puede invertir la relación en términos de los operadores $x$ y $p$:


\begin{align*}
\Gamma(t) \Ket{\Psi_\alpha (x,t)} =& \sqrt{\alpha}\Ket{\Psi_{\alpha-1} (x,t)} \\
\Gamma^\dagger(t) \Ket{\Psi_\alpha (x,t)} =& \sqrt{\alpha+1}\Ket{\Psi_{\alpha+1} (x,t)}
\end{align*}

Si se toma la suma y la resta de los operadores:

\begin{align*}
2i(\Gamma (t) + \Gamma ^\dagger (t)) =& (\dot{E} (t) - \dot{E}^* (t)) \sqrt{\frac{2m}{\hbar}}x + (E^* (t) - E (t))\sqrt{\frac{2}{m\hbar}} p \\
2i(\Gamma (t) - \Gamma ^\dagger (t)) =& (\dot{E} (t) + \dot{E}^* (t)) \sqrt{\frac{2m}{\hbar}}x -(E^* (t) + E (t))\sqrt{\frac{2}{m\hbar}} p
\end{align*}

Renombrando los coeficientes esto es simplemente:

\begin{align*}
2i(\Gamma (t) + \Gamma ^\dagger (t)) =& ax + bp \\
2i(\Gamma (t) - \Gamma ^\dagger (t)) =& cx - dp
\end{align*}

Si se toma la primera ecuación multiplicada por $\frac{d}{b}$ y se suma a la segunda, se obtiene:

\begin{align*}
(\frac{da}{b}+c)x =& 2i \frac{d}{b}(\Gamma (t) + \Gamma ^\dagger (t)) + 2i(\Gamma (t) - \Gamma ^\dagger (t)) \\
=&2i[(\frac{d}{b}+1)\Gamma (t)+(\frac{d}{b}-1)\Gamma^\dagger (t)]\\
\therefore x =& \frac{2i[(\frac{d}{b}+1)\Gamma (t)+(\frac{d}{b}-1)\Gamma^\dagger (t)]}{\frac{da}{b}+c}
\end{align*}

O lo que es equivalente:

\begin{equation}
x = 2i \sqrt{\frac{\hbar}{2m}}[(\frac{\alpha (t) +1}{\beta(t)})\Gamma (t) +(\frac{\alpha (t) -1}{\beta(t)})\Gamma^\dagger (t)]
\end{equation}

Donde:

\begin{align}
\alpha(t) =& \frac{d}{b} = \frac{(E^* (t) + E (t))}{(E^* (t) - E (t))} \\
\beta(t) = & \frac{da}{b}+c \\
=& \frac{(E^* (t) + E (t))(\dot{E} (t) - \dot{E}^* (t)) \sqrt{\frac{2m}{\hbar}}}{(E^* (t) - E (t))} + (\dot{E} (t) + \dot{E}^* (t)) \sqrt{\frac{2m}{\hbar}}\\
=&\sqrt{\frac{2m}{\hbar}}\frac{(E^* (t) + E (t))(\dot{E} (t) - \dot{E}^* (t))+(E^* (t) - E (t))(\dot{E} (t) + \dot{E}^* (t))}{(E^* (t) - E (t))}
\end{align}

Con esto, es posible sustituir el operador $x$ en el Hamiltoniano de radiación con operadores de Floquet, lo cual produce un nuevo Hamiltoniano:

\begin{equation}
H(t)_{rad} = -2ig \sqrt{\frac{\hbar^3}{2m}}  a^\dagger a[(\frac{\alpha (t) +1}{\beta(t)})\Gamma (t) +(\frac{\alpha (t) -1}{\beta(t)})\Gamma^\dagger (t)]
\end{equation}



\bibliographystyle{plain}
\bibliography{Bib}


\end{document}