\documentclass[a4paper,10pt]{report}
\usepackage{amsmath}
\usepackage{amssymb}
\usepackage{braket}
\usepackage{fancyvrb}
\usepackage{graphicx}
\usepackage{color}
\usepackage{colortbl}

\usepackage[spanish]{babel}
\selectlanguage{spanish}
\usepackage[utf8]{inputenc}

\DefineVerbatimEnvironment{code}{Verbatim}{fontsize=\small}
\DefineVerbatimEnvironment{example}{Verbatim}{fontsize=\small}


% Title Page
\title{Tesis de Maestría}
\author{Pablo Yanes}


\begin{document}
\maketitle

\begin{abstract}

\end{abstract}

\chapter{Sistemas Cuánticos Abiertos y Ecuaciones Maestras}

El objetivo es encontrar la evolución temporal de un sistema cuántico abierto, los cuales son modelados por ecuaciones del tipo:

\begin{equation}\label{MasterEq}
\dot{\rho} = \frac{1}{i\hbar}[H,\rho] + L\rho
\end{equation}

Donde $\rho$ es la matriz densidad del sistema, $H$ es el Hamiltoniano que modela las partes del sistema donde no hay pérdidas de energía y $L$ es el operador de Lindblat, el cual corresponde a la parte abierta del sistema. En la mecánica cuántica cerrada, el análisis se limíta al operador Hamiltoniano el cual es un operador auto adjunto. En ese caso, es suficiente resolver la ecuación de Schröedinger estacionaria:

\begin{equation}
H\Psi = E\Psi
\end{equation}

cuya solución formal es:

\begin{equation}
\Psi(t) = e^{\frac{-iHt}{\hbar}}\Psi(0)
\end{equation}

La función $\Psi(0)$ puede expresarse como una  combinación lineal de las funciones propias de $H$. Al hacer esto, la ecuación anterior toma la forma:

\begin{equation}
\Psi(t) = \sum_j c_j e^{\frac{-iE_jt}{\hbar}}\phi_j(0)
\end{equation}

Los coeficientes $c_j$ son las proyecciones de la función $\Psi(0)$ sobre las funciones propias de $H$

Se busca una solución equivalente para la ecuación \eqref{MasterEq}. Es posible transformar dicha ecuación al cuadro de interacción, donde solo importa la parte abierta del problema. En este cuadro, se busca una solución del tipo:

\begin{equation}
\rho(t) = \sum_{\Gamma} c_{\Gamma} e^{\Gamma t}P_{\Gamma} 
\end{equation}

Los coeficientes ahora son respecto a las funciones propias de $L$. Es importante notar que $L$ no es necesariamente un operador autoadjunto, por lo que no se puede garantizar que sus funciones propias formen una base completa del espacio ni que sus valores propios sean reales. También es importante notar  que $Psi$ es un vector de estado que pertenece al espacio de Hilbert mientras que $rho$ es un operador que pertenece al espacio de Liouville. Por esto, se dice que $L$ es un superoperador. La base que se obtiene de esta forma se conoce como \textit{damping base}. Una de las diferencias importantes con el caso de mecánica cuántica estandard es que como $L$ no es autoadjunto, los vectores que se obtienen al aplicar por la izquierda y por la derecha no son simplemente complejos conjugados uno del otro, se conocen como duales y son ortogonales los unos a otros bajo un producto definido mediante la traza:

\begin{equation}
Tr\{\check{P_{\Gamma'}}P_\Gamma\} = \delta_{\Gamma'\Gamma}
\end{equation}

Esto se puede emplear para obtener las constantes de la expansión.

\section{Ejemplo: Átomo de dos Niveles}

Antes de intentar resolver sistemas más complejos, es ilustrativo tratar el caso relativamente sencillo del átomo de niveles. A este sistema le corresponde el operador de Linblad:

\begin{align}\label{TwoLevel}
 L_{\sigma}\rho = 
 &-\frac{B}{2}(1-s)[\sigma_+\sigma_-\rho + \rho\sigma_+\sigma_--2\sigma_-\rho\sigma_+ ]\nonumber\\
 &-\frac{B}{2}s[\sigma_-\sigma_+\rho + \rho\sigma_-\sigma_+-2\sigma_+\rho\sigma_- ]\nonumber\\
 &-\frac{2C-B}{4}[\rho - \sigma_z\rho\sigma_z]
\end{align}

Es convenitente utilizar notación de Dirac en este tratamiento. Todos los operadores involucrados pueden verse como matrices de dos por dos. $\sigma_z$ es la matriz de Pauli correspondiente al eje z y:

\begin{equation}
\sigma_{\pm} = \sigma_x \pm \sigma_y = (\sigma_{\mp}) ^\dagger
\end{equation}

El método para encontrar los vectores y valores propios de $L$ consiste en proponer una forma adecuada para los vectores, aplicar $L$ a la propuesta y resolver las ecuaciones resultantes. En este caso se propone:

\begin{equation}
\rho_\sigma = \alpha\Ket{+}\Bra{+}+\beta\Ket{+}\Bra{-}+\gamma\Ket{-}\Bra{+}+\delta\Ket{-}\Bra{-}
\end{equation}

Ya que se conoce la forma en la cual las matrcies $\sigma$ operan sobre estos elementos:

\begin{align*}
&&\sigma_{+}\Bra{-} =& \Bra{+} \quad \sigma_{+}\Bra{+} = 0 \\
&&\sigma_{-}\Bra{+} =& \Bra{-} \quad \sigma_{-}\Bra{-} = 0 \\
&&\sigma_{z}\Bra{+} =& \Bra{+} \quad \sigma_{z}\Bra{-} = -\Bra{-}
\end{align*}

Al sustituir $\rho_{\sigma}$ en \eqref{TwoLevel} se obtiene, para el primer término de $L$:

\begin{equation}
-\frac{B}{2}(1-s)[2\alpha\Ket{+}\Bra{+}+\beta\Ket{+}\Bra{-}+\gamma\Ket{-}\Bra{+}-2\alpha\Ket{-}\Bra{-}]
\end{equation}

El segundo término es:

\begin{equation}
-\frac{B}{2}(s)[2\delta\Ket{+}\Bra{+}+\beta\Ket{+}\Bra{-}+\gamma\Ket{-}\Bra{+}-2\delta\Ket{-}\Bra{-}]
\end{equation}

Y el tercero:

\begin{equation}
-\frac{2C-B}{4}[2\beta\Ket{+}\Bra{-}+2\gamma\Ket{-}\Bra{+}]
\end{equation}

Esto se inserta en:

\begin{equation}
L\rho = \lambda\rho
\end{equation}

Y dado que cada uno de los cuatro componentes de $\rho$ son linealmente independientes, esto genera cuatro ecuaciones distintas:

\begin{align}
B(s(\alpha+\delta)-\alpha)\Ket{+}\Bra{+} =& \lambda\alpha\Ket{+}\Bra{+}\\
B(-s(\alpha+\delta)+\alpha)\Ket{-}\Bra{-} =& \lambda\delta\Ket{-}\Bra{-}\\
-\beta C \Ket{+}\Bra{-} =& \lambda \beta \Ket{+}\Bra{-} \\
-\gamma C \Ket{-}\Bra{+} =& \lambda \gamma \Ket{-}\Bra{+}
\end{align}

De las cuales se obtienen los cuatro vectores propios:

\begin{align}
L_\sigma \sigma_0 =& 0 \\
L_\sigma \sigma_z =& -B \sigma_z \\
L_\sigma \sigma_\pm =& -C \sigma_\pm
\end{align}

donde:

\begin{equation}
\sigma_0 = \frac{1}{2}[I + (2s-1)\sigma_z]
\end{equation}

Las constantes en \eqref{MasterEq} son todas positivas por lo que todos los valores propios son no positivos. Esto es importante, ya que al hacer tender el tiempo a infinito solo el término que corresponde al valor propio 0 sobrevive, los demás términos decaen exponencialmente con el tiempo. Esto lleva a identificar al término con valor propio 0 como el estado estacionario del sistema. A continuación se trata el caso del oscilador armónico.

\section{Caso Oscilador Armónico: Base de Decaimiento}

Se busca modelar interacciones que involucran un capo electromagnético. Este se modela simplemente como una suma infinita de modos de oscilador armónico. Este sistema en particular corresponde a un operador de Lindblat de la forma:

\begin{align}\label{EMField}
L_a \rho =& - \frac{A}{2}(\nu + 1)[a^\dagger a\rho + \rho a^\dagger a -2a\rho a^\dagger] \nonumber \\
 &- \frac{A}{2}(\nu)[ aa^\dagger\rho + \rho  aa^\dagger -2a^\dagger\rho a]
\end{align}

Esto modela un campo electromagnético en una cavidad con acoplamiento a un reservorio térmico con un número $\nu$ promedio de fotones térmicos. $A,\nu \geq 0$. La elección de un ansatz para este sistema es mucho más compleja, ya que ahora la base estados sobre la cual actúan los operadores que forman $L$ es infinita.  En base a lo visto en la sección anterior, se propone una base que acople estados de distinta energía:

\begin{equation}
 \rho^l_n = \sum_{n=0}^\infty Q_n^l\Ket{n}\Bra{n+l}
\end{equation} 

El objetivo de nuevo es sustituir el ansatz en \eqref{EMField} para obtener un sistema de ecuaciones para los valores y vectores propios. Se trabaja primero con el primer término de \eqref{EMField}, donde se ignora el coeficiente exterior por brevedad:

\begin{align*}
=&[a^\dagger a\sum_{n=0}^\infty Q_n^l\Ket{n}\Bra{n+l} + \sum_{n=0}^\infty Q_n^l\Ket{n}\Bra{n+l} a^\dagger a -2a\sum_{n=0}^\infty Q_n^l\Ket{n}\Bra{n+l} a^\dagger]\\
=&[\sum_{n=0}^\infty Q_n^la^\dagger a\Ket{n}\Bra{n+l} + \sum_{n=0}^\infty Q_n^l\Ket{n}\Bra{n+l} a^\dagger a -2\sum_{n=0}^\infty Q_n^la\Ket{n}\Bra{n+l} a^\dagger]\\
=&[\sum_{n=0}^\infty Q_n^ln\Ket{n}\Bra{n+l} + \sum_{n=0}^\infty Q_n^l\Ket{n}\Bra{n+l} (n+l) -2\sum_{n=0}^\infty Q_n^l\sqrt{n}\Ket{n-1}\Bra{n+l-1}\sqrt{n+l}]\\
=&[\sum_{n=0}^\infty Q_n^l(2n+l)\Ket{n}\Bra{n+l} -2\sum_{n=0}^\infty\sqrt{n(n+l)} Q_n^l\Ket{n-1}\Bra{n+l-1}]\\
\end{align*}

El segundo término actúa de una forma completamente análoga y se obtniene:

\begin{equation*}
=[\sum_{n=0}^\infty Q_n^l(2n+l+2)\Ket{n}\Bra{n+l} -2\sum_{n=0}^\infty\sqrt{(n+1)(n+l+1)} Q_n^l\Ket{n+1}\Bra{n+l+1}]
\end{equation*}

Como $p^l_n$  es un vector propio los términos anteriores deben de poder igualarse a un valor propio por $\rho^l_n$:

\begin{align}
\lambda_n^l \rho_n^l=& \sum_{n=0}^\infty Q_n^l \{ -\frac{A}{2}(\nu+1)[(2n+l)\Ket{n}\Bra{n+l} -2\sqrt{n(n+l)} Q_n^l\Ket{n-1}\Bra{n+l-1}] \nonumber \\
-&\frac{A}{2}(\nu)[(2n+l+2)\Ket{n}\Bra{n+l}-2\sqrt{(n+1)(n+l+1)} Q_n^l\Ket{n+1}\Bra{n+l+1}]\}
\end{align}

Si en el segundo término se recorre el índice de suma hacia arriba por uno y en el cuarto hacía abajo por uno, se puede utilizar independencia lineal respecto al elemento $\Ket{n}\Bra{n+l}$ para obtener una ecuación para cada elemento de la suma. Se obtiene que los coeficientes deben cumplir con:

\begin{align}
A&(\nu+1) Q_{n+1}^l \sqrt{(n+1)(n+l+1) }+ A(\nu) Q_{n-1}^l\sqrt{n(n+l)}\\ \nonumber
=&[\lambda_n^l+A\nu(2n+l+2)+\frac{A}{2}(2n+l)]Q_{n}^l
\end{align}

Esto es una relación de recuerrencia similar a la que se encuentra en los polinomios generalizados de Laguerre. En \cite{EnglertDB} se propone una solución:

\begin{equation}\label{Englert1993}
\rho_\lambda (a,a^\dagger) = :f(aa^\dagger):a^l
\end{equation}

Se puede en $a^l$ como

\begin{equation}
\sum_{n=0}^\infty C_n^l\Ket{n}\Bra{n+l}
\end{equation}

Es entonces posible relacionar este ansatz con el de Englert, lo cual lleva a su solución para $\rho_n^l$

\begin{align}\label{DefDB}
&a^{\dagger l}\frac{(-1)^n}{(\nu+1)^{l+1}}:L_n^l[\frac{a^\dagger a}{\nu+1}]e^{-[\frac{a^\dagger a}{\nu+1}]}:\quad l \geq 0 \\
&\frac{(-1)^n}{(\nu+1)^{|l|+1}}:L_n^{|l|}[\frac{a^\dagger a}{\nu+1}]e^{-[\frac{a^\dagger a}{\nu+1}]}:a^{|l|}\quad l \leq 0
\end{align}

con autovalores

\begin{equation}
\lambda_n^l = -A[n + \frac{|l|}{2}]
\end{equation}

con las condiciones

\begin{equation}
n=0,1,2...,\qquad l = 0,\pm 1, \pm 2,... 
\end{equation}

A continuación se muestra de forma explicita como calcular estos elementos. 

\subsection{Estado Base}

Este es el estado con valor propio 0, lo cual corresponde a $n=0, \qquad l=0$. En este caso, la solución tiene la forma

\begin{equation}
\frac{1}{(\nu+1)}:L_0^0[\frac{a^\dagger a}{\nu+1}]e^{-[\frac{a^\dagger a}{\nu+1}]}:
\end{equation}

ya que $L_0^0 = 1$ \cite{ArfkenMM} si se desarrolla la exponencial en una serie infinita se tiene, aplicando el ordenamiento normal:

\begin{equation}
\sum_{n=0}^\infty \frac{-(a^\dagger)^n (a)^n}{n!(\nu+1)^n}
\end{equation}

si se sustituye esto en la expresión para el estado base, junto con la relación de cerradura para la base de Foch se llega a \cite{EnglertDB}

\begin{equation}
\rho_0^0 = \frac{1}{1+\nu} [\frac{\nu}{1+\nu}]^{a^\dagger a}
\end{equation}

El cual corresponde a un estado térmico. El estado base es el estado estacionario del sistema.

\section{Estados Duales}

La base de decaimiento \cite{EnglertDB} definida en \ref{DefDB} corresponde a eigenvalores:

\begin{equation}
 -A[n+\frac{|l|}{2}]
\end{equation}

Lo cual corresponde a una doble degeneración $\pm l$. Antes de poder realizar la expansión de estados arbitrarios en esta base es necesario conocer los eigenestados del operador dual, los cuales son aquellos que cumplen la condición\cite{EnglertDB}:

\begin{equation}
\bar{\rho}_\gamma L = \gamma\bar{\rho}_\gamma
\end{equation}

Es decir, los eigenestados \textit{izquierdos} del operador de Lindblat. Los estados duales cumplen con:

\begin{equation}
Tr[\bar{\rho}_{\gamma}\rho_{\gamma'}] = \delta_{\gamma\gamma'}
\end{equation}

Donde la barra denota al estado dual y $\gamma$ denota algún eigenvalor particular del problema. Bajo este producto es posible entonces resolver las ecuaciones para los coeficientes de una expansión de un estado particular en estados de la base de decaimiento:

\begin{equation}
\Psi = \sum_{\gamma}\bar{c}_\gamma \rho_\gamma
\end{equation}

Entonces:

\begin{equation}
\bar{c}_\gamma = Tr[\Psi\bar{\rho}_{\gamma}]
\end{equation}

Con esto la evolución temporal del sistema queda determinada:

\begin{equation}
\Psi(t) = \sum_\gamma \bar{c}_\gamma e^{\frac{-i}{\hbar} \gamma t} \rho_{\gamma}
\end{equation}

\subsection{Estados Duales: Forma Explícita}

Ya que es necesario tomar trazas sobre los estados duales, es conveniente encontrar tanto su forma explícita en términos de operadores de creación y aniquilación, así como una relación de recurrencia que permita relacionar un estado a los estados de menor número $n$. Los polinomios asociados de Laguerre, en general, son de la forma\cite{ArfkenMM}

\begin{equation}\label{DefLaguerre}
L_n^k(x) = \sum_{m=0}^n (-1)^m \frac{(n+k)!}{(n-m)!(k+m)!m! x^m}
\end{equation}

Recordando que en este caso la variable $x$ corresponde a $\frac{a^\dagger a}{\nu}$, y que los estados se encuentran multiplicados por un coeficiente:

\begin{equation}
(\frac{-\nu}{1+\nu})^n \frac{n!}{(n+k)!}
\end{equation}

Se llega a la expresión general, la cual toma en cuenta el ordenamiento normal y los operadores de aniquilación a la derecha del mismo:

\begin{equation}\label{DefDual}
\sum_{m=0}^n \frac{(-\nu)^{n-m}}{(\nu+1)^n} \frac{n!}{(n-m)!(m+k)!m!} a^{\dagger m}a^{m+k}
\end{equation}

Basandose en \ref{DefDual}, se puede ver que para cualquier valor de $n$ y $k \\geq 0$ el primer término de la suma, es decir el correspondiente a $m=0$ es:

\begin{equation}
(\frac{-\nu}{1+\nu})^n \frac{a^k}{k!}
\end{equation} 

El último término de la suma, el correspondiente a $m=n$, es:

\begin{equation}
\frac{1}{(\nu+1)^n} \frac{1}{(n+k)!} a^{\dagger n}a^{n+k}
\end{equation}

Si se desea una representación de estos estados en la base de Foch, se puede recurrir a:

\begin{equation}
I = \sum^\infty _n \Ket{n}\Bra{n}
\end{equation}

Y aplicar todos los operadores desde la derecha. En el caso del primer término ya antes mencionado, se obtiene:

\begin{equation}
 (\frac{-\nu}{1+\nu})^n \sqrt{\frac{n!}{(n-k)!}}\sum^\infty _{l=k} \Ket{l-k}\Bra{l}
\end{equation}

Los estados con número de ocupación menor a $k$ se anulan por la definición del operadore de aniquilación. En el caso del último término de la suma, se obtiene:

\begin{equation}
(\frac{1}{1+\nu})^n \frac{n!}{(n-k)!}\sum^\infty _{l=n+k} \Ket{l-k}\Bra{l}
\end{equation}

Puede verse que ambos términos involucran la misma diferencia en el número de ocupación del ket y del bra, pero la suma correspondiente al último término no cubre tantos estados, pues ahora todo estado con número menor a $n+k$ es anulado.

\subsection{Relación de Recurrencia para Estados Duales}

A fin de obtener una relación de recurrencia para los estados duales, es posible basarse en una de las muchas relaciones de recurrencia de los polinomios asociados de Laguerre \cite{ArfkenMM}:

\begin{equation}
L_n^{k+1}(x) = \frac{n-x}{n}L_{n-1}^{k+1}(x) + \frac{n+k}{n}L_{n-1}^{k}(x)
\end{equation}

Si se sustituye esto en la expresión para el estado dual correspondiente se tiene, recordando que se utiliza $x$ para la variable únicamente por simplicida:

\begin{align}
\hat{\rho}_n ^{k+1} =& [\frac{-\nu}{1+\nu}]^n \frac{n!}{(n+k+1)!} :L_n^{k+1}:a^{k+1}\\
=&[\frac{-\nu}{1+\nu}]^n\frac{n!}{(n+k+1)!} :\frac{n-x}{n}L_{n-1}^{k+1}(x) + \frac{n+k}{n}L_{n-1}^{k}(x):a^{k+1}\\
=&[\frac{-\nu}{1+\nu}]^n\frac{n!}{(n+k+1)!}[:L_{n-1}^{k+1}(x)+L_{n-1}^{k}(x):+\frac{1}{n}:-xL_{n-1}^{k+1}(x)+L_{n-1}^{k}(x):]a^{k+1}\\
=&[(\frac{-\nu}{1+\nu})\frac{1}{(n+k+1)}+ \frac{(n+k)}{-\nu(\nu+1)}a^\dagger \bullet a]\hat{\rho}_{n-1} ^{k+1} + [(\frac{-\nu}{1+\nu}) (n+1)(n+k+1)(n+k)]\hat{\rho}_{n-1} ^{k}
\end{align}

Donde $\bullet$ indica que el estado dual va entre el operador de creación y el de aniquilación.

\chapter{Deducción de Ecuacion Maestra Mediante Operadores de Proyección}

Se busca obtener la ecuación maestra correspondiente a un sistema sujeto a interacción con un baño térmico:

\begin{equation}
\hat{H} = \hat{H}_{Sis}+\hat{H}_{Int}+\hat{H}_B
\end{equation}

Lo cual implica una ecuación de evolución temporal para $\rho$ de la forma:

\begin{equation}\label{LMeq}
\dot{\rho}_{total} = (L_{Sis}+L_{Int}+L_B)\rho
\end{equation}

Donde:

\begin{equation}\label{defL}
L\rho(t) = \frac{i}{\hbar}[\hat{H},\rho(t_0)]
\end{equation}

Se busca una solución para el operador densidad únicamente del sistema, por lo que se busca una ecuación para el operador total después de tomar la traza sobre las variables del baño:

\begin{equation}
\rho(t) \equiv Tr_B\{\rho_{total}(t)\}
\end{equation}

Se define un operador de proyección \cite{ZollerQN} que opere sobre $ \rho $:

\begin{equation}
P\rho_t = Tr_B\{\rho_{t}(t)\}\otimes \rho_B
\end{equation}

Y un operador:

\begin{equation}
Q = \mathbf{1}-P
\end{equation}

Tal que:

\begin{align}
\rho_t =& P\rho_t + Q\rho_t \\
\equiv & v(t)+u(t)
\end{align}

Se emplean las siguientes propiedades:

\begin{enumerate}
\item $ PL_{sis} = L_{sis}P$\qquad Ya que $ P $ y $ L_{sis} $ operan en espacios distintos.

\item $ PL_{B} = L_{B}P = 0 $\qquad Por conservación de la probabilidad.

\item $PL_{int}P=0$ \qquad Se asume que la interacción no tiene términos diagonales en las variables del baño

\item $P^2 = P \quad Q^2 = Q$ \qquad $P$ y $Q$ son proyectores
\end{enumerate}

Por convieniencia se trabaja con la transformada de Laplace [citar arfken] de las proyecciones $v(t)$ y $u(t)$:

\begin{equation}
\tilde{v}(t) = \int_0^\infty e^{-st}v(t)dt \qquad \tilde{u}(t) = \int_0^\infty e^{-st}u(t)dt
\end{equation}

Dado que las dos proyecciones se encuentran es espacios distintos, se puede sustituir $\rho$ por ellas en \eqref{LMeq} y resolver por separado para cada proyección. En el caso de $v(t)$ se tiene, antes de utilizar la transformada de Laplace, al aplicar el proyector $P$ a \eqref{LMeq}:

\begin{align*}
P\dot{p}(t)_t =& P(L_{Sis}+L_{Int}+L_B)\rho \\
 \dot{v}(t) =& (PL_{Sis}+PL_{Int}+PL_B)\rho \\
 =&(L_{Sis}P+PL_{Int}+\underbrace{PL_B}_0)\rho \\
 =& L_{sis}P\rho_t + PL_{int}\rho_t \\
 =& L_{sis}v + \underbrace{PL_{int}(P}_0+Q)\rho_t \\
 =& L_{sis}v + PL_{int}Q\rho_t \\
 =& L_{sis}v + PL_{int}u 
\end{align*}

De manera análoga se obtiene una ecuación para $u$ al proyectar con $Q$:

\begin{align*}
Q\dot{p}(t)_t =& Q(L_{Sis}+L_{Int}+L_B)\rho \\
 \dot{u}(t) =& (QL_{Sis}+QL_{Int}+QL_B)\rho \\
 =&(L_{Sis}Q+QL_{Int}+L_BQ)\rho \\
 =& L_{sis}Q\rho_t + L_{B}Q\rho_t + QL_{int}\rho_t \\
 =& L_{sis}u + L_{B}u + QL_{int}(Q+P)\rho_t  \\
 =& L_{sis}u + L_{B}u + QL_{int}v+QL_{int}u\rho_t \\
 =& (L_{Sis}+L_B+QL_{Int})u +QL_{Int}v 
\end{align*}

Al aplicar la transformada de Laplace a la ecuación para $v$ se obtiene:

\begin{equation}\label{eqv}
s\tilde{v}(s)-\tilde{v}(0) = L_{sis}\tilde{v}(s) + PL_{int}\tilde{u}(s)
\end{equation}

Y en el caso de $u$:

\begin{equation}\label{equ}
s\tilde{u}(s)-\tilde{u}(0) = (L_{Sis}+L_B+QL_{Int})\tilde{u}(s) + QL_{Int}\tilde{v}(s)
\end{equation}

Se resuelve \eqref{equ} en términos de $v$:

\begin{equation}
\tilde{u}(s)=\frac{QL_{Int}\tilde{v}(s)+\tilde{u}(0)}{s-\underbrace{(L_{Sis}+L_B+QL_{Int})}_\alpha}
\end{equation}

Y se sustituye el resultado en \eqref{eqv}:

\begin{equation}
s\tilde{v}(s)-(\tilde{v}(0)+PL_{int}\alpha\tilde{u}(0))=(L_{sis}+PL_{int}\alpha QL_{int})\tilde{v}(s)
\end{equation}

En este punto se toma la aproximación de aclopamiento débil:

\begin{equation}
L_{int} \rightarrow \Xi L_{int}
\end{equation}

Los términos asociados a $\tilde{u}0$ son únicamente correciones a la condición inicial, por lo que se desprecian. Tomando esto en cuenta, se obtiene:

\begin{equation}
s\tilde{v}(s)-\tilde{v}(0) = (L_{sis}+\Xi^2\frac{PL_{int}QL_{int}}{s-L_{sis}-L_{B}})s\tilde{v}(s)
\end{equation}

Y al tomar la transformada inversa de Laplace:

\begin{equation}\label{eqProyectores}
\dot{v}(t)=L_{sis}v(t) + \Xi^2 PL_{int} \int_{0}^{\infty}d\tau e^{(L_{sis}+L_{B})\tau}QL_{int}v(t-\tau)
\end{equation}

Se busca obtener una ecuación en términos de los operadores del Hamiltoniano \cite{ZollerQN},se procede con el segundo término de la ecuación \eqref{eqProyectores}:

\begin{align*}
QL_{int}v=&(1-P)L_{int}v\\
=& (1-P)L_{int}P\rho \\
=& L_{int}P\rho \\
=& L_{int}v
\end{align*}

Con esto el segundo término de la ecuación \eqref{eqProyectores}, y tomando en cuenta la definición de los términos $L$ dada en \eqref{defL} y el lema de Baker-Campbell-Housedorff \cite{SakuraiQM}

\begin{align*}
=& \Xi^2 PL_{int} \int_{0}^{\infty}d\tau e^{(L_{sis}+L_{B})\tau}L_{int}v(t-\tau)\\
=& \Xi^2 PL_{int} \int_{0}^{\infty}d\tau e^{\frac{-i}{\hbar}H_B\tau}e^{\frac{-i}{\hbar}H_s\tau}L_{int}v(t-\tau)e^{\frac{i}{\hbar}H_B\tau}e^{\frac{i}{\hbar}H_s\tau}\\
=&\Xi^2 PL_{int} \int_{0}^{\infty}d\tau L_{int}v(t-\tau) \\
=&\Xi^2 PL_{int} \int_{0}^{\infty}d\tau[H_{int}^*,P\rho(t-\tau)] \\
=&\Xi^2 P[H_{int},\int_{0}^{\infty}d\tau[H_{int}^*,tr_B(\rho(t-\tau))\otimes\rho_B]]
\end{align*}

En este punto se realiza la aproximación de Markov \cite{ZollerQN}, al hacer la sustitución:

\begin{equation}
t-\tau \rightarrow t
\end{equation}

Y se traza toda la ecuación sobre las variables del baño. Con esto se llega a la ecuación:

\chapter{Solución de la Ecuación Maestra Mediante Estados de Floquet}

Para resolver el problema correspondiente a un oscilador armónico amortiguado con frequencia dependiente del tiempo, se utiliza la teoría de Floquet \cite{WardFT}y se busca una forma de la ecuación maestra (eq capitulo anterior) expresada mediante operadores de Floquet, que se definirán más adelante.

\section{Teoría de Floquet}

Sea un problema del tipo:

\begin{equation}
x' = A(t)x
\end{equation}

Donde la función $A(t)$ es periódica con periodicidad $\tau$. En este caso el teorema de Floquet dice que la solución no necesariamnte es periódica pero debe tener la forma:

\begin{equation}\label{FloquetForm}
x(t)=e^{\mu t}p(t)
\end{equation}

Se considera que $x(t)$ es un vector de $n$ dimensiones y que $A(t)$ es una matriz de $n\times n$. Existen $n$ valores distintos de $\mu$ tales que:

\begin{equation}
e^{\mu_1 }...e^{\mu_n}=e^{\int_0^\tau tr[A(t)]}
\end{equation}

Los valores de $\mu$ se conocen como exponentes característicos. Es importante notar que la función $p(t)$ es periódica con período $\tau$. Los coeficientes $\mu$ son, en general complejos.

\section{Estados de Floquet en Mecánica Cuántica}

Se busca estudiar Hamiltonianos del tipo:

\begin{equation}
H(t)=H(t+\tau)
\end{equation}

El hecho de que el Hamiltoniano sea simétrico respecto a (ciertas) traslaciones en el tiempo, permite el uso del formalismo de Floquet \cite{HanngiDQS}. Se asume que la dependencia temporal puede ser vista como una perturbación sobre un Hamiltoniano original:

\begin{equation}
H(x,t)=H_0(x)+V(x,t) \qquad V(x,t)=V(x,t+\tau)
\end{equation} 

También se asume que el Hamiltoniano no perturbado poseé un conjunto completo de eigenfuciones $\{\phi_n\}$ con eigenvalores correspondientes $E_n$. En este caso, la ecuación de Schr\"{o}dinger tiene la forma:

\begin{equation}\label{SchrodingerEQ}
-i\hbar\dot{\Psi}(x,t) = H(x,t)\Psi(x,t)
\end{equation}

El problema admite una solución del tipo visto en la sección anterior:

\begin{equation}
\Psi_n(x,t) = e^{(\frac{-i}{\hbar}\mu_nt)}\Phi_n(x,t)
\end{equation}

Como se mencionó en la sección anterior, $\mu$ en general es un número complejo, lo cual puede llevar a soluciones inestables. En este caso $\Phi_n(x,t)$ es la función que contiene la periodícidad en el tiempo. Sustituir la solución en la ecuación \ref{SchrodingerEQ} genera una ecuación para las funciones periódicas:

\begin{equation}
H(x,t)\Phi_n(x,t)=\Xi_n\Phi_n(x,t)
\end{equation}

\section{Oscilador Armónico Dependinte del Tiempo: Solución Mediante Formalismo de Floquet}

En el caso clásico \cite{HanngiFM} se tiene, para un oscilador armónico unidimensional con frecuencia dependiente del tiempo y el cual experimenta una fuerza disipativa dependiente de la velocidad, que la posición cumple:

\begin{equation}
\ddot{x}+\gamma\dot{x}+\frac{k(t)}{m}x=0
\end{equation}

Se asume que la función $k(t)$ tiene es periódica con periódo $T$ Si se utiliza la sustitución $x=ye^{-\frac{\gamma t}{2}}$, se llega a la ecuación:

\begin{equation}
\ddot{y} +(\frac{k(t)}{m}-\frac{\gamma^2}{4})y=0
\end{equation}

El teorema de Floquet para ecuaciones de segundo orden con coeficientes del tiempo (ver Hanngi para referencia) asegura que esta ecuación tiene dos soluciones:


\begin{equation}
\Xi_1(t) = e^{i\mu t}\phi(t), \quad \Xi_2(t)=\Xi_1(-t)
\end{equation}

Recordando que la función $\phi$ debe tener la misma periodicidad que $k(t)$. Dado que la función cumple con esta condición, es posible realizar una expansión de Fourier \cite{ArfkenMM} de la misma:

\begin{equation}
\phi(t) = \sum_{-\infty}^\infty c_n e^{in\omega t}
\end{equation}

Para fijar los coeficientes se normaliza el Wronskiano tal que:

\begin{equation}
W = \dot{\Xi}_1(t)\Xi_2(t)-\Xi_1(t)\dot{\Xi}_2(t) = 2i
\end{equation}

Lo cual genera la regla de suma:

\begin{equation}
\sum_{-\infty}^\infty c_n^2(\mu + n\omega) = 1
\end{equation}

\section{Caso Cuántico}

En el caso de un Hamiltoniano con dependencia temporal como la vista anteriormente, existe un conjunto completo de soluciones \cite{BarnettSD}:

\begin{equation}
\Ket{\Psi_\alpha (t)} = e^{-i\mu_\alpha t}\Ket{\phi_\alpha t}, \qquad \Ket{\phi_\alpha (t)}=\Ket{\phi_\alpha (t+\tau)}
\end{equation}

Estas soluciones tienen la forma explícita (FALTA REFERENCIA):

\begin{equation}
\Psi_\alpha (x,t) = (\frac{\sqrt{m/\pi\hbar}}{2^\alpha n!E_1^0(t)})^{\frac{1}{2}}(\frac{E_1^0(t)}{E_2^0(t)})^\frac{\alpha}{2}H_\alpha(x\sqrt{\frac{m}{\hbar E_1^0(t) E_2^0(t)}})e^{(ix^2\frac{E_1^0(t)}{2E_2^0(t)})}
\end{equation}

Sin embargo, estas soluciones se comportan de manera análoga a los estados de la base de Foch bajo la acción de los operadores de Floquet:

\begin{align*}
\Gamma(t) \Ket{\Psi_\alpha (x,t)} =& \sqrt{\alpha}\Ket{\Psi_{\alpha-1} (x,t)} \\
\Gamma^\dagger(t) \Ket{\Psi_\alpha (x,t)} =& \sqrt{\alpha+1}\Ket{\Psi_{\alpha+1} (x,t)}
\end{align*}

Es importante notar que estos operadores dependen explicitamente del tiempo, en términos de los operadores de momento y posición, pueden expresarse de manera análoga a los operadores de creación y aniquilación del oscilador armónico:

\begin{equation}
\Gamma(t) = \frac{1}{2i}(\hat{x}\dot{E}_1^0(t)\sqrt{\frac{2m}{\hbar}}-\hat{p}E_1^0(t)\sqrt{\frac{\hbar}{2m}})
\end{equation}

\section{Ecuación Maestra Mejorada: Solución con Base de Decaimiento}

La ecuación maestra a resolver es:


\bibliographystyle{plain}
\bibliography{Bib}


\end{document}