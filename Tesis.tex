\documentclass[a4paper,10pt]{report}
\usepackage{amsmath}
\usepackage{amssymb}
\usepackage{braket}
\usepackage{fancyvrb}
\usepackage{graphicx}
\usepackage{color}
\usepackage{colortbl}

\usepackage[spanish]{babel}
\selectlanguage{spanish}
\usepackage[utf8]{inputenc}

\DefineVerbatimEnvironment{code}{Verbatim}{fontsize=\small}
\DefineVerbatimEnvironment{example}{Verbatim}{fontsize=\small}


% Title Page
\title{Tesis de Maestría}
\author{Pablo Yanes}
\makeindex

\begin{document}


\begin{abstract}
Se modela un sistema compuesto de un oscilador optomecánico formado por un tambor mecánico cuya frecuencia natural depende del tiempo acoplado a una cavidad óptica mediante presión de radiación. El tambor mecánico se modela como un oscilador armónico con frecuencia natural dependiente del tiempo mediante el formalismo de Floquet. Se obtiene una ecuación maestra mejorada para el caso general, y se obtiene una aproximación adiabática para el caso de un tambor cuya frecuencia presenta oscilaciones periódicas pequeñas en torno a una frecuencia central. Se obtiene una expresión para la ocupación promedio de excitaciones para el tambor mecánico en este caso específico.
\end{abstract}

\tableofcontents

\chapter{Introducción}

Se estudió el enfriamiento de un sistema mecánico macroscópico mediante métodos ópticos, cuando la frecuencia natural depende del tiempo. Específicamente, se estudió el enfriamiento de un tambor mecánico acoplado al campo electromagnético de una cavidad óptica. Típicamente \cite{BarberisLC}, la interacción entre ambos sistemas es demasiado débil para que el enfriamiento sea más rápido que la re-termalización del sistema, por lo que se recurre a un láser el cual se utiliza para aumentar la interacción. Se estudia de forma específica el caso donde la frecuencia del tambor mecánico es una función periódica del tiempo. Se toma el caso de un tambor mecánico cuya frecuencia oscila de forma pequeña y de manera periódica en torno a una frecuencia promedio. Se busca utilizar un formalismo que tome en cuenta la dependencia temporal del sistema al hacer la aproximación de Markov a fin de obtener una ecuación maestra mejorada. Se obtiene una predicción para el número promedio de excitaciones vibracionales a fin de tener una medida final para las temperaturas del estado estacionario.

\section{Sistemas Optomecánicos}

Se consideran sistemas donde puede existir un acoplamiento entre objetos mecánicos macroscópicos y sistemas ópticos. Un ejemplo muy sencillo es una cavidad óptica de Fabry-Perot, la cual consiste en dos espejos planos y paralelos alineados uno frente a otro. Este es un sistema lineal que se comporta como un resonador. Los espejos, hablando de manera estricta, deben ser planos. Sin embargo, el nombre se utiliza también para cavidades con espejos curvos\cite{PaschottaFP}. Las frecuencias resonantes de la cavidad pueden ajustarse mediante su longitud. Si uno de los dos espejos no se encuentra fijo, si no más bien se encuentra acoplado a un resorte, por ejemplo, esto causaría que, si la presión de radiación debida a la luz dentro de la cavidad fuese suficiente para mover el espejo, entonces la longitud de la cavidad, y por ende sus propiedades ópticas, se convierten en funciones del tiempo. Este es un sistema optomecánico. Otro posible ejemplo es una cavidad con una membrana dieléctrica reflectiva actuando como espejo \cite{JockelMR}. Este tema se discutirá con mayor detalle más adelante.

\section{Motivación Experimental}

A fin de estudiar el comportamiento de objetos macroscópicos en un régimen cuántico, estos deben estar muy cerca de su estado base cuántico \cite{ChanQM}. Esto requiere de técnicas de enfriamiento altamente efectivas. Como se detallará en secciones más adelante, los sistemas optomecánicos proveen una solución análoga al enfriamiento láser de sistemas atómicos \cite{NooshiLC}, en el sentido de que se utilizan para preparar estados puros. En este trabajo se busca una mejor teoría para describir este tipo de sistemas.Esta teoría se aplica para estudiar el caso específico de un resonador donde la frecuencia del mismo depende del tiempo, a fin de encontrar si esto podría llevar a un mejor enfriamiento.
 

\section{Esquema de Temas}

El trabajo se divide en cinco secciones distintas, seguidas de conclusiones y un apéndice. El esquema temático de cada sección es:

\begin{enumerate}
\item \textbf{Sistemas Cuánticos Abiertos y Ecuaciones Maestras}: En este capítulo se discute la teoría básica de las ecuaciones maestras como modelo para los sistemas cuánticos abiertos. Se realiza la deducción de una ecuación maestra mediante proyectores a sub-espacios específicos, para el caso de una interacción entre un sistema descrito con operadores de oscilador armónico y un baño térmico. 

\item \textbf{Ecuaciones Maestras: Átomo de dos Niveles, Oscilador Armónico y la Base de Decaimiento}: Se deduce las ecuaciones maestras correspondientes a los dos sistemas. Se estudia la solución para el átomo de dos niveles como primer ejemplo, y luego para el oscilador armónico. El segundo caso es de evidente importancia para el enfoque de este trabajo. Para resolver el caso del oscilador armónico se emplea el formalismo de la base de decaimiento \cite{EnglertDB}, la cual se estudia en detalle, y la que consiste, a grosso modo, en en una suma de distintas potencias de operadores de escalera de oscilador armónico en ordenamiento normal

\item \textbf{Solución de la Ecuación Maestra Mediante Estados de Floquet}: Se expone una breve introducción a la teoría de Floquet para ecuaciones diferenciales con coeficientes dependientes del tiempo. Se estudia como se traduce este formalismo al oscilador armónico dependiente del tiempo en mecánica cuántica. Se llega a una ecuación maestra para el sistema de interés del trabajo en este formalismo.

\item \textbf{Enfriamiento Optomecánico}: Se da una breve introducción a algunos tipos de sistemas optomecánicos utilizados actualmente. Se deduce el Hamiltoniano correspondiente al enfriamiento optomecánico con un tambor mecánico con frecuencia dependiente del tiempo.

\item \textbf{Ecuación Maestra de Enfriamiento Laser con Frecuencia Dependiente del Tiempo} se deduce la ecuación maestra correspondiente al Hamiltoniano del capítulo anterior, agregando un baño térmico y una interacción entre ambos sistemas. Esta ecuación se transforma al marco desplazado mediante una transformación unitaria. Los cálculos de esta transformación se hacen en detalle, notando especialmente como cambiar al tratar con operadores que dependen explícitamente del tiempo. Se deriva la ecuación maestra correspondiente al sistema para el caso de oscilaciones pequeñas y periódicas en torno a una frecuencia promedio. Estos resultados son originales.

\item \textbf{Teoría del Enfriamiento Láser para sistemas Optomecánicos}: Se identifican tres escalas de tiempo para los componentes del sistema y se realiza una aproximación adiabática para proyectar las ecuaciones a la escala de tiempo de la parte del sistema correspondiente a la interacción.Estas ecuaciones se resuelven para el caso de la interacción optomecánica utilizada en el Hamitloniano deducido en el capitulo de Enfriamiento Optomecánico. Estos resultados son originales.

\item \textbf{Conclusiones}

\end{enumerate}

\chapter{Sistemas Cuánticos Abiertos y Ecuaciones Maestras}

\section{Introducción}

Modelar la pérdida de energía en un sistema clásico como un oscilador armónico es un proceso directo \cite{CarmichaelQO} que consiste en modificar el Hamiltoniano usual

\begin{equation}
H = \frac{p^2}{2m} + \frac{1}{2} m\omega^2 q^2,
\end{equation} agregando un termino dependiente de la velocidad $-\gamma p$,a la ecuación de movimiento para $q$ lo cual lleva a la ecuación

\begin{equation}
\ddot{q} + \gamma \dot{q} + \omega^2 q = 0.
\end{equation}Donde el término correspondiente a $\gamma$ es el causante de la disipación. Sin embargo, este enfoque no es igualmente exitoso al tratar con sistemas cuánticos. Al repetir el procedimiento anterior se llega a las ecuaciones lineales

\begin{equation}
\dot{q} = \frac{p}{m} \qquad \dot{p} = -\gamma p -m\omega ^2 q,
\end{equation}Al ser estas lineales, deben de seguir siendo validas al cambiar $p$ y $q$ por operadores, sin embargo esto resulta problemático al considerar la evolución temporal de la relación canónica de conmutación, tomando ahora a $p$ y $q$ como operadores

\begin{align*}
 \dfrac{d}{dt}[q,p] =& \dot{q}p + q\dot{p} - \dot{p}q p-\dot{q}\\  
 =& -\gamma[q,p],
\end{align*}por lo que

\begin{equation}
[q(t),p(t)] = e^{-\gamma t}[q(0),p(0)] = i\hbar e^{-\gamma t}.
\end{equation} El enfoque que en el caso clásico lleva a la solución correcta, en el caso cuántico ocasiona a que la relación de conmutación canónica decaiga con el tiempo. Claramente, esto no es permisible y se requiere de otro enfoque.

\section{Enfoque de Sistema y Baño}

Este enfoque asume que el sistema no pierde energía, en realidad las perdidas, o ganancias, de energía se deben al acoplamiento a un sistema mucho más grande, al cual se le conoce como baño. Normalmente las propiedades del baño no son de interés, más allá de propiedades básicas como su temperatura. Se asume que el acoplamiento causa intercambios de energía entre el sistema de estudio y el baño,pero que este sistema no es lo suficientemente grande como para alterar las propiedades del baño. Esto se representa con un Hamiltoniano de la forma:

\begin{equation} \label{BathSysH}
H = H_{Sis} + H_{B} + H_{Int}.
\end{equation} Esto es, un Hamiltoniano total formado por la suma del Hamiltoniano del sistema original , un Hamiltoniano para el baño y un Hamiltoniano de acoplamiento entre los otros dos.  Dado que solo es de interés conocer las propiedades del sistema, esto sugiere el uso de matrices densidad, ya que en este formalismo es posible trazar sobre las variables no deseadas, es decir las variables del baño, para obtener únicamente la información de interés.

\subsection{Matriz Densidad}

Si un sistema se encuentra en un estado $\Ket{\Psi}$, su matriz densidad\cite{BlumDM}, usualmente denotada como $\rho$, está dada por

\begin{equation}
\rho = \Ket{\Psi}\Bra{\Psi}.
\end{equation} La matriz densidad, a veces también llamada operador densidad, contiene toda la información del sistema. Se dan a continuación algunas propiedades de posible interés. Si se desea conocer el valor de expectación de un operador, este es

\begin{equation}
<A> = Tr[A\rho],
\end{equation} su evolución temporal se comporta de acuerdo con la ecuación de Liouville

\begin{equation}
\dfrac{\partial \rho}{\partial t} = \frac{1}{i\hbar}[H,\rho].
\end{equation} Finalmente, y de especial utilidad, existe una forma crear un objeto matemático a partir de una matriz densidad correspondiente a un cierto espacio que contiene únicamnte información de un subespacio particular del espacio original. Si se conoce la matriz densidad de un sistema compuesto formado por dos sistemas $A$ y $B$ tal que el sistema compuesto se encuentra en un espacio formado por el producto directo de los dos subsistemas

\begin{equation}
H_{AB} = H_A \otimes H_B,
\end{equation} es posible encontrar la matriz densidad correspondiente a uno de los dos subespacios al tomar la traza sobre los grados de libertad del otro espacio

\begin{equation}
\rho_A = Tr_B [\rho_{AB}].
\end{equation}

Esto muestra la utilidad de este formalismo para trabajar con Hamiltonianos del tipo \eqref{BathSysH}, ya que si se encuentra una solución general basta con tomar la traza sobre los estados del baño para encontrar una solución para el sistema con pérdidas de energía. A continuación se verá como obtener una ecuación diferencial para el sistema reducido usando el formalismo de matriz densidad.

\section{Deducción de Ecuación Maestra Mediante Operadores de Proyección}

Se busca obtener la ecuación diferencial correspondiente a la evolución temporal a un sistema sujeto a interacción con un baño térmico

\begin{equation}
\hat{H} = \hat{H}_{Sis}+\hat{H}_{Int}+\hat{H}_B,
\end{equation} lo cual implica una ecuación de evolución temporal para $\rho$ de la forma

\begin{equation}\label{LMeq}
\dot{\rho}_{total} = (L_{Sis}+L_{Int}+L_B)\rho,
\end{equation}donde

\begin{equation}\label{defL}
L\rho(t) = \frac{i}{\hbar}[\hat{H},\rho(t_0)].
\end{equation} Se busca una ecuación para el operador densidad únicamente del sistema, por lo que se busca una ecuación para el operador total después de tomar la traza sobre las variables del baño

\begin{equation}
\rho(t) \equiv Tr_B\{\rho_{total}(t)\}.
\end{equation} Se define un operador de proyección \cite{ZollerQN} que opere sobre $ \rho $

\begin{equation}
P\rho_{total} = Tr_B\{\rho_{total}(t)\}\otimes \rho_B,
\end{equation} y un operador

\begin{equation}
Q = \mathbf{1}-P.
\end{equation} El operador $P$ proyecta a la matriz densidad $\rho$ a un espacio donde esta es separable, es el producto directo de $\rho$ al perder toda la información sobre los grados de libertad del baño y un estado del baño que se puede elegir de manera conveniente. De las definiciones anteriores es trivial ver que

\begin{equation}
1 = P + Q,
\end{equation} por lo que

\begin{align}
\rho_{total} =& P\rho_{total} + Q\rho_{total} ,\\
\equiv & v(t)+u(t).
\end{align} Se emplean las siguientes propiedades de los operadores $P$ y $Q$

\begin{enumerate}
\item $ PL_{Sis} = L_{Sis}P$\qquad Ya que $ P $ y $ L_{Sis} $ operan en espacios distintos.

\item $ PL_{B} = L_{B}P = 0 $\qquad Por conservación de la probabilidad.

\item $PL_{Int}P=0$ \qquad Se asume que la interacción no tiene términos diagonales en las variables del baño

\item $P^2 = P \quad Q^2 = Q$ \qquad $P$ y $Q$ son proyectores
\end{enumerate}

Por conveniencia se trabaja con la transformada de Laplace \cite{ArfkenMM} de las proyecciones $v(t)$ y $u(t)$

\begin{equation}
\tilde{v}(t) = \int_0^\infty e^{-st}v(t)dt \qquad \tilde{u}(t) = \int_0^\infty e^{-st}u(t)dt.
\end{equation} Dado que las dos proyecciones se encuentran en espacios distintos, se puede sustituir $\rho$ por cada una de ellas en \eqref{LMeq} y resolver por separado para cada proyección, ya que ambas deben cumplir la ecuación de forma independiente. En el caso de $v(t)$ se tiene, antes de utilizar la transformada de Laplace, que al aplicar el proyector $P$ a \eqref{LMeq}

\begin{align*}
P\dot{\rho}_{total}(t)_t =& P(L_{Sis}+L_{Int}+L_B)\rho, \\
 \dot{v}(t) =& (PL_{Sis}+PL_{Int}+PL_B)\rho, \\
 =&(L_{Sis}P+PL_{Int}+\underbrace{PL_B}_0)\rho, \\
 =& L_{Sis}P\rho_t + PL_{Int}\rho_t, \\
 =& L_{Sis}v + \underbrace{PL_{Int}(P}_0+Q)\rho_t, \\
 =& L_{Sis}v + PL_{Int}Q\rho_t, \\
 =& L_{Sis}v + PL_{Int}u.
\end{align*} De manera análoga se obtiene una ecuación para $u$ al proyectar con $Q$

\begin{align*}
Q\dot{p}(t)_t =& Q(L_{Sis}+L_{Int}+L_B)\rho, \\
 \dot{u}(t) =& (QL_{Sis}+QL_{Int}+QL_B)\rho, \\
 =&(L_{Sis}Q+QL_{Int}+L_BQ)\rho, \\
 =& L_{sis}Q\rho_t + L_{B}Q\rho_t + QL_{int}\rho_t, \\
 =& L_{sis}u + L_{B}u + QL_{int}(Q+P)\rho_t,  \\
 =& L_{sis}u + L_{B}u + QL_{int}v+QL_{int}u\rho_t, \\
 =& (L_{Sis}+L_B+QL_{Int})u +QL_{Int}v. 
\end{align*} Al aplicar la transformada de Laplace a la ecuación para $v$ se obtiene

\begin{equation}\label{eqv}
s\tilde{v}(s)-\tilde{v}(0) = L_{sis}\tilde{v}(s) + PL_{int}\tilde{u}(s),
\end{equation} y en el caso de $u$

\begin{equation}\label{equ}
s\tilde{u}(s)-\tilde{u}(0) = (L_{Sis}+L_B+QL_{Int})\tilde{u}(s) + QL_{Int}\tilde{v}(s).
\end{equation} Se resuelve \eqref{equ} en términos de $v$

\begin{equation}
\tilde{u}(s)=\frac{QL_{Int}\tilde{v}(s)+\tilde{u}(0)}{s-\underbrace{(L_{Sis}+L_B+QL_{Int})}_\alpha},
\end{equation} y se sustituye el resultado en \eqref{eqv}

\begin{equation}
s\tilde{v}(s)-(\tilde{v}(0)+PL_{int}\alpha\tilde{u}(0))=(L_{sis}+PL_{int}\alpha QL_{int})\tilde{v}(s).
\end{equation} En este punto se toma la aproximación de acoplamiento débil, la cual consiste en asumir que la interacción es muy pequeña y puede tratarse de forma perturbativa

\begin{equation}
L_{int} \rightarrow \epsilon L_{int}.
\end{equation} Los términos asociados a $\tilde{u}_0$ son únicamente correcciones a la condición inicial, ya que la transformada inversa resulta

\begin{equation}
e^{(L_{Sis}+L_B+QL_{Int})t}u_0,
\end{equation} por lo que se desprecian. Tomando esto en cuenta, se obtiene

\begin{equation}
s\tilde{v}(s)-\tilde{v}(0) = (L_{sis}+\epsilon^2\frac{PL_{int}QL_{int}}{s-L_{sis}-L_{B}})s\tilde{v}(s).
\end{equation} Y al tomar la transformada inversa de Laplace

\begin{equation}\label{eqProyectores}
\dot{v}(t)=L_{sis}v(t) + \epsilon^2 PL_{int} \int_{0}^{\infty}d\tau e^{(L_{sis}+L_{B})\tau}QL_{int}v(t-\tau).
\end{equation} Se busca obtener una ecuación en términos de los operadores del Hamiltoniano correspondiente únicamente al sistema y no al baño \cite{ZollerQN}, entoncesse procede con el segundo término de la ecuación \eqref{eqProyectores}

\begin{align*}
QL_{int}v=&(1-P)L_{int}v,\\
=& (1-P)L_{int}P\rho, \\
=& L_{int}P\rho, \\
=& L_{int}v,
\end{align*}

Con esto el segundo término de la ecuación \eqref{eqProyectores}, y tomando en cuenta la definición de los términos $L$ dada en \eqref{defL} y el lema de Baker-Campbell-Housedorff \cite{SakuraiQM}

\begin{align*}
=& \epsilon^2 PL_{int} \int_{0}^{\infty}d\tau e^{(L_{sis}+L_{B})\tau}L_{int}v(t-\tau),\\
=& \epsilon^2 PL_{int} \int_{0}^{\infty}d\tau e^{\frac{-i}{\hbar}H_B\tau}e^{\frac{-i}{\hbar}H_s\tau}L_{int}v(t-\tau)e^{\frac{i}{\hbar}H_B\tau}e^{\frac{i}{\hbar}H_s\tau},\\
=&\epsilon^2 PL_{int} \int_{0}^{\infty}d\tau L_{int}v(t-\tau), \\
=&\epsilon^2 PL_{int} \int_{0}^{\infty}d\tau[H_{int}^*,P\rho(t-\tau)], \\
=&\epsilon^2 P[H_{int},\int_{0}^{\infty}d\tau[H_{int}^*,tr_B(\rho(t-\tau))\otimes\rho_B]],
\end{align*}

En este punto se realiza la aproximación de Markov \cite{ZollerQN}, al hacer la sustitución

\begin{equation}
t-\tau \rightarrow t,
\end{equation} y se traza toda la ecuación sobre las variables del baño. Con esto se llega a la ecuación:

\begin{equation}
=\epsilon^2 \int_{0}^{\infty}d\tau Tr_B[[H_{int}(t),[H_{int}(t-\tau)\rho(t)\otimes\rho_B]]].
\end{equation} Esta ecuación, al insertar distintas interacciones genera las ecuaciones maestras deseadas. Si se utiliza una interacción del tipo $\sum_m X^+_m\Gamma_m + X^-_m\Gamma^\dagger_m$, donde los operadores $X_m^\pm$ son operadores con reglas de conmutación de oscilador armónico, correspondiente a una frecuencia $\omega_m$ se genera la ecuación \cite{ZollerQN}

\begin{align}
\dot{\rho} =& i\sum_m [\delta_m X^+_m X^-_m + e_mX^-_mX^+_m,\rho],\\
&+ \nonumber \frac{1}{2}\sum_m K_m D[X^-_m] + G_m\frac{1}{2}\sum_m K_m D[X^+_m],
\end{align} donde

\begin{align}
\int_0 ^\infty e^{i\omega_m \tau}Tr_B[\Gamma(\tau)_m\Gamma(0)^\dagger_m] \equiv \frac{1}{2}K_m+i\delta_m,\\
\int_0 ^\infty e^{i\omega_m \tau}Tr_B[\Gamma(0)_m\Gamma(\tau)^\dagger_m]\equiv \frac{1}{2}K_m-i\delta_m,\\
\int_0 ^\infty e^{i\omega_m \tau}Tr_B[\Gamma(\tau)^\dagger_m\Gamma(0)_m]\equiv\frac{1}{2}G_m+ie_m,\\
\int_0 ^\infty e^{i\omega_m \tau}Tr_B[\Gamma(0)^\dagger_m\Gamma(\tau)_m]\equiv\frac{1}{2}G_m-ie_m.
\end{align}

A continuación se estudiaran posibles soluciones a este tipo de ecuaciones.


\chapter{Ecuaciones Maestras: Átomo de dos Niveles, Oscilador Armónico y la Base de Decaimiento}

Se busca poder resolver ecuaciones como las encontradas en el capítulo anterior, las cuales normalmente se escriben de la forma

\begin{equation}\label{MasterEq}
\dot{\rho} = \frac{1}{i\hbar}[H,\rho] + L\rho,
\end{equation} donde $\rho$ es la matriz densidad del sistema, $H$ es el Hamiltoniano que modela el sistema sin tomar en cuenta pérdidas de energía y $L$ es el superoperador de Lindblad, el cual codifíca el efecto de la parte abierta del sistema, es decir la parte del sistema que puede interaccionar con el medio ambiente y causar intercambios de energía. En la mecánica cuántica correspondiente a un sistema cerrado, el análisis se limita al operador Hamiltoniano el cual es un operador auto adjunto. En ese caso, es suficiente resolver la ecuación de Schröedinger estacionaria

\begin{equation}
H\Psi = E\Psi,
\end{equation} cuya solución formal es

\begin{equation}
\Psi(t) = e^{\frac{-iHt}{\hbar}}\Psi(0).
\end{equation} La función $\Psi(0)$ puede expresarse como una  combinación lineal de las eigenfunciones de $H$. Al hacer esto, la ecuación anterior toma la forma

\begin{equation}
\Psi(t) = \sum_j c_j e^{\frac{-iE_jt}{\hbar}}\phi_j(0),
\end{equation}los coeficientes $c_j$ son las proyecciones de la función $\Psi(0)$ sobre las funciones propias de $H$, las cuales se denominan $\phi_j(0)$,

\begin{equation}
c_j(t)= (\phi_j(t),\Psi(t)).
\end{equation}Donde $(,)$ denota el producto del espacio. Se busca una expansión equivalente para la ecuación \eqref{MasterEq}. Es posible transformar dicha ecuación al cuadro de interacción, donde solo importa la parte del sistema descrita por términos de Lindblad. En este cuadro, se busca una solución del tipo

\begin{equation}
\rho(t) = \sum_{\lambda} c_{\lambda} e^{\lambda t}\rho_{\lambda} ,
\end{equation}en analogía al caso donde el operador en cuestión es hermítico. Los coeficientes ahora son respecto a las funciones propias de $L$. Ya que $L$ actúa no sobre vectores en el espacio de las funciones $\Psi(t)$, si no más bien sobre matrices densidad como son las $\rho(t)$ definidas anteriormente, las cuales son operadores y existen en otro espacio vectorial, se le conoce como superoperador. Es importante notar que $L$ no es necesariamente autoadjunto, por lo que no se puede garantizar que sus funciones propias formen una base completa del espacio ni que sus valores propios sean reales. En el caso de este tipo de operadores existen dos tipos de vectores propios\cite{BarnettSD}, izquierdos y derechos

\begin{align*}
L\rho =& \lambda \rho, \\
\check{\rho} L =& \lambda \check{\rho}.
\end{align*} Los estados que se obtienen al aplicar el superoperador por la derecha se conocen también como estados duales, se denotan por $\check{\rho}$. También es importante notar que los vectores que se obtienen al aplicar por la izquierda y por la derecha no son simplemente complejos conjugados uno del otro, se conocen como duales y son ortogonales los unos a otros bajo un producto definido mediante la traza\cite{EnglertDB}, sin embargo sus valores propios son iguales. Los vecotres obtenidos no forman necesariamente una base del espacio de operadores. 

\begin{equation}
Tr[\check{\rho}_\lambda \rho_{\lambda'}] = \delta_{\lambda'\lambda}.
\end{equation} Esto se puede emplear para obtener las constantes de la expansión

\begin{equation}
\rho(0) = \sum_{\lambda} \check{c}_{\lambda} \rho_{\lambda},
\end{equation} donde

\begin{equation}
\check{c}_\lambda = Tr[\check{\rho}_\lambda \rho(0)].
\end{equation} Es importante notar que para realizar una expansión es estados de este tipo es necesario poder trazar sobre los estados duales. Por esto, estos estados se estudiarán de manera más cuidadosa en una sección posterior. Primero se tratan algunos ejemplos.

\section{Átomo de dos Niveles}

Antes de intentar resolver sistemas más complejos, es ilustrativo tratar el caso relativamente sencillo del átomo de dos niveles. A este sistema le corresponde el operador de Linblad \cite{EnglertDB}

\begin{align}\label{TwoLevel}
 L_{\sigma}\rho = 
 &-\frac{B}{2}(1-s)[\sigma_+\sigma_-\rho + \rho\sigma_+\sigma_--2\sigma_-\rho\sigma_+ ]\nonumber\\
 &-\frac{B}{2}s[\sigma_-\sigma_+\rho + \rho\sigma_-\sigma_+-2\sigma_+\rho\sigma_- ]\nonumber\\
 &-\frac{2C-B}{4}[\rho - \sigma_z\rho\sigma_z].
\end{align} Es conveniente utilizar notación de Dirac en este tratamiento. Todos los operadores involucrados pueden verse como matrices de dos por dos. $\sigma_z$ es la matriz de Pauli correspondiente al eje z y

\begin{equation}
\sigma_{\pm} = \sigma_x \pm \sigma_y = (\sigma_{\mp}) ^\dagger.
\end{equation} Un método \cite{BarnettSD} para encontrar los vectores y valores propios de $L$ consiste en expandir $\rho$ en una base adecuada, aplicar $L$ a la propuesta y resolver las ecuaciones resultantes. En este caso se propone una combinación lineal general

\begin{equation}
\rho_\sigma = \alpha\Ket{+}\Bra{+}+\beta\Ket{+}\Bra{-}+\gamma\Ket{-}\Bra{+}+\delta\Ket{-}\Bra{-}.
\end{equation} Ya que se conoce la forma en la cual las matrices $\sigma$ operan sobre estos elementos

\begin{align*}
&&\sigma_{+}\Bra{-} =& \Bra{+} \quad \sigma_{+}\Bra{+} = 0, \\
&&\sigma_{-}\Bra{+} =& \Bra{-} \quad \sigma_{-}\Bra{-} = 0, \\
&&\sigma_{z}\Bra{+} =& \Bra{+} \quad \sigma_{z}\Bra{-} = -\Bra{-}.
\end{align*} Al sustituir $\rho_{\sigma}$ en \eqref{TwoLevel} se obtiene, para el primer término de $L$

\begin{align}
&-\frac{B}{2}(1-s)[2\alpha\Ket{+}\Bra{+}+\beta\Ket{+}\Bra{-}+\gamma\Ket{-}\Bra{+}-2\alpha\Ket{-}\Bra{-}]\\
&\nonumber -\frac{B}{2}(s)[2\delta\Ket{+}\Bra{+}+\beta\Ket{+}\Bra{-}+\gamma\Ket{-}\Bra{+}-2\delta\Ket{-}\Bra{-}]\\
&\nonumber -\frac{2C-B}{4}[2\beta\Ket{+}\Bra{-}+2\gamma\Ket{-}\Bra{+}].
\end{align}

Esto se sustituye en la ecuación de valores propios correspondiente a $L$

\begin{equation}
L\rho = \lambda\rho,
\end{equation} dado que cada uno de los cuatro componentes de $\rho$ son linealmente independientes, esto genera cuatro ecuaciones distintas

\begin{align}
B(s(\alpha+\delta)-\alpha)\Ket{+}\Bra{+} =& \lambda\alpha\Ket{+}\Bra{+},\\
B(-s(\alpha+\delta)+\alpha)\Ket{-}\Bra{-} =& \lambda\delta\Ket{-}\Bra{-},\\
-\beta C \Ket{+}\Bra{-} =& \lambda \beta \Ket{+}\Bra{-}, \\
-\gamma C \Ket{-}\Bra{+} =& \lambda \gamma \Ket{-}\Bra{+},
\end{align}de las cuales se obtienen los cuatro vectores propios

\begin{align}
L_\sigma \sigma_0 =& 0, \\
L_\sigma \sigma_z =& -B \sigma_z, \\
L_\sigma \sigma_\pm =& -C \sigma_\pm,
\end{align} donde:

\begin{equation}
\sigma_0 = \frac{1}{2}[I + (2s-1)\sigma_z].
\end{equation} Las constantes en \eqref{MasterEq} son todas positivas por lo que todos los valores propios son no positivos. Esto es importante, ya que al hacer tender el tiempo a infinito solo el término que corresponde al valor propio cero sobrevive, los demás términos decaen exponencialmente con el tiempo. Esto se debe a que la evolución temporal del sistema, expresada en la expansión en la base propia de $L$ se ve como

\begin{equation}
\sum_\lambda e^{\lambda t} \sigma_\lambda,
\end{equation} no hay términos oscilatorios por estar en el cuadro de interacción. Todos los valores propios son negativos o cero. Esto lleva a identificar al término con valor propio cero como el estado estacionario del sistema. Los estados duales se obtienen de la misma forma y resultan ser\cite{EnglertDB}

\begin{align}
\check{\sigma_0} L_\sigma  =& 0, \\
\check{\sigma_z} L_\sigma =& -B \check{\sigma_z}, \\
\check{\sigma_{\pm}} L_\sigma =& -C \check{\sigma_{\pm}}.
\end{align} Donde:

\begin{align*}
\check{\sigma_0} =& 1,\\
\check{\sigma_z} =& \frac{1}{2}[\sigma_z-(2s-1)],\\
\check{\sigma_{\pm}} =& \frac{1}{4}\sigma_{\mp}.
\end{align*}

La ortogonalidad de los estados no duales con los estados duales puede comprobarse de forma directa, por ejemplo utilizando $\check{\sigma_0}$

\begin{align*}
Tr[\sigma_0\check{\sigma_0}] =& Tr[\frac{1}{2}[I + (2s-1)\sigma_z]],\\
=& \frac{1}{2} Tr[I] + \frac{2s-1}{2}Tr[\sigma_z],\\
=& 1.
\end{align*} Donde se utiliza el hecho de que las matrices de Pauli tienen traza cero. Con esta misma propiedad se puede verificar que los productos restantes con $\check{\sigma_0}$ se anulan. A continuación se trata un sistema más complejo, donde se modela una cavidad óptica con perdidas.

\section{Oscilador Armónico: Primer Ansatz}

Se busca modelar interacciones entre un tambor mecánico y un campo electromagnético el cual actúa como reservorio térmico, o baño. Esta interacción, con estos sistemas en particular corresponde a un superoperador de Lindblad de la forma \cite{EnglertDB}

\begin{align}\label{EMField}
L_a \rho =& - \frac{A}{2}(\nu + 1)[a^\dagger a\rho + \rho a^\dagger a -2a\rho a^\dagger] \nonumber \\
 &- \frac{A}{2}(\nu)[ aa^\dagger\rho + \rho  aa^\dagger -2a^\dagger\rho a].
\end{align} Esto modela un campo electromagnético en una cavidad con acoplamiento a un reservorio térmico con un número $\nu$ promedio de fotones térmicos. $A,\nu \geq 0$. Los operadores $a,a^{\dagger}$ son operadores los cuales siguen la misma álgebra que los operadores de escalera para el oscilador armónico en mecánica cuántica. La elección de un ansatz para este sistema es mucho más compleja, ya que ahora la base de estados sobre la cual actúan los operadores que forman $L$ es infinita.  En base a lo visto en la sección anterior, se propone como ansatz una base que acople estados de distintas energía

\begin{equation}
 \rho^l_n = \sum_{n=0}^\infty Q_n^l\Ket{n}\Bra{n+l},
\end{equation} El procedimiento de nuevo es sustituir el ansatz en \eqref{EMField} para obtener un sistema de ecuaciones para los valores y vectores propios. Se trabaja primero con el primer término de \eqref{EMField}, donde se ignora el coeficiente que multiplica los términos dentro de $[...]$ por brevedad

\begin{align*}
a^\dagger& a\rho^l_n + \rho^l_n a^\dagger a -2a\rho^l_n a^\dagger\\
=&[a^\dagger a\sum_{n=0}^\infty Q_n^l\Ket{n}\Bra{n+l} + \sum_{n=0}^\infty Q_n^l\Ket{n}\Bra{n+l} a^\dagger a -2a\sum_{n=0}^\infty Q_n^l\Ket{n}\Bra{n+l} a^\dagger],\\
=&[\sum_{n=0}^\infty Q_n^la^\dagger a\Ket{n}\Bra{n+l} + \sum_{n=0}^\infty Q_n^l\Ket{n}\Bra{n+l} a^\dagger a -2\sum_{n=0}^\infty Q_n^la\Ket{n}\Bra{n+l} a^\dagger],\\
=&[\sum_{n=0}^\infty Q_n^ln\Ket{n}\Bra{n+l} + \sum_{n=0}^\infty Q_n^l\Ket{n}\Bra{n+l} (n+l) -2\sum_{n=0}^\infty Q_n^l\sqrt{n}\Ket{n-1}\Bra{n+l-1}\sqrt{n+l}],\\
=&[\sum_{n=0}^\infty Q_n^l(2n+l)\Ket{n}\Bra{n+l} -2\sum_{n=0}^\infty\sqrt{n(n+l)} Q_n^l\Ket{n-1}\Bra{n+l-1}].\\
\end{align*} El segundo término actúa de una forma completamente análoga y se obtiene

\begin{equation}
aa^\dagger\rho^l_n + \rho^l_n  aa^\dagger -2a^\dagger\rho^l_n a=[\sum_{n=0}^\infty Q_n^l(2n+l+2)\Ket{n}\Bra{n+l} -2\sum_{n=0}^\infty\sqrt{(n+1)(n+l+1)} Q_n^l\Ket{n+1}\Bra{n+l+1}].
\end{equation} Como $\rho^l_n$  es un vector propio los términos anteriores deben de poder igualarse a un valor propio multiplicando a $\rho^l_n$

\begin{align}
\lambda_n^l \rho_n^l=& \sum_{n=0}^\infty Q_n^l \{ -\frac{A}{2}(\nu+1)[(2n+l)\Ket{n}\Bra{n+l} -2\sqrt{n(n+l)} Q_n^l\Ket{n-1}\Bra{n+l-1}] \nonumber, \\
-&\frac{A}{2}(\nu)[(2n+l+2)\Ket{n}\Bra{n+l}-2\sqrt{(n+1)(n+l+1)} Q_n^l\Ket{n+1}\Bra{n+l+1}]\}.
\end{align} Si en el segundo término se recorre el índice de suma hacia arriba por uno y en el cuarto hacía abajo por uno, se puede utilizar independencia lineal respecto al elemento $\Ket{n}\Bra{n+l}$ para obtener una ecuación para cada elemento de la suma. Se obtiene que los coeficientes deben cumplir con

\begin{align}\label{RecurrenciaStengholm}
A&(\nu+1) Q_{n+1}^l \sqrt{(n+1)(n+l+1) }+ A(\nu) Q_{n-1}^l\sqrt{n(n+l)}\\ \nonumber
=&[\lambda_n^l+A\nu(2n+l+2)+\frac{A}{2}(2n+l)]Q_{n}^l.
\end{align}  En condiciones generales, esta relación de recurrencia es difícil de resolver de forma analítica. Sin embargo, en el caso de temperatura cero, se tiene que $\nu = 0$ y \eqref{RecurrenciaStengholm} se simplifica considerablemente

\begin{equation}
AQ_{n+1}^l\sqrt{(n+1)(n+l+1)}=[\lambda_n^l + \frac{A}{2}(2n+l)]Q_n^l,
\end{equation} Lo cual permite resolver para una de las dos $Q$

\begin{equation}
Q_{n+1}^l = \frac{\lambda_n^l + \frac{A}{2}(2n+l)}{A\sqrt{(n+1)(n+l+1)}}Q_n^l.
\end{equation}Esta expresión tiene un número infinito de soluciones. En \cite{BarnettSD} se corta la suma con el valor de $\lambda$ y se llega a una solución para A, que ahora también depende de un nuevo entero $m$

\begin{equation}
A_m^l = \sum_{n=0}^m (-1)^m \frac{m!}{(m-n)!}\sqrt{\frac{l!}{n!(n+l)!}}\Ket{n}\Bra{n+l},
\end{equation} a la cual le corresponde a un valor para $\lambda$ de

\begin{equation}
\lambda_l^m = -(2m+l).
\end{equation}Este número particular se elige a fin de que los estados sean expresados como una suma finita. Los estados duales del sistema se calculan a base de la relación de ortogonalidad con los estados ya encontrados, estos resultan ser \cite{BarnettSD}

\begin{equation}
B_m^l = \sum_{n=m}^\infty \sqrt{\frac{n!(n+l)!}{l!}}\frac{1}{(n-m)!}\Ket{n}\Bra{n+l}
\end{equation}

Sin embargo esto solo es valido en el caso de temperatura cero. Para poder tratar el caso general se requiere de otro enfoque.

\section{Oscilador Armónico: Base de Decaimiento}

Se sigue el ansatz propuesto en \cite{EnglertDB}, el cual consiste en una función arbitraria de operadores de creación y aniquilación en ordenamiento normal, y posibles operadores adicionales de aniquilación:

\begin{equation}\label{Englert1993}
\rho_\lambda (a,a^\dagger) = :f(aa^\dagger):a^l.
\end{equation} Los $::$ denotan ordenamiento normal, lo cual puede requerir el desarrollo en serie de la función $f$. Se puede expresar $a^l$ en la base de número como

\begin{equation}
\sum_{n=0}^\infty C_n^l\Ket{n}\Bra{n+l},
\end{equation} donde puede verse la relación de este ansatz con el de la sección anterior. A partir de este se llega a la solución para $\rho_n^l$ \cite{EnglertDB}

\begin{align}\label{DefDB}
&a^{\dagger l}\frac{(-1)^n}{(\nu+1)^{l+1}}:L_n^l[\frac{a^\dagger a}{\nu+1}]e^{-[\frac{a^\dagger a}{\nu+1}]}:\quad l \geq 0, \\
&\frac{(-1)^n}{(\nu+1)^{|l|+1}}:L_n^{|l|}[\frac{a^\dagger a}{\nu+1}]e^{-[\frac{a^\dagger a}{\nu+1}]}:a^{|l|}\quad l \leq 0,
\end{align} con valores propios

\begin{equation}
\lambda_n^l = -A[n + \frac{|l|}{2}],
\end{equation} los cuales cumplen con las condiciones

\begin{equation}
n=0,1,2...,\qquad l = 0,\pm 1, \pm 2,... 
\end{equation}Es importante notar que esto se obtiene en el cuadro de interacción, por lo que los valores finales deben incluir los valores propios de la parte correspondiente al Hamiltoniano sin intercambios de energía.

A continuación se muestra de forma explicita como calcular estos elementos. 

\subsection{Estado Base}

Este es el estado con valor propio cero, lo cual corresponde a $n=0, \qquad l=0$. En este caso, la solución tiene la forma explícita

\begin{equation}
\frac{1}{(\nu+1)}:L_0^0[\frac{a^\dagger a}{\nu+1}]e^{-[\frac{a^\dagger a}{\nu+1}]}:,
\end{equation} ya que $L_0^0 = 1$ \cite{ArfkenMM} si se desarrolla la exponencial en una serie infinita se tiene, aplicando el ordenamiento normal

\begin{equation}
\sum_{n=0}^\infty \frac{-(a^\dagger)^n (a)^n}{n!(\nu+1)^n},
\end{equation} si se sustituye esto en la expresión para el estado base, junto con la relación de cerradura para la base de Fock se llega a \cite{EnglertDB}

\begin{equation}
\rho_0^0 = \frac{1}{1+\nu} [\frac{\nu}{1+\nu}]^{a^\dagger a}.
\end{equation}El cual corresponde a un estado térmico\cite{EnglertDB}. A continuación se estudian los estados duales correspondientes al sistema.

\section{Estados Duales de la base de decaimiento}

La base de decaimiento \cite{EnglertDB} definida en \ref{DefDB} corresponde a eigenvalores:

\begin{equation}
 -A[n+\frac{|l|}{2}],
\end{equation}lo cual corresponde a una doble degeneración $\pm l$. Antes de poder realizar la expansión de estados arbitrarios en esta base es necesario conocer los eigenestados del operador dual, los cuales son aquellos que cumplen la condición\cite{EnglertDB}

\begin{equation}
\check{\rho}_\gamma L = \gamma\check{\rho}_\gamma.
\end{equation} Es decir, los eigenestados \textit{izquierdos} del operador de Lindblad. Los estados duales cumplen con

\begin{equation}
Tr[\check{\rho}_{\gamma}\rho_{\gamma'}] = \delta_{\gamma\gamma'}.
\end{equation} Donde la barra denota al estado dual y $\gamma$ denota algún eigenvalor particular del problema. Bajo este producto es posible entonces resolver las ecuaciones para los coeficientes de la expansión de un estado particular en estados de la base de decaimiento

\begin{equation}
\Psi = \sum_{\gamma}\check{c}_\gamma \rho_\gamma.
\end{equation} Entonces

\begin{equation}
\check{c}_\gamma = Tr[\Psi\check{\rho}_{\gamma}].
\end{equation} Con esto la evolución temporal del sistema queda determinada

\begin{equation}
\Psi(t) = \sum_\gamma \check{c}_\gamma e^{\frac{-i}{\hbar} \gamma t} \rho_{\gamma}.
\end{equation} Se trata ahora la forma explícita de los estados duales.

\subsection{Estados Duales: Forma Explícita}

Ya que es necesario tomar trazas sobre los estados duales, es conveniente encontrar tanto su forma explícita en términos de operadores de creación y aniquilación, así como una relación de recurrencia que permita relacionar un estado a los estados de menor número $n$. Los polinomios asociados de Laguerre pueden expresarse de la forma\cite{ArfkenMM}

\begin{equation}\label{DefLaguerre}
L_n^k(x) = \sum_{m=0}^n (-1)^m \frac{(n+k)!}{(n-m)!(k+m)!m!} x^m.
\end{equation} Recordando que en este caso la variable $x$ corresponde a $\frac{a^\dagger a}{\nu}$, y que los estados se encuentran multiplicados por un coeficiente

\begin{equation}
(\frac{-\nu}{1+\nu})^n \frac{n!}{(n+k)!}.
\end{equation}Se llega a la expresión general, la cual toma en cuenta el ordenamiento normal y los operadores de aniquilación a la derecha del mismo

\begin{equation}\label{DefDual}
\sum_{m=0}^n \frac{(-\nu)^{n-m}}{(\nu+1)^n} \frac{n!}{(n-m)!(m+k)!m!} a^{\dagger m}a^{m+k},
\end{equation}basándose en \eqref{DefDual}, se puede ver que para cualquier valor de $n$ y $k \geq 0$ el primer término de la suma, es decir el correspondiente a $m=0$ es

\begin{equation}
(\frac{-\nu}{1+\nu})^n \frac{a^k}{k!}.
\end{equation} El último término de la suma, el correspondiente a $m=n$, es

\begin{equation}
\frac{1}{(\nu+1)^n} \frac{1}{(n+k)!} a^{\dagger n}a^{n+k}.
\end{equation}Si se desea una representación de estos estados en la base de Foch, se puede recurrir a

\begin{equation}
I = \sum^\infty _n \Ket{n}\Bra{n},
\end{equation}Y aplicar todos los operadores desde la derecha. En el caso del primer término ya antes mencionado, se obtiene

\begin{equation}
 (\frac{-\nu}{1+\nu})^n \sqrt{\frac{n!}{(n-k)!}}\sum^\infty _{l=k} \Ket{l-k}\Bra{l}.
\end{equation} Los estados con número de ocupación menor a $k$ se anulan por la definición del operador de aniquilación. En el caso del último término de la suma, se obtiene

\begin{equation}
(\frac{1}{1+\nu})^n \frac{n!}{(n-k)!}\sum^\infty _{l=n+k} \Ket{l-k}\Bra{l}.
\end{equation} Puede verse que ambos términos involucran la misma diferencia en el número de ocupación del ket y del bra, pero la suma correspondiente al último término no cubre tantos estados, pues ahora todo estado con número menor a $n+k$ es anulado.

\subsection{Relación de Recurrencia para Estados Duales}

A fin de obtener una relación de recurrencia para los estados duales, es posible basarse en una de las muchas relaciones de recurrencia de los polinomios asociados de Laguerre \cite{ArfkenMM}:

\begin{equation}
L_n^{k+1}(x) = \frac{n-x}{n}L_{n-1}^{k+1}(x) + \frac{n+k}{n}L_{n-1}^{k}(x).
\end{equation} Si se sustituye esto en la expresión para el estado dual correspondiente se tiene, recordando que se utiliza $x$ para la variable únicamente por simplicidad

\begin{align}
\hat{\rho}_n ^{k+1} =& [\frac{-\nu}{1+\nu}]^n \frac{n!}{(n+k+1)!} :L_n^{k+1}:a^{k+1},\\
=&[\frac{-\nu}{1+\nu}]^n\frac{n!}{(n+k+1)!} :\frac{n-x}{n}L_{n-1}^{k+1}(x) + \frac{n+k}{n}L_{n-1}^{k}(x):a^{k+1},\\
=&[\frac{-\nu}{1+\nu}]^n\frac{n!}{(n+k+1)!}[:L_{n-1}^{k+1}(x)+L_{n-1}^{k}(x):+\frac{1}{n}:-xL_{n-1}^{k+1}(x)+L_{n-1}^{k}(x):]a^{k+1},\\
=&[(\frac{-\nu}{1+\nu})\frac{1}{(n+k+1)}+ \frac{(n+k)}{-\nu(\nu+1)}a^\dagger \bullet a]\hat{\rho}_{n-1} ^{k+1} + [(\frac{-\nu}{1+\nu}) (n+1)(n+k+1)(n+k)]\hat{\rho}_{n-1} ^{k}.
\end{align} Donde $\bullet$ indica que el estado dual va entre el operador de creación y el de aniquilación.

Con esto se puede proceder a atacar problemas de enfriamiento optomecánico. Primero se tratará el problema de un oscilador armónico con frecuencia dependiente del tiempo.


\chapter{Teoría de Floquet Clásica y Cuántica}

Para resolver el problema de obtener la solución para las ecuaciones de movimiento de un oscilador armónico amortiguado con frecuencia dependiente del tiempo, se utiliza la teoría de Floquet \cite{WardFT} y se busca una forma de la ecuación maestra para un tambor mecánico con frecuencia dependiente del tiempo el cual interactúa con el campo electromagnético de una cavidad óptica expresada mediante operadores de Floquet, los cuales se definirán más adelante.

\section{Teoría de Floquet}

Se desea resolver una ecuación diferencial que involucra coeficientes con dependencia temporal, tal como:

\begin{equation}\label{FloquetEquation}
x' = A(t)x,
\end{equation} donde la función $A(t)$ es periódica con periodicidad $\tau$. En este caso el teorema de Floquet\cite{WardFT} dice que la solución no necesariamente es periódica pero debe tener la forma

\begin{equation}\label{FloquetForm}
x(t)=e^{\mu t}p(t).
\end{equation} Los valores $\mu$ se conocen como los exponentes característicos o de Floquet y la función $p(t)$ es periódica con período $\tau$, es decir el mismo periodo que el coeficiente en la ecuación diferencial. Los coeficientes $\mu$ son, en general, complejos. Claramente, el hecho de que la solución tenga la forma \eqref{FloquetForm} puede llevar a que la solución se dispare con el tiempo, por lo que se desea entender el criterio de estabilidad de este tipo de soluciones. Antes de esto, es necesario establecer algunas definiciones y propiedades, las cuales se presentan sin demostración debido a que no son el enfoque principal de este trabajo. Si el lector se encuentra interesado, el tratamiento se encuentra con mayor detalle en las notas de las cuales surge la sección siguiente \cite{WardFT}.

\subsection{Propiedades Básicas}

Sea la ecuación \eqref{FloquetEquation} en $n$ dimensiones. Esto es, se piensa en $x$ como un vector de $n$ dimensiones y en $A(t)$ como una matriz de $n \times n$. En este caso, si la ecuación tiene $n$ soluciones $x_1, x_2, ... , x_n$, se define la \textbf{matriz fundamental} como la matriz formada utilizando las soluciones como columnas, siempre y cuando estas sean linealmente independientes

\begin{equation}
X(t) = [[x_1][x_2]...[x_n]],
\end{equation}Si $X(t_0) = I$ la matriz se conoce como la \textbf{matriz fundamental principal}. Se tiene que

\begin{center}
\textbf{Lema:} \textit{Si $X(t)$ es una matriz fundamental, también lo es $X(t)C$ para cualquier matriz constante y no singular $C$.}
\end{center}Y que

\begin{center}
\textbf{Lema:} \textit{Sea $W(t)$ el Wronskiano de $X(t)$ el determinante de X(t), entonces:}

\begin{equation}
W(t) = W(t_0) e^{\int_{t_0}^{t}tr[A(s)]ds}.
\end{equation}
 
\end{center} Se tiene entonces un teorema

\begin{center}
\textbf{Teorema:} \textit{Sea A(t) una matriz con periodicidad $\tau$. Si $X(t)$ es una matriz fundamental entonces $X(t+\tau)$ también lo es y existe una única matriz constante no singular $B$ tal que:}\linebreak \linebreak i) $X(t+\tau) = X(t)B \qquad\forall t$, \linebreak ii) $det(B) = e^{\int_0^t tr[A(s)]ds}.$
\end{center}Si se toma $X(0)=I$ entonces $B=X(\tau)$. Con esto se pueden definir los \textbf{multiplicadores característicos}, los cuales son los valores propios de la matriz $B$, y se denominan con la letra $\rho$. Estos cumplen que

\begin{equation}
\rho_1 = e^{\mu_1 \tau}, \quad \rho_2 = e^{\mu_2 \tau}, ... , \rho_n = e^{\mu_n \tau},
\end{equation} donde los valores $\mu$ son los exponentes de Floquet definidos anteriormente. Se cumplen cuatro propiedades:

1) Los multiplicadores característicos de $B=X(\tau)$ cumplen que

\begin{equation}
det(B) = \rho_1 \rho_2 ... \rho_n = e^{\int_0^T tr[A(s)]ds}.
\end{equation}

2) Trivialmente, como la traza es la suma de los valores propios

\begin{equation}
Tr[B] = \rho_1 + \rho_2 + ... + \rho_n.
\end{equation}

3) Los multiplicadores característicos no son únicos, ya que

\begin{equation}
e^{\mu \tau} = e^{(\mu  +\frac{2\pi i}{\tau} )\tau}.
\end{equation}

4) Los multiplicadores característicos son una propiedad de la ecuación \eqref{FloquetEquation} y no dependen de la elección de matriz fundamental.

Con estas propiedades, se puede pasar a analizar la estabilidad de las soluciones para el caso específico de ecuaciones de segundo orden.

\subsection{Estabilidad para Ecuaciones de Segundo Orden}\label{EstabilidadSO}

Si se piensa en una ecuación diferencial de segundo orden del tipo

\begin{equation}
\ddot{x} + a(t)x= 0,
\end{equation} donde $a(t)$ tiene periodo $\tau$. Si se toma $x_1 = x$ y $x_2 = \dot{x}$, la ecuación puede re-escribirse como

\begin{equation}
[\begin{array}{c}
\dot{x_1} \\
\dot{x_2}
\end{array}] = [\begin{array}{cc}
0 & 1 \\
-a(t) & 0
\end{array}][\begin{array}{c} 
x_1 \\ 
x_2

\end{array}],
\end{equation} si se toma la condición inicial $[\begin{array}{c} 1 \\ 0 \end{array}]$, se obtiene una solución de la forma

\begin{equation}
[\begin{array}{c}
x_1^1(t) \\
\dot{x_1^1(t)}
\end{array}],
\end{equation} y para la condición inicial $[\begin{array}{c} 0 \\ 1 \end{array}]$, se obtiene una solución de la forma

\begin{equation}
[\begin{array}{c}
x_1^2(t) \\
\dot{x_1^2(t)}
\end{array}],
\end{equation} esto permite generar la matriz $B$

\begin{equation}
B= [\begin{array}{cc}

x_1^1(\tau) & x_1^2(\tau) \\
\dot{x_1^1(\tau)} & \dot{x_1^2(\tau)}

\end{array}],
\end{equation} lo cual permite calcular los multiplicadores característicos, ya que

\begin{equation}
\rho_1 \rho_2 = e^{\int_0^\tau Tr[A(s)]ds} = e^0 = 1,
\end{equation} y

\begin{equation}
\rho_1 + \rho_2 = Tr[B] =x_1^1(\tau)+ \dot{x_1^{(2)}(\tau)} = 2\phi.
\end{equation} Esto permite obtener la ecuación

\begin{equation}
\rho = \phi \pm \sqrt{\phi^2 -1},
\end{equation} o en términos de $\mu$

\begin{equation}
cosh(\mu_1 \tau) = \phi.
\end{equation} Esto lleva a analizar cinco situaciones distintas.

\textbf{Caso $ -1 < \phi < 1$}: En este caso, para algún valor $\sigma$ se tiene que $\phi = cos(\sigma \tau)$ por lo que:

\begin{align*}
\rho =& \phi \pm \sqrt{\phi^2 -1},\\
=& cos(\sigma \tau) \pm isen(\sigma \tau), \\
=& e^{\pm i\sigma \tau},
\end{align*} lo cual lleva a una solución general de tipo:

\begin{equation}
x(t) = c_1 Re(e^{i\sigma t} p(t)) + c_2 Im(e^{i\sigma t} p(t)),
\end{equation} la cual es estable y pseudo periódica.

\textbf{Caso $1 < \phi$:} en este caso $\rho > 1$ y como $\rho_1 = \frac{1}{\rho_2}$, tenemos que $\mu_1 = -\mu_2$. Por esto, la solución es de la forma:

\begin{equation}
x(t) = c_1 e^{\mu_1 t}p_1(t) + c_2 e^{\mu_2 t}p_2(t)
\end{equation} donde las funciones $p(t)$ son periódicas con periodo $\pi$. La solución es inestable.

\textbf{Caso $\phi < -1$:} en este caso se tiene una solución del tipo:

\begin{equation}
x(t) =c_1 e^{\gamma_1 t}q_1(t) + c_2 e^{-\gamma_2 t}q_2(t),
\end{equation} donde las funciones $q(t)$ tienen periodo $2\pi$ y los coeficientes $\gamma = \mu + \frac{i\pi}{\tau}$. La solución de nuevo es inestable.

\textbf{Caso $\phi = -1$:} para este caso también se tiene una solución inestable, de la forma:

\begin{equation}
x(t) = (c_1 + tc_2)q_1(t) + c_2q_2(t)
\end{equation} de nuevo la funciones $q(t)$ tienen periodo $2\pi$.

\textbf{Caso $\phi = 1$:}

para este caso también se tiene una solución inestable, de la forma:

\begin{equation}
x(t) = (c_1 + tc_2)p_1(t) + c_2p_2(t)
\end{equation} de nuevo la funciones $p(t)$ tienen periodo $\pi$.

Es muy importante notar que en estos dos últimos casos, esta forma de la solución solo es correcta si la matriz $B$ tiene un solo eigenvector linealmente independiente. Si este no es el caso, la solución tiene la forma usual con las funciones $p(t)$ o $q(t)$, estos dos casos marcan el límite entre la estabilidad y la inestabilidad en este problema. Finalmente, se verá como estos criterios aplican a una ecuación que será relevante más adelante, la ecuación de Hill.

\subsection{Estabilidad de las Soluciones de Floquet para la Ecuación de Hill}

La ecuación de Hill es una ecuación diferencial de segundo orden con coeficientes dependientes del tiempo de forma periódica\cite{WardFT}

\begin{equation}
\ddot{x}(t) + (\delta + \epsilon b(t))x = 0,
\end{equation} nuevamente, la función $b(t)$ tiene periodo $\tau$ y se considera que $\delta$ y $\epsilon$ son constantes reales. Para el caso $\epsilon = 0$ claramente la ecuación se reduce al oscilador armónico usual y las soluciones son estables. Sin embargo, para ciertos valores de $\delta$ puede encontrarse la región donde la solución aún es periódica, esto se puede resolver para los casos $\phi = \pm 1$, donde $\phi$ es la función definida en la sección \eqref{EstabilidadSO}, de forma que se tiene soluciones estables y periódicas para los casos

\begin{equation}
\delta = (2m\frac{\pi}{\tau})^2, 
\end{equation} que corresponde a $\phi=1$ y

\begin{equation}
\delta = ((2m+1)\frac{\pi}{\tau})^2,
\end{equation} que corresponde a $\phi=-1$. Estos valores representan la frontera de la región de soluciones estables, las cuales corresponden a periodo de $\tau$ y $2\tau$ respectivamente. Más adelante se buscaran soluciones en esta región para el caso donde $\epsilon \ll 1$.


\section{Estados de Floquet en Mecánica Cuántica}

Ahora se busca estudiar Hamiltonianos con una dependencia periódica en el tiempo, donde se utilizaran los resultados obtenidos en la sección anterior

\begin{equation}
H(t)=H(t+\tau).
\end{equation} El hecho de que el Hamiltoniano sea simétrico respecto a (ciertas) traslaciones en el tiempo, permite el uso del formalismo de Floquet \cite{HanngiDQS}. Se asume que la dependencia temporal puede ser vista como una perturbación sobre un Hamiltoniano original

\begin{equation}
H(x,t)=H_0(x)+V(x,t) \qquad V(x,t)=V(x,t+\tau).
\end{equation} Se utiliza que el Hamiltoniano no perturbado posee un conjunto completo de eigenfuciones $\{\phi_n\}$ con valores propios correspondientes $E_n$. En este caso, la ecuación de Schr\"{o}dinger tiene la forma

\begin{equation}\label{SchrodingerEQ}
-i\hbar\dot{\Psi}(x,t) = H(x,t)\Psi(x,t).
\end{equation} El problema cumple con las condiciones necesarias para utilizar una solución del tipo visto en la sección anterior

\begin{equation}
\Psi_n(x,t) = e^{(\frac{-i}{\hbar}\mu_nt)}\Phi_n(x,t).
\end{equation} Como se mencionó en la sección anterior, $\mu$ en general es un número complejo, lo cual puede llevar a soluciones inestables. En este caso $\Phi_n(x,t)$ es la función que contiene la periodicidad en el tiempo. Sustituir la solución en la ecuación \eqref{SchrodingerEQ} genera una ecuación para las funciones periódicas

\begin{equation}
H(x,t)\Phi_n(x,t)=E_n\Phi_n(x,t).
\end{equation} Antes de buscar formas explícitas para estos estados, es necesario resolver el problema clásico correspondiente a este sistema. La razón para esto se verá más adelante, y se debe sencillamente a que estas soluciones clásicas juegan un papel clave en las expresiones explícitas para los estados y operadores involucrados en la solución del problema cuántico.

\section{Oscilador Armónico Dependiente del Tiempo: Solución Mediante Formalismo de Floquet}

En el caso clásico \cite{HanngiFM} se tiene, para un oscilador armónico unidimensional con frecuencia dependiente del tiempo y el cual experimenta una fuerza disipadora dependiente de la velocidad, que la posición cumple

\begin{equation}
\ddot{x}+\gamma\dot{x}+\frac{k(t)}{m}x=0
\end{equation}

Se asume que la función $k(t)$ es periódica con periodo $T$. Si se utiliza la sustitución $x=ye^{-\frac{\gamma t}{2}}$, se llega a la ecuación

\begin{equation}
\ddot{y} +(\frac{k(t)}{m}-\frac{\gamma^2}{4})y=0
\end{equation}

El teorema de Floquet para ecuaciones de segundo orden con coeficientes del tiempo \cite{HanngiFM} asegura que esta ecuación tiene dos soluciones

\begin{equation}
E_1(t) = e^{i\mu t}\phi(t), \quad E_2(t)=E_1(-t),
\end{equation} Recordando que la función $\phi$ debe tener la misma periodicidad que $k(t)$. Dado que la función cumple con esta condición, es posible realizar una expansión de Fourier \cite{ArfkenMM} de la misma

\begin{equation}
\phi(t) = \sum_{-\infty}^\infty c_n e^{in\omega t}.
\end{equation} Para fijar los coeficientes se elije una normalización tal que el Wronskiano sea

\begin{equation}
W = \dot{E}_1(t)E_2(t)-E_1(t)\dot{E}_2(t) = 2i.
\end{equation}Esto genera la regla de suma

\begin{equation}
\sum_{-\infty}^\infty c_n^2(\mu + n\omega) = 1,
\end{equation} y permite, en teoría, calcular las constantes de la expansión para un caso general. A continuación se trata el caso en mecánica cuántica.

\section{Caso Cuántico}

En el caso de un Hamiltoniano con dependencia temporal como la vista anteriormente, existe un conjunto completo de soluciones \cite{BarnettSD}

\begin{equation}
\Ket{\Psi_\alpha (t)} = e^{-i\mu_\alpha t}\Ket{\phi_\alpha t}, \qquad \Ket{\phi_\alpha (t)}=\Ket{\phi_\alpha (t+\tau)},
\end{equation}

Estas soluciones tienen la forma explícita\cite{BrownPT}

\begin{equation}
\Psi_\alpha (x,t) = (\frac{\sqrt{m/\pi\hbar}}{2^\alpha n!E_1^0(t)})^{\frac{1}{2}}(\frac{E_1^0(t)}{E_2^0(t)})^\frac{\alpha}{2}H_\alpha(x\sqrt{\frac{m}{\hbar E_1^0(t) E_2^0(t)}})e^{(ix^2\frac{E_1^0(t)}{2E_2^0(t)})}
\end{equation} donde el superíndice cero indica que se toma el límite donde $\gamma$ tiende a cero. Sin embargo, estas soluciones se comportan de manera análoga a los estados de la base de Fock bajo la acción de los operadores de Floquet, los cuales pueden expresarse en términos de los operadores de momento y posición usuales en mecánica cuántica

\begin{equation}\label{FloquetOperators}
\Gamma(t) = \frac{1}{2i}(\hat{x}\dot{E}_1^0(t)\sqrt{\frac{2}{\hbar m}}-\hat{p}E_1^0(t)\sqrt{\frac{\hbar}{2m}}).
\end{equation} Así como su complejo conjugado. Su acción sobre la base de Floquet queda definida por

\begin{align*}
\Gamma(t) \Ket{\Psi_\alpha (x,t)} =& \sqrt{\alpha}\Ket{\Psi_{\alpha-1} (x,t)}, \\
\Gamma^\dagger(t) \Ket{\Psi_\alpha (x,t)} =& \sqrt{\alpha+1}\Ket{\Psi_{\alpha+1} (x,t)}.
\end{align*}Es importante notar que estos operadores dependen explícitamente del tiempo. Es conveniente entender el origen de estos operadores. Se toma entonces un Hamiltoniano usual de oscilador armónico, con la excepción de que la frecuencia del oscilador es una función periódica del tiempo

\begin{equation}\label{TDHO}
H = \frac{1}{2m}p^2 + \frac{1}{2}k(t)q^2.
\end{equation} Este lleva a la ecuación de movimiento usual

\begin{equation}
m\ddot{q}(t) + k(t)q(t) = 0,
\end{equation} para el operador $q(t)$. Lo que se busca es una transformación unitaria que lleve este problema al problema usual del oscilador armónico en mecánica cuántica. Se trabaja en el cuadro de Heisenberg \cite{SakuraiQM}, tal que

\begin{align}
\tilde{q}(t) =& U^{-1}(t)q(t)U(t),\\
\tilde{p}(t) =& U^{-1}(t)p(t)U(t).
\end{align} Y donde entonces el nuevo Hamiltoniano queda dado por

\begin{equation}
\tilde{H} = H + U^{-1}i\dot{U}.
\end{equation} Para la transformación se elige

\begin{equation}
U = e^{-i\chi(t)q^2(t)},
\end{equation} donde

\begin{equation}
\chi(t) = \frac{m}{4}(\frac{\dot{f}}{f}+\frac{\dot{f^*}}{f^*})
\end{equation} Las funciones $f$ son las soluciones al problema clásico correspondiente al Hamiltoniano \eqref{TDHO} el cual tiene dos soluciones linealmente independientes, pero una es la compleja conjugada de la otra. Estas soluciones corresponden a $E_1^0$  y $E_2^0$ vistas en la sección anterior. Bajo esta transformación

\begin{align}
\tilde{q}(t)=&q(t),\\
\tilde{p}(t)=&p(t)-2\chi(t)q(t).
\end{align}Utilizando esto se puede escribir el Hamiltoniano en las nuevas coordenadas tomando en cuenta que $\ddot{f}= -k(t)f$ y el Wronskiano, $W$

\begin{equation}
 H = \frac{1}{2m}\tilde{p}^2 + \frac{\chi(t)}{m}(\{\tilde{q},\tilde{p}\}) + \frac{mW^2}{|f|^2}k(t)\tilde{q}^2.
\end{equation}Para eliminar el término cruzado se utiliza una segunda transformación

\begin{equation}
U_2(t)=e^{\frac{i}{4}(\{\tilde{q},\tilde{p}\})ln|f|^2}.
\end{equation}Esto es una transformación de escala que deja como variables finales

\begin{align}
Q=&U_2^{-1}\tilde{q}U_2 =\frac{1}{|f|}q(t),\\
P=&U_2^{-1}\tilde{p}U_2 = |f|(p-2\chi q), 
\end{align} en estas variables, el Hamiltoniano es

\begin{equation}\label{QTDHO}
\tilde{H} = \frac{1}{|f(t)|^2}(\frac{1}{2m}P^2(t)+\frac{1}{2}mW^2Q^2(t)).
\end{equation}Este Hamiltoniano es, salvo por un coeficiente general dependiente del tiempo, el Hamiltoniano usual de oscilador armónico y se puede resolver por medio de operadores de escalera

\begin{equation}
\Gamma = \sqrt{\frac{mW}{2}}Q + i \sqrt{\frac{1}{2mW}}P.
\end{equation} La expresión \eqref{FloquetOperators} se obtiene expresando los operadores en las coordenadas usuales, no en las transformadas. Es de mas utilidad expresar este Hamiltoniano en terminos de estos operadores $\Gamma(t)$. Se obtiene

\begin{equation}
\tilde{H} = \frac{W}{|f(t)|^2}(\Gamma^\dagger(t)\Gamma(t) + \frac{1}{2}).
\end{equation} Se puede proceder a encontrar una ecuación maestra para este tipo de sistema.

\section{Ecuación Maestra Mejorada: Solución con Base de Decaimiento}

Se busca obtener una ecuación maestra mejorada al tomar en cuenta la dependencia temporal del Hamiltoniano al realizar la aproximación de Markov \cite{HanngiDQS}. Usualmente, esta dependencia se ignora y se toma en cuenta únicamente de forma posterior, por lo que se puede considerar que esto es insuficiente. Se busca un procedimiento que conserve esta dependencia a fin de que el modelo final también lo haga. Como el Hamiltoniano es explícitamente dependiente del tiempo, no pueden usarse los operadores usuales de creación y aniquilación, es aquí donde entra en juego la teoría de Floquet de la sección anterior. Utilizando dichos operadores, los elementos de matriz del operador de posición son

\begin{align}
X_{\alpha \beta}(t) =& e^{i(\mu_{\alpha}-\mu_{\beta})}\Bra{\phi_{\alpha}(t)}x\Ket{\phi_{\beta}(t)},\\
=& \sum{k} e^{i\Delta_{\alpha\beta k}t} X_{\alpha\beta k},
\end{align}Con

\begin{align*}
X_{\alpha\beta k} =& \frac{1}{T} \int_0^T dt e^{-ik\Omega t}\Bra{\phi_{\alpha}(t)}x\Ket{\phi_{\beta}(t)}, \\
\Delta_{\alpha \beta k} =& \mu_{\alpha}-\mu_{\beta} + k\Omega.
\end{align*}Debido a la periodicidad de los estados de Floquet, es posible hacer un desarrollo en series de Fourier. Con esto, es posible expresar los elementos de matriz de la forma

\begin{equation}
X_{\alpha\beta k} =  \sqrt{\frac{\hbar}{2m}}(\sqrt{\beta}c_{-k} \delta_{\alpha,\beta-1}+\sqrt{\alpha}c_k\delta_{\alpha,\beta+1}).
\end{equation} Si se regresa ahora a la ecuación maestra en el límite de acoplamiento débil, como se se dedujo en el capítulo 3, utilizando un baño de tipo $I(\omega) = m\gamma \omega$

\begin{equation}
\dot{\rho} = -\frac{i}{\hbar}[H(t)_{sis},\rho] + \frac{1}{\pi \hbar} \int_{-\infty}^\infty d\omega I(\omega)n_{th}(\omega)\int_0^\infty e^{i\omega t}[\tilde{x}(t-\tau,t)\rho,x] + C.H..
\end{equation}Aquí $C.H.$ indica el conjugado hermitico de la parte disipadora y $n(\omega)$ es la ocupación térmica de los osciladores del baño. Utilizando que los estados de Floquet son la solución a la parte no disipadora, se llega a la ecuación en representación de Floquet

\begin{align*}
\dot{\rho}_{\alpha \beta} =& \frac{1}{\pi \hbar} \int_{-\infty}^\infty d\omega I(\omega)n_{th}(\omega) \int_0 ^{\infty} d\tau e^{i\omega \tau}, \\
 \times & \sum_{\alpha' \beta '} [X_{\alpha \alpha'}(t-\tau)\rho_{\alpha' \beta'}X^* _{\beta \beta'}(t),\\
 -& X^* _{\alpha' \alpha}(t) X_{\alpha' \beta'}(t-\tau)\rho_{\beta' \beta}] + C.H..
\end{align*} Y se llega a

\begin{align*}
\dot{\rho}_{\alpha \beta} =& \frac{1}{\hbar} \sum_{\alpha' \beta'}\sum_{k k'}[-I(\Delta_{\alpha' \beta' k'}) n_{th} (\Delta_{\alpha' \beta' k'}),\\
\times & e^{i(\Delta_{\alpha' \beta' k'}-\Delta_{\alpha \alpha' k})t} X^*_{\alpha' \alpha k}X_{\alpha' \beta' k'}\rho_{\beta \beta'} + I(\Delta_{\alpha \alpha' k}),\\
\times & n_{th}(\Delta_{\alpha \alpha' k}) e^{i(\Delta_{\alpha \alpha' k}-\Delta_{\beta \beta' k})}X_{\alpha \alpha' k} \rho_{\alpha' \beta'}X^*_{\beta \beta' k'}],\\
+& C. H..  
\end{align*} La función $\Delta_{\alpha \beta k}$ contiene las cuasienergías del sistema sin pérdidas de energía \cite{HanngiDQS}. Se utiliza la aproximación de onda rotante, sin embargo en este sistema no se desprecian todos los términos, ya que $(\alpha-\beta,k)=(\alpha'-\beta',k')$ es suficiente para asegurar que $\Delta_{\alpha \beta k}=\Delta_{\alpha' \beta' k'}$, de esta forma se obtiene la ecuación

\begin{align*}
\dot{\rho}_{\alpha \beta} = \frac{\gamma}{2}&(N+1)[2\sqrt{(\alpha+1)(\beta + 1)}\rho_{\alpha+1,\beta+1} - (\alpha + \beta)\rho_{\alpha \beta}],\\
+& \frac{\gamma}{2}N[2\sqrt{\alpha \beta}\rho_{\alpha-1,\beta-1} - (\alpha + \beta + 2)\rho_{\alpha \beta}].
\end{align*}Utilizando las propiedades de los operadores definidos en \eqref{FloquetOperators} se obtiene la ecuación en términos de operadores

\begin{align*}
\dot{\rho} = -&\frac{i}{\hbar}[H_S (t),\rho] + \frac{\gamma}{2}(N+1)(2\Gamma\rho\Gamma^\dagger - \Gamma^\dagger\Gamma \rho - \rho \Gamma^\dagger \Gamma )\\
 +& \frac{\gamma}{2}N(2\Gamma^\dagger \rho \Gamma - \Gamma \Gamma^\dagger \rho - \rho \Gamma\Gamma^\dagger).
\end{align*} Formalmente, la parte disipadora de la ecuación tiene la misma forma que la ecuación que genera la base \eqref{DefDB}, así que transformando al cuadro de interacción, donde se toma el Hamiltoniano del sistema como \eqref{QTDHO}, el cual puede expresarse como\cite{BrownPT} 

\begin{equation}
H_S = \frac{W}{|f(t)|^2}(\Gamma^\dagger \Gamma + \frac{1}{2}).
\end{equation} Dado que los operadores de Floquet conmutan con este Hamiltoniano, se puede realizar la transformación al cuadro de interacción con el operador unitario \cite{SakuraiQM}:

\begin{equation}
 e^{\frac{i}{\hbar}\frac{W}{|f(t)|^2}t(\Gamma^\dagger \Gamma + \frac{1}{2})},
\end{equation}lo que lleva a la ecuación

\begin{equation} \label{GammaLindblat}
\dot{\rho}=\frac{\gamma}{2}(N+1)(2\Gamma\rho\Gamma^\dagger - \Gamma^\dagger\Gamma \rho - \rho \Gamma^\dagger \Gamma )
 + \frac{\gamma}{2}N(2\Gamma^\dagger \rho \Gamma - \Gamma \Gamma^\dagger \rho - \rho \Gamma\Gamma^\dagger)
\end{equation}La cual queda formalmente resuelta mediante la base de decaimiento. En \cite{HanngiFM} se demuestra que esta ecuación es un mejor modelo de la disipación para el sistema en cuestión. En el capitulo siguiente se utilizará este resultado para modelar enfriamiento optomecánico con coeficientes dependientes del tiempo.


\chapter{Enfriamiento Optomecánico}

\section{Introducción}

Se describe un sistema optomecánico donde un tambor mecánico macroscópico se acopla de forma paramétrica al campo electromagnético de una cavidad óptica que permite enfriar el sistema macroscópico \cite{NooshiLC}. En esencia, se busca un efecto análogo al enfriamiento láser de átomos, pero para la preparación de estados mecánicos puros. El objetivo es enfriar un único modo de un tambor mecánico mediante su interacción con un láser. Por ejemplo, se ha reportado enfriamiento de un micro oscilador armónico de forma toroidal desde temperatura ambiente a 11K mediante técnicas de enfriamiento basadas en la presión de radiación\cite{SchliesserRPC}. En esta tesis se piensa en un sistema equivalente a una cavidad optomecánica de Fabry-Perot, en la cual uno de los dos espejos en los extremos se encuentra a su vez acoplado a un resorte. Este acoplamiento permite la utilización de medios ópticos para controlar el objeto mecánico, los cuales pueden llevar a enfriamiento láser del objeto mecánico\cite{NooshiLC}. En analogía con sistemas de enfriamiento láser de átomos, el movimiento del espejo lleva a que este vea dos frecuencias distintas debido al efecto Doppler\cite{KippenberCO}, la diferencia de las frecuencias del láser y de la frecuencia resonante de la cavidad permiten controlar cual de las dos frecuencias Doppler domina. Una de estas frecuencias lleva a que el sistema mecánico se caliente y la otra lo enfría. En efecto, el láser y su ambiente funcionan como un baño térmico para enfriar al oscilador armónico mecánico. A la absorción de quantos de movimiento (enfriamiento) se le asocia con procesos de dispersión de Raman donde un quanto se sube de nivel mientras que los procesos de emisión (calentamiento) se le asocia con dispersión de Raman donde un fotón decae \cite{LCNooshi}. Este enfriamiento es posible siempre y cuando el ancho de banda de la cavidad sea mucho menor que la frecuencia de oscilación mecánica. \cite{LCNooshi} \cite{MarquardtSC} Este régimen solo es válido cuando el acoplamiento optomecánico puede ser tratado como una perturbación, así que el esquema deja de ser valido cuando el acoplamiento es comparable al ancho de banda de la cavidad o a la frecuencia del oscilador mecánico.

\section{Posibles Sistemas Optomecánicos}

Existen muchas implementaciones posibles de acoplamientos entre elementos ópticos y elementos mecánicos \cite{KippenberCO}. En esta sección se detallan algunas de las posibilidades.

\subsection{Espejos Suspendidos}

Estos sistemas consisten en cavidades ópticas donde uno o más de los espejos pueden cambiar de posición y así alteran la longitud de la cavidad. La primera realización experimental de este tipo de sistemas se debe a los primeros esfuerzos para detectar ondas gravitacionales \cite{AbramoviciLIGO}. El sistema consiste en un interferometro con los espejos fijos en masas suspendidas, a manera que una onda gravitacional, al interactuar con las masas cambiaría la posición de los espejos y así la longitud de camino óptico. El propósito de suspender las masas no es optomecánico, sin embargo, las fluctuaciones en la potencia del láser, debido a la incertidumbre en el número de fotones, son un efecto cuántico que impone un límite a la precisión de las mediciones \cite{CavesIF}. Experimentos en este tipo de sistemas han demostrado varios efectos, entre ellos el enfriamiento mediante presión de radiación \cite{CorbittOC}.

\subsection{Microresonadores}

Otro tipo posible de sistema son los microresonador o microcavidades. En este tipo de sistemas, es posible confinar a la luz a viajar en modos \textit{whispering gallery}, los cuales implican que la luz es guiada a lo largo del perímetro del resonador,el cual puede tener forma esférica, circular, o toroidal\cite{VahalaOM}. Si este vibra, esto puede alterar el camino óptico de la luz y se logra un acoplamiento optomecánico. Debido a su tamaño, es posible obtener acoplamiento fuerte entre sistemas cuánticos y el resonador\cite{VerhagenMOC}.

\subsection{Objetos Suspendidos o Levitados}

En este tipo de sistemas, se considera una cavidad óptica rígida donde se coloca un objeto mecánico dentro de la cavidad. Este esquema permite el acoplamiento de objetos mecánicos de tamaños inferiores a la longitud de onda de la luz \cite{KippenberCO}, como por ejemplo una membrana dieléctrica de $SI_3N_4$ de  $1mm \times 1mm \times 50nm$
de dimensión\cite{SankeyMC}. En ese caso, se puede observar que parámetros de la cavidad como la sintonización y la finesa dependen del desplazamiento de la membrana. Otra posibilidad consiste en un nano cable de carbón, de aproximadamente $10^9$ átomos, el cual se coloca dentro de una micro cavidad de Fabri-Perot. Las propiedades de la cavidad cambian no solo dependiendo de la posición del objeto, sino también de sus modos vibracionales\cite{FaveroCR}.  

\section{Hamiltoniano para Enfriamiento Optomecánico}

En particular, se analiza el sistema cuando la frecuencia natural del tambor mecánico depende del tiempo. Se asume que el oscilador se encuentra acoplado a un único modo forzado de la cavidad el cual tiene frecuencia $\omega_{cav}$. Se asume que el marco de referencia rota con la frecuencia de la bomba. Se modela el sistema mediante el siguiente Hamiltoniano\cite{BarberisLC}

\begin{equation}
H(t) = H_{cav} + H_{mec}(t) + H_{rad} + H_{Bomba}.
\end{equation} En donde

\begin{align}
H_{cav} =& -\hbar \delta a^\dagger a,\\
H_{mec}(t) =& \frac{p^2}{2m} + \frac{1}{2}m \nu (t) x^2,\\
H_{rad} =& -\hbar g a^\dagger a x,\\
Bomba =& \hbar\frac{\Omega}{2}(a^\dagger + a),
\end{align} en este caso, $\delta = w_{bomba} - w_{cav}$ representa la diferencia de frecuencias entre la bomba de fotones y la cavidad y $\hbar g$ representa la fuerza de radiación que un fotón aplica sobre el oscilador mecánico sin modulación. El término $H_{rad}$ modela una interacción simple entre los fotones y el espejo. Dado que en este caso la longitud de la cavidad no es fija, la frecuencia de la cavidad debe tener una dependencia en la coordenada $x$, entonces, tomando el hamiltoniano para el oscilador que modela la cavidad y expandiendo hasta primer orden en $x$ (el resultado obtenido mediante un procedimiento más completo es el mismo)\cite{KippenberCO} 

\begin{equation}
\hbar w_{cav}(x) a^\dagger a \simeq \hbar(w_{cav}-g_0x)a^\dagger a,
\end{equation}lo cual acopla los dos sistemas. Por \eqref{QTDHO}, se modela al oscilador mecánico utilizando operadores de Floquet

\begin{equation}
H_{mec}(t) = \hbar\frac{W}{|f(t)|^2}(\Gamma^\dagger \Gamma + \frac{1}{2}).
\end{equation} Recordando la definición de los operadores de Floquet \eqref{FloquetOperators}, se puede invertir la relación en términos de los operadores $x$ y $p$


\begin{align*}
\Gamma(t) \Ket{\Psi_\alpha (x,t)} =& \sqrt{\alpha}\Ket{\Psi_{\alpha-1} (x,t)}, \\
\Gamma^\dagger(t) \Ket{\Psi_\alpha (x,t)} =& \sqrt{\alpha+1}\Ket{\Psi_{\alpha+1} (x,t)}.
\end{align*}Si se toma la suma y la resta de los operadores

\begin{align*}
2i(\Gamma (t) + \Gamma ^\dagger (t)) =& (\dot{E} (t) - \dot{E}^* (t)) \sqrt{\frac{2m}{\hbar}}x + (E^* (t) - E (t))\sqrt{\frac{2}{m\hbar}} p, \\
2i(\Gamma (t) - \Gamma ^\dagger (t)) =& (\dot{E} (t) + \dot{E}^* (t)) \sqrt{\frac{2m}{\hbar}}x -(E^* (t) + E (t))\sqrt{\frac{2}{m\hbar}} p,
\end{align*} renombrando los coeficientes esto es simplemente

\begin{align*}
2i(\Gamma (t) + \Gamma ^\dagger (t)) =& ax + bp, \\
2i(\Gamma (t) - \Gamma ^\dagger (t)) =& cx - dp,
\end{align*} si se toma la primera ecuación multiplicada por $\frac{d}{b}$ y se suma a la segunda, se obtiene

\begin{align*}
(\frac{da}{b}+c)x =& 2i \frac{d}{b}(\Gamma (t) + \Gamma ^\dagger (t)) + 2i(\Gamma (t) - \Gamma ^\dagger (t)), \\
=&2i[(\frac{d}{b}+1)\Gamma (t)+(\frac{d}{b}-1)\Gamma^\dagger (t)]\\
\therefore x =& \frac{2i[(\frac{d}{b}+1)\Gamma (t)+(\frac{d}{b}-1)\Gamma^\dagger (t)]}{\frac{da}{b}+c},
\end{align*} o lo que es equivalente

\begin{equation}
x = 2i \sqrt{\frac{\hbar}{2m}}[(\frac{A(t) +1}{B(t)})\Gamma (t) +(\frac{A(t) -1}{B(t)})\Gamma^\dagger (t)],
\end{equation} donde

\begin{align}
A(t) =& \frac{d}{b} = \frac{(E^* (t) + E (t))}{(E^* (t) - E (t))}, \\
B(t) = & \frac{da}{b}+c \\
=& \frac{(E^* (t) + E (t))(\dot{E} (t) - \dot{E}^* (t)) \sqrt{\frac{2m}{\hbar}}}{(E^* (t) - E (t))} + (\dot{E} (t) + \dot{E}^* (t)) \sqrt{\frac{2m}{\hbar}}\\
=&\sqrt{\frac{2m}{\hbar}}\frac{(E^* (t) + E (t))(\dot{E} (t) - \dot{E}^* (t))+(E^* (t) - E (t))(\dot{E} (t) + \dot{E}^* (t))}{(E^* (t) - E (t))}.
\end{align}Con esto, es posible sustituir el operador $x$ en el Hamiltoniano de radiación con operadores de Floquet, lo cual produce un nuevo Hamiltoniano

\begin{equation}
H(t)_{rad} = 2ig\sqrt{\frac{\hbar^3}{2m}}  a^\dagger a[(\frac{A(t) +1}{B(t)})\Gamma (t) +(\frac{A(t) -1}{B(t)})\Gamma^\dagger (t)]
\end{equation} con esto, el Hamiltoniano final es el siguiente

\begin{equation}\label{LaserCoolingHamiltonian}
H(t) = -\hbar \delta a^\dagger a + \frac{W}{|f(t)|^2}(\Gamma^\dagger \Gamma + \frac{1}{2}) +  g'a^\dagger a[\gamma_+(t)\Gamma (t) +\gamma_-(t)\Gamma^\dagger (t)] + \hbar\frac{\Omega}{2}(a^\dagger + a),
\end{equation} donde se hicieron las redefiniciones

\begin{align*}
g'=&g\sqrt{\frac{\hbar^3}{2m}},\\
\gamma_+(t)=2i&\frac{A(t) +1}{B(t)},\\
\gamma_-(t)=2i&\frac{A(t) -1}{B(t)},
\end{align*} 

Ahora se desea resolver la ecuación maestra correspondiente a este Hamiltoniano. Los términos de Lindblad que corresponden a este Hamiltoniano son $L_a$ para la cavidad y $L_\Gamma$ para el oscilador, ambos tienen la misma forma funcional, la cual es la de \eqref{GammaLindblat}. Esto lleva a la ecuación maestra

\begin{equation}\label{LCMasterEquation}
\dot{\rho} = \frac{1}{i\hbar}[H,\rho] + L_a\rho + L_\Gamma \rho
\end{equation}Esta ecuación es uno de los resultados originales de este trabajo, ya que esta contiene la dependencia temporal de la frecuencia del oscilador mecánico de forma natural, codificada en los operadores de Floquet. Es un mejor modelo para el sistema, de acuerdo a \cite{HanngiFM} Antes de poder resolver esta ecuación, es necesario realizar una serie de manipulaciones a fin de llegar a una forma más conveniente.

\chapter{Ecuación Maestra de Enfriamiento Laser con Frecuencia Dependiente del Tiempo}

\section{Transformación Mediante Operador de Desplazamiento}

Para poder encontrar una solución a \eqref{LCMasterEquation} es necesario eliminar los términos de tercer orden en operadores, ya que estos son no-lineales y causan dificultades. Esto puede lograrse mediante una transformación unitaria. Esta técnica es análoga a la utilizada para el enfriamiento láser optomecánico usual. Se utiliza la transformación

\begin{equation}
U_{a,\Gamma} = e^{(\alpha(t) a^\dagger - \alpha(t)^*a)}e^{(\beta(t) \Gamma^\dagger - \beta(t)^*\Gamma)},
\end{equation}la cual es unitaria. Es importante notar que tanto $\alpha$ como $\beta$ dependen del tiempo, está dependencia no se escribirá de forma explícita a futuro por brevedad. Bajo la transformación, el operador densidad es

\begin{equation}
\rho' = U_{a,\Gamma}^\dagger \rho U_{a,\Gamma}.
\end{equation} Se puede despejar en términos de $\rho$, utilizando el hecho de que la transformación es unitaria

\begin{equation}
\rho = U_{a,\Gamma} \rho' U_{a,\Gamma}^\dagger,
\end{equation}y derivando respecto al tiempo

\begin{equation}
\dot{\rho} = L\rho = \frac{d}{dt}(U_{a,\Gamma} \rho' U_{a,\Gamma}^\dagger).
\end{equation} En este caso, $L$ representa el operador de Liouville. Esto permite obtener una ecuación maestra para $\rho'$. 

\begin{align}
 U_{a,\Gamma} \dot{(\rho')} U_{a,\Gamma}^\dagger =& L[U_{a,\Gamma} \rho' U_{a,\Gamma}^\dagger] - \dot{U}_{a,\Gamma}\rho'U_{a,\Gamma}^\dagger -U_{a,\Gamma} \rho' \dot{U}_{a,\Gamma}^\dagger\\
\dot{\rho} =& U_{a,\Gamma}^\dagger L[U_{a,\Gamma} \rho' U_{a,\Gamma}^\dagger]U_{a,\Gamma}-U_{a,\Gamma}^\dagger\dot{U}_{a,\Gamma}\rho'-\rho'\dot{U}_{a,\Gamma}^\dagger U_{a,\Gamma}.
\end{align}A partir de este punto, se omiten los sub-índices de las transformaciones y la $'$ para el operador densidad. Para ver los cálculos de la transformación en detalle, ver los apéndices, en esta sección únicamente se ilustrara el procedimiento. Para entender el efecto de la transformación se utiliza la fórmula de Baker-Campbell-Hausdorff \cite{SakuraiQM}, para dos operadores $A$ y $B$

\begin{equation}
e^{A} B e^{-A} = B + [A,B] + \frac{1}{2}[A,[A,B]] + ... .
\end{equation} Debido a que los operadores involucrados si conmutan con sus conmutadores, la serie se corta de manera automática. Aplicando esta regla a los operadores que forman la ecuación maestra se obtiene

\begin{align}
U^{\dagger} a U =& a + \alpha, \\
U^{\dagger} a^{\dagger} U =& a^{\dagger} + \alpha^*, \\
U^{\dagger} \Gamma U =& \Gamma + \beta, \\
U^{\dagger} \Gamma^{\dagger} U =& \Gamma^{\dagger} + \beta^*, 
\end{align} Con esto, los Hamiltonianos transforman en

\begin{align}
H_{cav}' =& -\hbar \delta(a^{\dagger}a +\alpha a^{\dagger}+\alpha^* a + |\alpha|^2),\\
H_{mec}' =& \frac{W}{|f(t)|^2}(\Gamma^{\dagger}\Gamma + \beta \Gamma^{\dagger} + \beta^* \Gamma + |\beta|^2 ),\\
H_{rad}'=&-\hbar g'[(a^{\dagger}a +\alpha a^{\dagger}+\alpha^* a + |\alpha|^2)(\gamma_-(t)(\Gamma^{\dagger}+\beta^*)+\gamma_+(t)(\Gamma+\beta))],\\
B' =& \frac{\hbar \Omega}{2}(a^{\dagger} + a +\alpha + \alpha^*),
\end{align}Los términos de Lindblad que modelan el decaimiento se convierten en

\begin{align}
L_a ' =& L_a + \frac{A}{2}[(\alpha a^\dagger - \alpha^*a),\rho], \\
L_{\Gamma}' =& L_{\Gamma} + \frac{\gamma}{2}[(\beta\Gamma^{\dagger}-\beta^* \Gamma),\rho].
\end{align}Finalmente es necesario tomar en cuenta los términos provenientes de las derivadas temporales. Estos cálculos requieren de especial atención debido a la dependencia temporal de los operadores de Floquet. Los detalles se encuentran en el apéndice. Los términos obtenidos son

\begin{align*}
U^{\dagger}\dot{U}\rho + \rho \dot{U}^\dagger U =& -(\dot{\alpha}a \rho + \rho\dot{\alpha}a^{\dagger}) + \dot{\alpha}a^{\dagger}\rho + \rho \dot{\alpha}^*a,\\
&+ \dot{\beta}\Gamma^{\dagger}\rho + \rho\dot{\beta}^*\Gamma-(\dot{\beta}^*\Gamma + \beta^*\dot{\Gamma})\rho - \rho(\dot{\beta} \Gamma^{\dagger} + \beta \dot{\Gamma}^{\dagger}) +\beta \dot{\Gamma}^{\dagger} + \rho\beta^* \dot{\Gamma},\\
&+3(\beta^*)^2C_{--}(t)\rho + |\beta|^2(C_{+-}(t) - C_{-+}(t))\rho -  \beta^2 C_{++}(t)\rho,
\end{align*}  donde las funciones $C_{\pm \pm}(t)$ son los conmutadores entre los operadores de Floquet y sus derivadas temporales, ya que estos, en general, no son nulos

\begin{align*}
C_{++}(t) =& [\dot{\Gamma}^{\dagger}, \Gamma^{\dagger}],\\
C_{+-}(t) =& [\dot{\Gamma}^{\dagger}, \Gamma],\\
C_{-+}(t) =& [\dot{\Gamma}, \Gamma^{\dagger}],\\
C_{--}(t) =& [\dot{\Gamma}, \Gamma].
\end{align*} 
 
Es interesante notar que toda la dependencia relacionada con estas funciones va acompañada de coeficientes de orden $|\beta|^2$. Por el momento estos términos se tratarán de forma separada. Todos los demás términos se pueden expresar como conmutadores

\begin{equation}
-(U^{\dagger}\dot{U}\rho + \rho \dot{U}^\dagger U) = [\dot{\alpha}^*a-\dot{\alpha}a^\dagger+\dot{\beta}^*\Gamma-\dot{\beta}\Gamma^\dagger,\rho]+ [\beta^*\dot{\Gamma} - \beta \dot{\Gamma}^\dagger,\rho],
\end{equation} lo cual sugiere tratarlos, junto con los términos adicionales que surgen al transformar los términos de Lindblad, como parte del Hamiltoniano. Esto lleva a que el Hamiltoniano efectivo en el marco transformado 

\begin{align}
H_{rot} =& i\hbar(\dot{\alpha}^*a-\dot{\alpha}a^\dagger+\dot{\beta}^*\Gamma-\dot{\beta}\Gamma^\dagger),\\
H_{dec} =& i\hbar(\frac{A}{2}(\alpha a^\dagger - \alpha^*a)+\frac{\gamma}{2}(\beta\Gamma^{\dagger}-\beta^* \Gamma)),\\
H_{temp}=& i\hbar(\beta^*\dot{\Gamma} - \beta \dot{\Gamma}^\dagger),
\end{align} si se escribe el Hamiltoniano total del sistema en el nuevo marco de referencia, se llega a que este es

\begin{align*}
H =& -\hbar \delta(a^{\dagger}a +\alpha a^{\dagger}+\alpha^* a)+\hbar\frac{W}{|f(t)|^2}(\Gamma^{\dagger}\Gamma + \beta \Gamma^{\dagger} + \beta^* \Gamma)\\
 &-\hbar g'[(a^{\dagger}a +\alpha a^{\dagger}+\alpha^* a + |\alpha|^2)(\gamma_-(t)(\Gamma^{\dagger}+\beta^*)+\gamma_+(t)(\Gamma+\beta))]\\
 &+\frac{\hbar \Omega}{2}(a^{\dagger} + a)+i\hbar(\dot{\alpha}^*a-\dot{\alpha}a^\dagger+\dot{\beta}^*\Gamma-\dot{\beta}\Gamma^\dagger)\\
 &+i\hbar(\frac{A}{2}(\alpha a^\dagger - \alpha^*a)+\frac{\gamma}{2}(\beta\Gamma^{\dagger}-\beta^* \Gamma))+i\hbar(\beta^*\dot{\Gamma} - \beta \dot{\Gamma}^\dagger),
\end{align*} aquí es importante notar que se eliminaron los términos que no contenían ningún operador, puesto que estos claramente conmutan con $\rho$ y por lo tanto no contribuyen a la ecuación maestra. Se desea eliminar los términos que únicamente contienen un operador de creación o aniquilación, por lo que se agrupan todos estos términos y se obtienen cuatro ecuaciones

\begin{align}
a&(-\hbar\delta\alpha^* -\hbar g' \alpha^*(\gamma_-(t) \beta^* + \gamma_+(t) \beta)+ \frac{\hbar\Omega}{2} + i\hbar\dot{\alpha}^* -i\hbar\frac{A}{2}\alpha^*),\\
a^\dagger&(-\hbar\delta\alpha -\hbar g' \alpha(\gamma_-(t) \beta^* + \gamma_+(t) \beta)+ \frac{\hbar\Omega}{2} - i\hbar\dot{\alpha} +i\hbar\frac{A}{2}\alpha),\\
\Gamma&(\frac{W}{|f(t)|^2}\beta^*-\hbar g'|\alpha|^2\gamma_- +  i\hbar\dot{\beta}^*+i\hbar\frac{\gamma}{2}\beta^*),\\
\Gamma^\dagger&(\hbar\frac{W}{|f(t)|^2}\beta-\hbar g'|\alpha|^2\gamma_+ -  i\hbar\dot{\beta}-i\hbar\frac{\gamma}{2}\beta),
\end{align} si se desea eliminar todos estos términos, los términos asociados a cada operador deben eliminarse de manera independiente, lo cual genera ecuaciones diferenciales independientes para los coeficientes $\alpha$ y $\beta$. Es importante notar que solo se obtienen dos ecuaciones, ya que dos de las ecuaciones son simplemente los complejos conjugados de las otras dos. Estas son

\begin{align}
\dot{\alpha} =& \alpha(-\frac{A}{2}+i(\delta+g'(\gamma_-(t) \beta^* + \gamma_+(t) \beta))-i\frac{\Omega}{2},\\
\dot{\beta} =& \beta(-\frac{\gamma}{2}-i\frac{W}{|f(t)|^2})+ig'|\alpha|^2\gamma_+(t),
\end{align} si estas ecuaciones se cumplen


\begin{align*}
H'=& -\hbar \delta' a^\dagger a + \frac{W}{|f(t)|^2}\Gamma \Gamma^\dagger -\hbar g'[(a^{\dagger}a +\alpha a^{\dagger}+\alpha^* a)(\gamma_-(t)\Gamma^{\dagger}+\gamma_+(t)\Gamma)]\\
&+ i\hbar(\beta^*\dot{\Gamma} - \beta \dot{\Gamma}^\dagger),
\end{align*}  donde se ha hecho el cambio $\delta' = \delta + g'(\beta + \beta^*)$. Con esto se obtiene la ecuación maestra para el enfriamiento optomecánico con un oscilador con frecuencia dependiente del tiempo, en el marco de referencia desplazado

\begin{equation}\label{DLCMasterEquation}
\dot{\rho} = \frac{1}{i\hbar}[H',\rho] + L_a\rho + L_\Gamma\rho + 3(\beta^*)^2C_{--}(t)\rho + |\beta|^2(C_{+-}(t) - C_{-+}(t))\rho -  \beta^2 C_{++}(t)\rho,
\end{equation} en donde $C(t)$ representa los términos de conmutadores vistos anteriormente. Para poder avanzar más, se requiere una forma específica para la dependencia funcional de la frecuencia del oscilador. Con esta dependencia es posible calcular todos los términos con dependencia temporal.


\section{Solución para Oscilaciones Pequeñas Respecto a una Frecuencia Promedio}

La ecuación \eqref{DLCMasterEquation} depende fuertemente de las soluciones clásicas del problema del oscilador armónico con frecuencia dependiente del tiempo. Estas soluciones no solo determinan la forma exacta de los operadores $\Gamma$, también determinan la forma de los términos denominados $C(t)$. En esta sección se trabajará con el supuesto de que la dependencia temporal en la frecuencia se puede tratar como pequeñas variaciones périódicas en torno a una frecuencia promedio. En esta sección se seguirá el tratamiento en \cite{WardFT}. La ecuación a resolver es la ecuación usual de oscilador armónico, con la excepción de $k(t)$ 

\begin{equation}
k(t) = \nu_0 + \epsilon cos(2\omega t)
\end{equation}Si se hace esta sustitución, en la ecuación de movimiento correspondiente al Hamiltoniano \eqref{TDHO}, se llega a

\begin{equation}
m\ddot{x} + (\nu_0 + \epsilon cos(2\omega t))x = 0,
\end{equation} la cual es muy similar a la ecuación de Mathieu, la cual es un caso particular de la ecuacion de Hill vista anteriormente. A continuación se resuelve el caso donde $\epsilon \ll \nu_0$. En particular, se buscaran soluciones periódicas para los casos donde $\nu_0 \simeq n^2$ con $n$ un número entero. Se resolverá el caso donde $n \simeq 1$. Primero se debe transformar la ecuacion a la forma estandar

\begin{equation}
\ddot{x} + (\delta+\epsilon cos(2t))x=0,
\end{equation} para esto primero se hace un cambio de variables $t' = wt$ lo cual lleva a la ecuación

\begin{equation}
m\omega^2\ddot{x}+(\nu_0+\epsilon cos(2t'))x=0,
\end{equation} y se divide toda la ecuación por $m\omega^2$. Con esto la ecuación queda en la forma deseada si se toma $\epsilon' = \frac{\epsilon}{m\omega^2}$ y se pide $\frac{\nu_0}{m\omega^2} = n^2$, omitiendo las primas por simplicidad, se tiene

\begin{equation} \label{MathieuEquation}
\ddot{x} + (n^2+\epsilon cos(2t))x=0.
\end{equation} Para encontrar una solución a primer orden en $\epsilon$, se hacen desarrollos en términos de esta

\begin{align}
x(t)=& x_0(t) + \epsilon x_1(t) + \epsilon^2 x_2(t) + ... , \\
\delta =& n^2 + \epsilon \delta_1 + \epsilon^2 \delta_2 + ..., 
\end{align} al sustituir estos términos en la ecuación \eqref{MathieuEquation}, se obtienen dos ecuaciones distintas, si únicamente se consideran términos hasta primer orden en $\epsilon$

\begin{align}
\ddot{x_0}& + n^2x_0 = 0, \\
\ddot{x_1}& + n^2x_1 = -\delta_1x_0 -x_0cos(2t) ,
\end{align} al agrupar de acuerdo a la potencia de la dependencia en $\epsilon$. La primera ecuación es simplemente el oscilador armónico usual y su solución es

\begin{equation}
x_0(t) = a cos(nt) + b sin(nt),
\end{equation} esta solución puede sustituirse en la segunda ecuación y se obtiene que

\begin{equation}
\ddot{x_1} + n^2x_1 = -\delta_1(a cos(nt) + b sin(nt)) -(a cos(nt) + b sin(nt))cos(2t),
\end{equation} utilizando identidades trigonometricas, se tiene

\begin{align}
\ddot{x_1} + n^2x_1 =& \nonumber -\delta_1a cos(nt) -\delta_1b sin(nt) -\frac{a}{2}(cos((n+2)t)\\
 &+cos((n-2)t) - \frac{b}{2} ((sin((n+2)t)+sin((n-2)t)),
\end{align} y ahora se toma el caso $n=1$

\begin{equation}
\ddot{x_1} + x_1 = -\delta_1a cos(t) -\delta_1b sin(t) -\frac{a}{2}(cos(3t)+cos(t)) - \frac{b}{2} ((sin(3t)-sin(t)),
\end{equation} para asegurar que la solución no diverja, se debe asegurar que la solución para $x_1$ sea más pequeña que la solución para $x_0$, por lo que se pide que $-\delta_1a - \frac{a}{2}=0$ y que $-\delta_1b + \frac{b}{2}=0$, con esto se obtiene la ecuación

\begin{equation}
\ddot{x_1} + x_1 = -\frac{a}{2}cos(3t)-\frac{a}{2}sin(3t),
\end{equation} se propone una solución en forma de combinación lineal de senos y cosenos. Por independencia lineal de estas funciones, del lado izquierdo solo pueden ser no nulos los términos con $n=3$. Esto lleva a que

\begin{align*}
-\frac{a}{2} =& c_3 - 9c_3, \\
=& 8c_3, \\
\therefore \frac{a}{16} =& c_3,
\end{align*} y de forma completamente análoga para los coeficientes de seno. Con esto, si se toma $a=1$ y $b=i$ para expresar las soluciones de forma exponencial, se tiene que la solución a orden $\epsilon$ es


\begin{equation}
f(t)= E_1^0(t) = e^{i\omega t} + \frac{\epsilon}{16} e^{3i\omega t},
\end{equation} esta función es la función mencionada en el tratamiento de los operadores establecidos en \eqref{FloquetOperators}. La otra solución que se debe obtener según esa sección es la correspondiente a $f(-t)$ que en este caso equivale al complejo conjugado. Esta solución es la que corresponde a $E_2^0(t)$. Con este resultado se pueden calcular los términos correspondientes a los conmutadores de los operadores $\Gamma$ y sus derivadas.

\subsection{Cálculo de Conmutadores}

Se comienza con el conmutador $C_{++}(t)$, recordando que en este contexto $x$ y $p$ son los operadores usuales en mecánica cuántica

\begin{align*}
C_{++}(t) =& [\dot{\Gamma}^\dagger,\Gamma]\\
 =&[\frac{-1}{2i}(x\sqrt{\frac{2m}{\hbar}}\ddot{f}^*-p\sqrt{\frac{2}{m\hbar}}\dot{f}^*),\frac{-1}{2i}(x\sqrt{\frac{2m}{\hbar}}\dot{f}^*-p\sqrt{\frac{2}{m\hbar}}f^*)],\\
 =&[\frac{-1}{2i}x\sqrt{\frac{2m}{\hbar}}\ddot{f}^*,\frac{1}{2i}p\sqrt{\frac{2}{m\hbar}}f^*]+[\frac{1}{2i}p\sqrt{\frac{2}{m\hbar}}\dot{f}^*,\frac{-1}{2i}x\sqrt{\frac{2m}{\hbar}}\dot{f}^*],\\
 =&(\frac{-1}{2i})(\frac{1}{2i})\sqrt{\frac{2m}{\hbar}}\sqrt{\frac{2}{m\hbar}}\ddot{f}^*f^*[x,p]+(\frac{1}{2i})(\frac{-1}{2i})\sqrt{\frac{2m}{\hbar}}\sqrt{\frac{2}{m\hbar}}\dot{f}^*\dot{f}^*[p,x],\\
 =&\frac{1}{2\hbar}(\ddot{f}^*f^*(i\hbar)-\dot{f}^*\dot{f}^*(i\hbar)),\\
 =&\frac{i\omega^2}{2}[(-e^{-i\omega t}-\frac{9\epsilon}{16}e^{-3i\omega t})(e^{-i\omega t}+\frac{\epsilon}{16}e^{-3i\omega t})-(ie^{-i\omega t}+\frac{3i\epsilon}{16}e^{-3i\omega t})(ie^{-i\omega t}+\frac{3i\epsilon}{16}e^{-3i\omega t})],\\ 
 =&\frac{i\omega^2}{2}[-e^{-2i\omega t}-\frac{\epsilon}{16}e^{-4i\omega t}-\frac{9\epsilon}{16}e^{-4i\omega t})-(-e^{-2i\omega t}-\frac{6\epsilon}{16}e^{-4i\omega t})],\\
 =&\frac{i\omega^2}{2}[-\frac{10\epsilon}{16}e^{-4i\omega t}+\frac{6\epsilon}{16}e^{-4i\omega t}],\\
 =&\frac{i\omega^2}{2}[-\frac{4\epsilon}{16}e^{-4i\omega t}],\\
 =&-\epsilon\frac{i}{8}e^{-4i\omega t}.
\end{align*} El factor $\omega^2$ queda absorbido en $\epsilon$, recordando que al transformar la ecuación se dividió por $\omega^2$.  $\epsilon$ se rescribe entonces como $\frac{\epsilon}{m}$. Para calcular el conmutador $C_{--}(t)$ es útil notar que

\begin{align*}
([a,b])^{\dagger} =& (ab-ba)^{\dagger}, \\
=& (b^{\dagger}a^{\dagger}-a^{\dagger}b^{\dagger}),\\
=& [b^{\dagger},a^{\dagger}],\\
=& -[a^{\dagger},b^{\dagger}], 
\end{align*} por lo que

\begin{align*}
([a,b])^{\dagger} =& (C_{++}(t))^{\dagger},\\
=& -[a^{\dagger},b^{\dagger}],\\
\Rightarrow & [a^{\dagger},b^{\dagger}] = -(C_{++}(t))^{\dagger}
\end{align*} lo que lleva a

\begin{equation*}
C_{--}(t) = -\epsilon\frac{i}{8}e^{4i\omega t}.
\end{equation*} El cálculo de los otros dos conmutadores es análogo, y se llega a

\begin{align}
C_{++}(t) =& -\epsilon\frac{i}{8}e^{-4i\omega t},\\
C_{--}(t) =& -\epsilon\frac{i}{8}e^{4i\omega t},\\
C_{+-}(t) =& i [1 -\frac{\epsilon}{16}e^{2i\omega t}-\frac{6\epsilon}{16}e^{-2i\omega t}],\\
C_{-+}(t) =& i [1 -\frac{\epsilon}{16}e^{-2i\omega t}-\frac{6\epsilon}{16}e^{2i\omega t}].
\end{align} con esto, el término correspondiente a los conmutadores es

\begin{equation}
C(t)\rho=-\beta^2\epsilon\frac{i}{8}e^{4i\omega t}\rho + |\beta|^2(-\frac{\epsilon}{16}e^{2i\omega t}-\frac{6\epsilon}{16}e^{-2i\omega t} + \frac{\epsilon}{16}e^{-2i\omega t}-\frac{6\epsilon}{16}e^{2i\omega t})\rho +(\beta^*)^2\epsilon\frac{i3}{8}e^{-4i\omega t}\rho,
\end{equation} ya que $\epsilon$ es en realidad $\frac{\epsilon}{m}$ y todos los términos tienen una amplitud aún menor que esto, se desprecian estos términos. Aún es necesario calcular la forma explícita del resto de los términos con dependencia temporal, ya que todos estos quedan determinados por la solución del problema clásico.

\subsection{Coeficientes $\gamma_\pm$}

Se calcúlan los factores $\gamma_{\pm}(t)$. Para esto primero es necesario calcular los factores $A(t)$ y $B(t)$. Recordando su forma explícita

\begin{align}
A(t) =& \frac{(f^* (t) + f(t))}{(f^* (t) - f(t))}, \\
B(t) =& \sqrt{\frac{2m}{\hbar}}\frac{(f^* (t) + f(t))(\dot{f}(t) - \dot{f}^*(t))+(f^* (t) - f(t))(\dot{f}(t) + \dot{f}^* (t))}{(f^* (t) - f(t))}.
\end{align} Se procede primero con el calculo correspondiente a $A(t)$. Se tiene que los factores son

\begin{align*}
(f^* (t) + f(t)) =& 2cos(\omega t) +\frac{\epsilon}{8}cos(3\omega t), \\
(f^* (t) - f(t)) =& -2isin(\omega t) -i\frac{\epsilon}{8}sin(3\omega t),
\end{align*} si de nuevo se desprecian los términos con amplitud menor que $\epsilon$, se tiene que

\begin{equation}
A(t) = i \frac{cos(\omega t)}{sin(\omega t)}.
\end{equation} Se procede de igual forma y se obtiene, despreciando como siempre los términos menores que $\epsilon$

\begin{align*}
(\dot{f}^* (t) + \dot{f}(t))=&-2\omega sin(\omega t),\\
(\dot{f} (t) - \dot{f}^*(t))=&-2\omega i cos(\omega t),
\end{align*} con lo que se puede calcular el valor de $B(t)$

\begin{equation}
B(t) = \frac{-2\omega }{sen(wt)}.
\end{equation} Recordando ahora la forma explícita de los coeficientes $\gamma_{\pm}$

\begin{align*}
\gamma_+(t)=&-2i\frac{A(t) +1}{B(t)},\\
\gamma_-(t)=&-2i\frac{A(t) -1}{B(t)},
\end{align*} simplemente se sustituyen los valores obtenidos
a
\begin{align}
\gamma_\pm =& 2i\frac{A(t) \pm 1}{B(t)},\\
=& 2i(\frac{i \frac{cos(\omega t)}{sen(\omega t)}\pm 1}{\frac{-2\omega }{sen(wt)}}), \\
=& 2i(\frac{i \frac{cos(\omega t)}{sen(\omega t)}}{\frac{-2\omega }{sen(wt)}} \pm \frac{1}{\frac{-2\omega }{sen(wt)}}), \\
=& 2(\frac{cos(\omega t)}{2\omega} \pm \frac{i sen(\omega t)}{-2\omega}),\\
=& \frac{1}{\omega}e^{\mp i\omega t}.
\end{align}

Falta la forma explícita de los operadores $\dot{\Gamma}$ para poder continuar con los cálculos.

\subsection{Operadores $\dot{\Gamma}$}

Dado que se tiene una forma explícita para la solución clásica, es posible dar una forma explícita de las derivadas temporales de los operadores $\Gamma$. Se puede ver que las derivadas temporales de la solución clásica pueden expresarse de una forma particularmente conveniente

\begin{equation}
\dot{f} = i\omega e^{i \omega t} + 3i\omega\frac{\epsilon}{16} e^{3i\omega t}= i\omega f + 2i\omega \frac{\epsilon}{16} e^{3i\omega t},
\end{equation} y de forma análoga:

\begin{equation}
\ddot{f} = i\omega\dot{f} - 6\omega^2\frac{\epsilon}{16} e^{3i\omega t}.
\end{equation}Si se recuerda la expresión explícita del operador $\Gamma$ y se desprecian los términos menores a $\epsilon$, se llega al resultado

\begin{align}
\dot{\Gamma}(t) =& i\omega \Gamma(t),\\
\dot{\Gamma}^\dagger(t) =& -i\omega \Gamma^\dagger(t),
\end{align} lo cual convierte el término que depende de estas derivadas en el Hamiltoniano en

\begin{equation}
i\hbar(\beta^*\dot{\Gamma}-\beta\dot{\Gamma}^\dagger) = \hbar\omega(\beta^*\Gamma+\beta\Gamma^{\dagger}),
\end{equation} esto lleva necesariamente a modificar las ecuaciones diferenciales de los coeficientes $\beta$ puesto que estos términos deben cancelarse también. Finalmente, es necesario obtener una forma explícita para el factor $\frac{W}{|f|^2}$ que acompaña al oscilador armónico expresado en operadores $\Gamma$.

\subsection{Factores del Hamiltoniano del Oscilador Dependiente del Tiempo}

Se busca calcular el factor $\frac{W}{|f|^2}$, primero se calcula el denominador

\begin{align*}
|f|^2=&ff^*,\\
=& (e^{i\omega t} + \frac{\epsilon}{16} e^{3i\omega t})(e^{-i\omega t} + \frac{\epsilon}{16} e^{-3i\omega t}),\\
=& 1+\frac{\epsilon}{16}(e^{2i\omega t}+e^{-2i\omega t}),\\
\simeq & 1
\end{align*} donde de nuevo se han descartado terminos debido a ser oscilaciones con amplitud menor que $\epsilon$. Falta calcular el Wronskiano $W$,tomando en cuenta que \cite{BrownPT} define como

\begin{equation}
2iW = \dot{f}f^* - f\dot{f}^*,
\end{equation} utilizando los resultados para $\dot{f}$ de la sección anterior, esto puede expresarse como:

\begin{equation}
2iW = i\omega ff^* + i\omega ff^* = 2i\omega ff^*,
\end{equation} al utilizar el resultado apenas obtenido, se llega a

\begin{equation}
W = \omega,
\end{equation} lo que resulta que el factor general entonces, despreciando las correcciones menores a $\epsilon$

\begin{equation}
\frac{W}{|f|^2} = \omega.
\end{equation} Se procede a resolver las ecuaciones para los coeficientes $\alpha$ y $\beta$. Para este caso específico, las ecuaciones a resolver cambian a

\begin{align}
\dot{\alpha} =& \alpha(-\frac{A}{2}+i(\delta+g'e^{i\omega t} \beta^* + e^{-i\omega t} \beta))-i\frac{\Omega}{2},\\
\dot{\beta} =& \beta(-\frac{\gamma}{2}-i 2\omega)+ig'|\alpha|^2e^{i\omega t},
\end{align}

\subsection{Solución para $\alpha$ y $\beta$}

A fin de encontrar una solución analítica, se restringe el estudio al caso estacionario para las ecuaciones, es decir $\dot{\alpha}= \dot{\beta} = 0$. Se asume también que el acoplamiento es débil, por lo cual se toma el caso de menor orden en $g'$. Con estas restricciones, las ecuaciones son

\begin{align}
0 =& \alpha(-\frac{A}{2}+i\delta)-i\frac{\Omega}{2},\\
0 =& \beta(-\frac{\gamma}{2}-i 2\omega),
\end{align} este sistema se puede resolver de manera sencilla

\begin{align}
\alpha_0 =& \frac{\Omega}{2\delta-iA},\\
\beta_0 =& 0.
\end{align} Trabajando con estás soluciones de orden cero en $g'$, el Hamiltoniano es

\begin{equation}
H = -\hbar \delta a^{\dagger}a +\hbar\omega\Gamma^{\dagger}\Gamma -\hbar g'(a^{\dagger}a +\alpha_0 a^{\dagger}+\alpha^*_0 a)(\gamma_-(t)\Gamma^{\dagger}+\gamma_+(t)\Gamma)
\end{equation}

En este caso $|\alpha| \gg 1$, por lo que en el término de interacción se desprecia el término $a^\dagger a$ al ser mucho menor que los otros dos. Con esto, el Hamiltoniano final es

\begin{equation}
H = -\hbar \delta a^{\dagger}a +\hbar\omega\Gamma^{\dagger}\Gamma +\frac{\hbar g'}{\omega}(\alpha_0 a^{\dagger}+\alpha^*_0 a)(e^{i\omega t}\Gamma^{\dagger}+e^{-i\omega t}\Gamma)
\end{equation} este Hamiltoniano es una forma simplíficada la cual es útil para realizar cálculos numéricos. Finalmente, la ecuación maestra correspondiente a este Hamiltoniano es

\begin{equation}\label{LCMasterEq}
\dot{\rho} = \frac{1}{i\hbar}[H,\rho] +L_A\rho + L_\gamma \rho.
\end{equation}

En el siguiente capitulo se explicará de forma breve la teoría de enfriamiento láser y se aplicará para este caso particular.

\chapter{Teoría del Enfriamiento Láser para sistemas Optomecánicos con Frecuencia Dependiente del Tiempo}

La ecuación maestra definida en el capítulo anterior se divide naturalmente en distintas escalas de tiempo ya que $\nu$ y $A$ son normalmente del orden de megahertz y $\gamma$ y $\alpha$ del orden de 100 hertz lo cual lleva a establecer una jerarquía de tiempos donde el movimiento optomecánico es mucho más rápido que la interacción optomecánica, la cual es mucho más rápida que el amortiguamiento mecánico. Se desea dividir la ecuación maestra en términos de un parámetro que se trata como parámetro perturbativo. Como parámetro se toma $\eta=\frac{g'}{4\omega}$. Esto divide la ecuación maestra en dos ordenes

\begin{equation}
L = L_0 + L_1,
\end{equation} con

\begin{align*}
L_0 =& L_c + L_m,\\
L_c =& \frac{1}{i\hbar}[H_c,\cdot] + L_A,\\
L_m =& \frac{1}{i\hbar}[H_m,\cdot] + L_\gamma,\\
L_1 =& \frac{1}{i\hbar}[H_{int},\cdot].
\end{align*} El punto indica la posición de la matríz densidad al aplicar el superoperador. Es decir la evolución libre correspondiente a la cavidad y al oscilador así como los términos de Lindblat correspondientes corresponden a orden cero. La interacción entre ambos sistemas ocurre a primer orden en $\eta$.

\section{Proyectores a Escalas Temporales}

Ya que se desea estudiar el comportamiento del sistema en el estado estacionario, se definen proyectores independientes del tiempo tales que permitan proyectar la ecuación a dicho estado. Se debe recordar por lo visto en la sección de la base de decaimiento que el estado estacionario es el correspondiente al valor propio cero. Estos proyectores son

\begin{align}
P = P_c^{\lambda_c = 0}P_m^{\lambda_m = 0},\\
Q = 1 - P.
\end{align} Se busca encontrar una ecuación maestra válida en el subespacio seleccionado por $P$, el cual corresponde a las escalas de tiempo lentas, mientras que $Q$ selecciona las escalas de tiempo rápidas. Se utiliza una transformación para pasar al cuadro de decaimiento

\begin{align*}
 \rho' =& e^{L_0t}\rho,\\
 L' =& e^{-L_0t}Le^{L_0t} = L_1'.
\end{align*} La ecuación \eqref{LCMasterEq} se proyecta a cada uno de los dos espacios, por simplicidad se omiten las primas, pero se debe tomar en cuenta que se trabaja en el cuadro de decaimiento. En el caso del proyector al espacio $P$

\begin{align*}
P\dot{\rho} =& PL\rho,\\
=& PL(P+Q)\rho,\\
=& PL_1P\rho + PL_1Q\rho,\\
=& PL_1Q\rho.
\end{align*} Aquí se utilizó el hecho de que $PL_0 = L_0P = 0$ debido a que $P$ proyecta a los estados estacionarios,  que $P+Q=1$ de la definición de los proyectores y que $PL_1P=0$ debido a que $L_1$ no concecta estados del espacio $P$. Para el caso de Q se tiene que

\begin{align*}
Q\dot{\rho} =& QL\rho,\\
=& QL(Q+P)\rho,\\
=& QLQ\rho + QLP\rho.
\end{align*} Se resuelve la ecuación del espacio $Q$ a fin de encontrar una ecuación cerrada para el espacio $P$. Esta se puede integrar formalmente como

\begin{align*}
Q\rho =& Q\rho(t_0) + \int_{t_0}^{t}dt' QL(t')P\rho(t')+\int_{t_0}^{t}dt'QL(t')Q\rho(t'),\\
\simeq & Q\rho(t_0) + \int_{t_0}^{t}dt' QL_1(t')P\rho(t_0)+\int_{t_0}^{t}dt'QL_1(t')Q\rho(t_0)+O(\eta^2),
\end{align*} y esto puede sustituirse en la ecuación correspondiente a $P$

\begin{equation}
P\dot{\rho}(t) = PL_1Q\rho(t-\Delta t) + PL_1\int_{t_0}^{t}dt' QL_1(t')P\rho(t-\Delta t) + PL_1\int_{t_0}^{t}dt'QL_1(t')Q\rho(t-\Delta t),
\end{equation} donde únicamente sobrevive el último término. Regresando del cuadro de decaimiento

\begin{equation}\label{ProyectionEQ}
P e^{-L_0 t}L_1e^{L_0 t}\int_{t-\Delta t}^{t}dt'Qe^{-L_0 t}L_1(t')e^{L_0 t}Pe^{-L_0 t}\rho(t-\Delta t).
\end{equation} Es necesario entender como actúa un superador junto con un proyector. Los proyectores pueden expresarse en la base de los vectores propios de los superoperadores, en la cual la acción de los superoperadores es conocida. La forma general de estos proyectores es

\begin{equation}
P = \sum_\lambda \hat{\rho_\lambda}\otimes\bar{\rho_\lambda},
\end{equation} donde la variable $\lambda$ corre sobre el sub espacio sobre el cual se está proyectando. Los dos estados cuyo producto se está tomando son los estados derechos e izquierdos de algún super operador. El producto que se utiliza es el definido anteriormente, basado en la traza. Para algún operador cualquiera $X$

\begin{equation}
PX=\sum_\lambda \hat{\rho_\lambda}\otimes\bar{\rho_\lambda}X = \sum_\lambda \hat{\rho_\lambda}Tr[\bar{\rho_\lambda}X],
\end{equation} si el operador actúa por la izquierda, la traza involucra a los estados izquierdos. En el caso de los super operadores $L_0$, se está proyectando utilizando sus eigenestados, por lo que

\begin{equation}
PL_0=\sum_\lambda \hat{\rho_\lambda}\otimes Tr[\bar{\rho_\lambda}L_0] = \sum_\lambda \hat{\rho_\lambda}\lambda.
\end{equation} Es importante recordar que a ambos estados les corresponde el mismo eigenvalor. Por esto

\begin{equation}
e^{L_0 t} P = \sum_\lambda e^{\lambda t} \hat{\rho}_\lambda.
\end{equation} Para poder realizar la integral temporal en \eqref{ProyectionEQ}, es necesario conocer cierta información sobre $L_1$, específicamente si debe integrarse con respecto al tiempo. En este caso, la dependencia temporal de $L_1$ se encuentra en $\gamma_-(t)\Gamma(t)$ y su complejo conjugado, sin embargo

\begin{equation}
e^{-i\omega t}\Gamma(t) = e^{-i\omega t}\frac{1}{2i}(\hat{x}i\omega \sqrt{\frac{2m}{\hbar}}e^{i\omega t}-\hat{p}\sqrt{\frac{\hbar}{2m}}e^{i\omega t}), 
\end{equation}por lo que la dependencia temporal de $L_1$, a orden más bajo de perturbación, se elimina, y el factor puede salir de la integral. Los proyectores dentro de la integral se desarrollan en términos de sumas. Los índices primados corresponden al espacio correspondinte a $P$ y los no primados a $Q$

\begin{equation}
P e^{-L_0 t}L_1e^{L_0 t}\sum_{\lambda ,\lambda'}\int_{t-\Delta t}^{t}dt'\hat{\rho_\lambda}\otimes\bar{\rho_\lambda}e^{-L_0 t}L_1e^{L_0 t}\hat{\rho_{\lambda'}}\otimes\bar{\rho_{\lambda'}}e^{-L_0 t}\rho(t-\Delta t),
\end{equation} se conoce como actúan las exponenciales sobre los proyectores, lo que lleva a

\begin{equation}
\sum_{\lambda ,\lambda'} P e^{-L_0 t} L_1 \rho_{\lambda} L_1 \rho_{\lambda'} \rho(t-\Delta t) e^{\lambda t} \int_{t-\Delta t}^{t} dt'[ e^{(\lambda'-\lambda)t'}] e^{-\lambda'(t-\Delta t)},
\end{equation} el resultado de la integral es, tomando en cuenta las exponenciales fuera de la integral,

\begin{equation}
\frac{1}{\lambda' - \lambda} [e^{\lambda' \Delta t}-e^{\lambda \Delta t}],
\end{equation} debido a que los índices primados corresponden a los tiempos lentos, el primer factor es aproximadamente uno y el segundo, que corresponde a tiempos rápidos, vale cero. Esto convierte a la ecuacion en

\begin{equation}
P\dot{\rho}=\sum_{\lambda ,\lambda'} P e^{-L_0 t} L_1 \rho_{\lambda} L_1 \rho_{\lambda'} \rho(t-\Delta t)\frac{1}{\lambda' - \lambda} ,
\end{equation} se multiplica por $Pe^{L_0 t}$ y se regresan los proyectores a su estado previo a expresarlos como sumas. Se obtiene

\begin{equation}
P\dot{\rho} = PL_1 \frac{Q}{\lambda_m-L_0}L_1\rho_{st}\mu,
\end{equation} donde se ha asumido que $\rho(t-\Delta t)=rho_{st} \otimes Tr_c[P\rho]$, es decir que los estados iniciales son separables. $\mu$ representa a $Tr_c[P\rho]$. Para obtener la ecuación deseada simplemente se traza sobre los grados de libertad de la cavidad, y así se obtiene una ecuación proyectada a los estados deseados, sobre las variables del oscilador. A continuación se trata el caso específico para la interacción que se ha utilizado hasta ahora. vale la pena recordar

\begin{equation}
L_1\rho= \frac{g'}{i\hbar \omega}[(\alpha_0 a^{\dagger}+ \alpha_0^* a)(e^{-i\omega t}\Gamma(t)+e^{i\omega t}\Gamma(t)^{\dagger}),\rho]
\end{equation} 
 
\section{Solución para Interacción Optomecánica}

Hasta ahora, salvo por las consideraciones de la dependencia temporal de $L_1$, el tratamiento ha sido general. Para poder continuar, se debe tomar en cuenta la forma de la interacción. Por simplicidad se denota $L_1=Fx$ donde $F$ corresponde a la parte de la interacción que opera sobre estados de la cavidad y $x$ a la parte que opera sobre el oscilador. Se tiene entonces

\begin{align*}
=&\frac{-g'^2}{\hbar^2\omega^2} Tr_c[L_1\frac{1}{\lambda_m - L_0}L_1 \rho_{st}\mu],\\
=&\frac{-g'^2}{\hbar^2\omega^2} Tr_c[L_1\frac{1}{\lambda_m - L_0}[Fx, \rho_{st}\mu]],\\
=&\frac{-g'^2}{\hbar^2\omega^2} Tr_c[L_1\frac{1}{\lambda_m - L_0}(Fx\rho_{st}\mu-\rho_{st}\mu Fx)],\\
=&\frac{-g'^2}{\hbar^2\omega^2} Tr_c[L_1\frac{1}{\lambda_m - L_0}x\mu F\rho_{st}]+\frac{1}{\hbar^2}Tr_c[L_1\frac{1}{\lambda_m - L_0}x\mu \rho_{st}F],\\
=&\frac{-g'^2}{\hbar^2\omega^2} Tr_c[[Fx,x\mu]\frac{1}{\lambda_m - L_0} F\rho_{st}]+\frac{1}{\hbar^2}Tr_c[[Fx,\mu x]\frac{1}{\lambda_m - L_0}\rho_{st}F],\\
=&\frac{-g'^2}{\hbar^2\omega^2} Tr_c[(Fxx\mu-x\mu Fx)\frac{1}{\lambda_m - L_0} F\rho_{st}]+\frac{1}{\hbar^2}Tr_c[(Fxx\mu-x\mu Fx)\frac{1}{\lambda_m - L_0}\rho_{st}F],\\
=&\frac{-g'^2}{\hbar^2\omega^2} Tr_c[F\frac{1}{\lambda_m - L_0}F\rho_{st}]xx\mu+\frac{1}{\hbar^2}Tr_c[F\frac{1}{\lambda_m - L_0}F\rho_{st}]x\mu x\\
 &+\frac{1}{\hbar^2} Tr_c[F\frac{1}{\lambda_m - L_0}\rho_{st}F]x\mu x-\frac{1}{\hbar^2}Tr_c[F\frac{1}{\lambda_m - L_0}\rho_{st}F]\mu xx,\\
=&-T_1xx\mu+T_1x\mu x + T_2x\mu x - T_2\mu xx.
\end{align*} Se debe de calcular los términos con operadores fuera de la traza

\begin{align*}
xx\mu=& (x_++x_-)(x_++x_-)\mu,\\
=&x_+x_+\mu + x_+x_-\mu + x_-x_+\mu + x_-x_-\mu, 
\end{align*} donde $x_- = \gamma_- \Gamma$ y $x_+ = \gamma_+ \Gamma^{\dagger}$. De esta forma, los términos son

\begin{align}
xx\mu=&x_+x_+\mu + x_+x_-\mu + x_-x_+\mu + x_-x_-\mu,\\
x\mu x=&x_+\mu x_+ + x_+\mu x_- + x_-\mu x_+ x_-\mu x_-,\\
\mu xx=& \mu x_+x_+ + \mu x_+x_-+ \mu x_-x_+ +\mu x_-x_-,
\end{align} sin embargo, $\mu$ corresponde a una proyección a estados diagonales, por lo que solo sobreviven los términos que corresponden a un operador de creación y uno de aniquilación. Los términos sobrevivientes son entonces

\begin{align}
=&(T_1+T_2)x_+\mu x_- -T_1 x_+x_-\mu - T_2\mu x_+x_- \\ 
&+ \nonumber (T_1+T_2)x_-\mu x_+  -T_1 x_-x_+\mu -T_2 \mu x_-x_+.
\end{align} Ya que estos operadores tienen el mismo comportamiento algebráico que el utilizado en \cite{BarberisLC}, esto puede expresarse como

\begin{equation}
\dot{\mu} = \frac{1}{i\hbar}[\hat{H},\mu] + \frac{A_-}{2}D[\Gamma]\mu + \frac{A_+}{2}D[\Gamma^\dagger]\mu.
\end{equation} En este caso, $\hat{H}$ es una pequeña corrección al Hamiltoniana proporcional a $\Gamma^\dagger \Gamma$ y $A_\pm$ dependen únicamente del valor de las trazas. En este caso son funcionalmente equivalentes al trabajo citado. Sin embargo, se desea estudiar el número de excitaciones del tambor mecánico, por lo que es necesario regresar a la notación de operadores usuales para oscilador armónico. Para este fin se toma la frecuencia del oscilador como una frecuencia promedio $\nu$. Con esto, los operadores $\Gamma$ pueden expresarse de la forma deseada simplemente susituyendo los operadores $x$ y $p$ por sus equivalentes en operadores de escalera del oscilador armónico

\begin{equation}
\Gamma = \frac{1}{2i}(\dot{f} \frac{1}{\sqrt{\nu}} (b^\dagger + b) - fi\sqrt{\nu}((b^\dagger + b))),
\end{equation} y al insertar la solución específica del problema

\begin{equation}
\Gamma = \frac{\omega+\nu}{2\sqrt{\nu}}e^{i\omega t}b+\frac{\omega-\nu}{2\sqrt{\nu}}e^{i\omega t}b^\dagger,
\end{equation} y $\Gamma^\dagger$ se obtiene tomando el complejo conjugado. Para obtener el número vibracional medio se toma

\begin{equation}
<m> = Tr[\Gamma^\dagger \Gamma \mu_{st}] = \frac{A_+}{A_+ - A_-},
\end{equation} de acuerdo con \cite{BarberisLC}. Este número corresponde a la temperatura final del sistema. Al cambiar a los operadores usuales de oscilador, el resultado ahora es

\begin{equation}
<m> = Tr[(\frac{(\omega + \nu )^2}{4\nu})b^\dagger b +\frac{\omega^2 - \nu^2}{4\nu}(b^\dagger b^\dagger +bb) +\frac{(\omega - \nu )^2}{4\nu}) b b^\dagger)\mu_st].
\end{equation} Esto establece una nueva predicción para la temperatura final del sistema en el caso del enfriamiento optomecánico dependiente del tiempo.


\chapter{Conclusiones}

Se aplicó el formalismo de operadores de Floquet al caso de enfriamiento optomecánico y se obtuvo una ecuación maestra con un término que modela la discipación de un tambor mecánico con parámetros dependientes del tiempo de mejor manera.  Se logró demostrar que la base de decaimiento resuelve la ecuación maestra correspondiente a un tambor mecánico con frecuencia natural dependiente del tiempo con pérdidas de energía si esta se expresa en los estados propios de los operadores de Floquet.

\begin{itemize}
\item En el caso de la ecuación maestra correspondiente al tambor mecánico dependiente del tiempo, se mostró que se puede construir una ecuación maestra con la misma estructura algebráica con nuevos operadores que toman en cuenta la dependencia temporal del sistema. Esta forma permite aplicar los formalismos conocidos para obtener mejores resultados. Esta dependencia temporal se toma en cuenta cuando se realiza la aproximación de Markov, lo cual es más consistente a fin de modelar el sistema de la forma menos aproximada posible. La ecuación obtenida, por construcción, es un mejor modelo.

\item Siguiendo el procedimiento establecido en \cite{BarberisLC}, se logró obtener un número promedio de ocupación para los modos vibracionales del oscilador, lo cual corresponde a una temperatura final del sistema. Al traducir este número a los operadores usuales del oscilador armónico, se encuentra un resultado distinto al obtenido en trabajos anteriores.

\item Todo el trabajo específico se hizo al más bajo nivel de perturbación con respecto a la solución clásica del oscilador dependiente del tiempo. Aún cuando se muestra una nueva dinámica al hacer el tratamiento presentado en este trabajo, parece existir más dinámica si se logra tratar el problema a niveles más altos de perturbacion.
\end{itemize}

Como trabajos futuro, se busca extudiar en más detalle las soluciones de la ecuación obtenida. Se creará un modelo numérico a fin de poder explorar el comportamiento de la solución obtenida y compararla con las soluciones obtenidas mediante el formalismo utilizado en trabajos como \cite{BarberisLC}. Así mismo, se buscará estudiar soluciones con menos restricciones en los valores de los parámetros y a mayores ordenes de perturbación.

\appendix

\chapter{Transformación al Marco Desplazado}

En este apéndice se dan los cálculos detallados para la transformación al marco desplazado. Primero se realiza la transformación del Hamiltoniano del sistema, luego se transforma los términos de Lindblad y finalmente se realizan las derivadas temporales.

\section{Términos Hamiltonianos}

Para hacer la transformación al marco desplazado, primero se recuerda el operador mediante el cual se realiza la transformación

\begin{equation}
U_{a,\Gamma} = e^{(\alpha a^\dagger - \alpha^*a)}e^{(\beta \Gamma^\dagger - \beta^*\Gamma)},
\end{equation} y su complejo conjugado

\begin{equation}
U_{a,\Gamma}^{\dagger} = e^{-(\alpha a^\dagger - \alpha^*a)}e^{-(\beta \Gamma^\dagger - \beta^*\Gamma)},
\end{equation} donde claramente

\begin{equation}
UU^{\dagger} = 1.
\end{equation} Primero se debe obtener como transforman los operadores individuales. En el caso del operador $a$, utilizando la formula de Baker-Campbell-Hausdorff

\begin{equation}
U^{\dagger}aU = a +[a,(\alpha a^\dagger - \alpha^*a)] = a +\alpha[a,a^{\dagger}]=a+\alpha.
\end{equation} Aquí se omitió el conmutador con los coeficientes $\beta$ al ser trivialmente cero. Debido a las reglas de conmutación de los operadores de creación y aniquilación la serie solo contiene un término. Si se toma el complejo conjugado

\begin{equation}
U^{\dagger}a^{\dagger}U = a^\dagger + \alpha^*.
\end{equation} Los operadores $\Gamma$ siguen las mismas reglas de conmutación y por lo tanto el resultado es completamente analogo

\begin{align}
U^{\dagger}\Gamma U =& \Gamma + \beta, \\
U^{\dagger}\Gamma^\dagger U =& \Gamma^\dagger + \beta^*.
\end{align} También es necesario transformar el operador de número, primero se calcula el conmutador

\begin{align*}
[a^\dagger a,(\alpha a^\dagger - \alpha^*a) ] =& [a^\dagger a,\alpha a^\dagger] - [a^\dagger a,\alpha^*a],\\
=&\alpha [a^\dagger a,a^\dagger] - \alpha^*[a^\dagger a,a],\\
=&\alpha a^{\dagger} + \alpha^* a.
\end{align*} Entonces es necesario calcular el conmutador

\begin{align*}
[\alpha a^{\dagger} + \alpha^* a,\alpha a^\dagger - \alpha^*a] = |\alpha|^2[a^{\dagger},a]-|\alpha|^2[a,a^{\dagger}],\\
=2|\alpha|^2.
\end{align*} Al sustituir esto en la serie, se obtiene

\begin{equation}
U^\dagger a^\dagger a U = a^\dagger a +\alpha a^\dagger + \alpha^* a + |\alpha|^2,
\end{equation} y de manera completamente equivalente

\begin{equation}
U^\dagger \Gamma^\dagger \Gamma U = \Gamma^\dagger \Gamma +\beta \Gamma^\dagger + \beta^* \Gamma + |\beta|^2,
\end{equation}. Utilizando estas reglas de transformación, los términos Hamiltonianos son

\begin{align}
H_{cav}' =& -\hbar \delta(a^{\dagger}a +\alpha a^{\dagger}+\alpha^* a + |\alpha|^2),\\
H_{mec}' =& \frac{W}{|f(t)|^2}(\Gamma^{\dagger}\Gamma + \beta \Gamma^{\dagger} + \beta^* \Gamma + |\beta|^2 ),\\
H_{rad}'=&-\hbar g'[(a^{\dagger}a +\alpha a^{\dagger}+\alpha^* a + |\alpha|^2)(\gamma_-(t)(\Gamma^{\dagger}+\beta^*)+\gamma_+(t)(\Gamma+\beta))],\\
B' =& \frac{\hbar \Omega}{2}(a^{\dagger} + a +\alpha + \alpha^*).
\end{align}

Se procede a transformar los términos de Lindblad.

\section{Términos de Lindblad}

Se recuerda que los términos correspondientes son

\begin{equation}
U^\dagger [L_aU\rho U^\dagger]U + U^\dagger [L_\Gamma U\rho U^\dagger]U,
\end{equation} se procede primero con $L_a$. Se recuerda la forma explícita de $L_a \rho$

\begin{equation}
L_a\rho = 2a\rho a^\dagger - a^\dagger a \rho - \rho a^\dagger a,
\end{equation} por lo que

\begin{align*}
U^\dagger [L_aU\rho U^\dagger]U =& U^\dagger(2aU\rho U^\dagger a^\dagger - a^\dagger a U\rho U^\dagger - U\rho U^\dagger a^\dagger a)U,\\
=& 2U^\dagger a U \rho U^\dagger a^\dagger U - U^\dagger a^\dagger a U \rho -\rho U^\dagger a^\dagger a U, \\
=& 2(a+\alpha)\rho(a^\dagger+\alpha^*)-\{a^\dagger a +\alpha a^\dagger + \alpha^* a + |\alpha|^2,\rho \},\\
=& 2a\rho a^\dagger + 2\alpha^* a\rho +2\alpha \rho a^\dagger + 2|\alpha|^2\rho -\{a^\dagger a +\alpha a^\dagger + \alpha^* a + |\alpha|^2,\rho \},\\
=& 2a\rho a^\dagger + 2\alpha^* a\rho - a^\dagger a \rho -\alpha a^\dagger \rho -\rho a^\dagger a -\alpha^* \rho a,\\
=& L_a\rho + [\alpha^* a - \alpha a^\dagger].
\end{align*} En términos de forma algebráica, el término $L_\Gamma$ funciona de la misma forma, y se transforma de igual manera. Con esto se llega a

\begin{align}
U^\dagger [L_aU\rho U^\dagger]U=& L_a\rho + [\alpha a^\dagger - \alpha^* a,\rho],\\
U^\dagger [L_\Gamma U\rho U^\dagger]U =& L_\Gamma \rho + [ \beta \Gamma^\dagger - \beta^* \Gamma,\rho].
\end{align} La transformación es equivalente al término usual de Lindblad más un Hamiltoniano efectivo adicional. Finalmente, se calculan las derivadas temporales.

\section{Derivadas temporales de $U$ y $U^\dagger$}

Para poder derivar el operador $U$, es necesario utilizar la identidad:

\begin{equation}
e^{A+B} = e^A e^B e^{\frac{1}{2}[A,B]},
\end{equation} con $A$ y $B$ operadores. Esto es válido únicamente cuando tanto $A$ y $B$ conmutan con $[A,B]$. Utilizando esto es posible expresar $U_a$ como

\begin{equation}
U_a = e^{\alpha a^\dagger}e^{\alpha^* a}e^{-\frac{|\alpha|^2}{2}},
\end{equation} y de manera análoga

\begin{equation}
U_\Gamma = e^{\beta \Gamma^\dagger}e^{\beta^* \Gamma}e^{-\frac{|\beta|^2}{2}}.
\end{equation} Primero se nota que

\begin{equation}
\dot{U} = \dot{U_aU_\Gamma} = \dot{U}_a U_\Gamma +  U_a \dot{U}_\Gamma,
\end{equation} por lo que se calcula cada derivada por separado. Para el caso de $U_a$

\begin{align*}
\dot{U}_a =& \dot{(e^{\alpha a^\dagger})}e^{\alpha^* a}e^{-\frac{|\alpha|^2}{2}}
+e^{\alpha a^\dagger}\dot{(e^{\alpha^* a})}e^{-\frac{|\alpha|^2}{2}}+e^{\alpha a^\dagger}e^{\alpha^* a}\dot{(e^{-\frac{|\alpha|^2}{2}})},\\
=&\dot{\alpha} a^\dagger U_a - U_a\dot{\alpha}^* - \frac{1}{2}(\dot{\alpha} \alpha^*+\dot{\alpha}^* \alpha).
\end{align*} Tomar el complejo conjugado es equivalente a cambiar el signo de los coeficientes $\alpha$ y $\alpha^*$, por lo que inmediatamente se puede obtener $\dot{U}_a^\dagger$, tenemos entonces

\begin{align}
\dot{U}_a=&\dot{\alpha} a^\dagger U_a - U_a\dot{\alpha}^*a - \frac{1}{2}(\dot{\alpha} \alpha^*+\dot{\alpha}^* \alpha)U_a,\\
\dot{U}_a^\dagger=&-\dot{\alpha} a^\dagger U_a^\dagger + U_a^\dagger\dot{\alpha}^*a - \frac{1}{2}(\dot{\alpha} \alpha^*+\dot{\alpha}^* \alpha)U_a^\dagger.
\end{align} Se realiza un procedimiento similar con $U_\Gamma$, pero aquí debe tenerse en cuenta que los operadores $\Gamma$ dependen explícitamente del tiempo. También es importante tener en cuenta que mientras en el cálculo anterior esto no es problemático, para el caso de los operadores $\Gamma$, al aparecer derivadas de los operadores temporales no puede asumirse necesariamente que los operadores conmuten con sus derivadas. 

\begin{align*}
\dot{U}_\Gamma =& \dot{(e^{\beta \Gamma^\dagger})}e^{\beta^* \Gamma}e^{-\frac{|\beta|^2}{2}}
+e^{\beta \Gamma^\dagger}\dot{(e^{\beta^* \Gamma})}e^{-\frac{|\beta|^2}{2}}+e^{\beta \Gamma^\dagger}e^{\beta^* \Gamma}\dot{(e^{-\frac{|\beta|^2}{2}})},\\
=&(\dot{\beta}\Gamma^\dagger +\beta \dot{\Gamma}^\dagger)U_\Gamma -e^{\beta \Gamma^\dagger}(\dot{\beta}^*\Gamma +\beta^* \dot{\Gamma})e^{\beta^* \Gamma}e^{-\frac{|\beta|^2}{2}} -\frac{1}{2}(\dot{\beta} \beta^*+\dot{\beta}^* \beta)U_\Gamma,\\
=&(\dot{\beta}\Gamma^\dagger +\beta \dot{\Gamma}^\dagger)U_\Gamma - U_\Gamma\dot{\beta}^*\Gamma-U_\Gamma\beta^* \dot{\Gamma} + (\beta^*)^2 C_{--}(t)U_\Gamma-\frac{1}{2}(\dot{\beta} \beta^*+\dot{\beta}^* \beta)U_\Gamma.
\end{align*} En este caso $C_{--}(t)=[\dot{\Gamma},\Gamma]$. Se utilizó la identidad

\begin{equation}
[A,F(B)]=[A,B]F'(B).
\end{equation} La cual es válida si $A$ y $B$ conmutan con $[A,B]$, para encontrar que

\begin{equation}
\beta^*\dot{\Gamma} e^{-\beta^* \Gamma} = e^{-\beta^* \Gamma}\beta^*\dot{\Gamma} + (\beta^*)^2 C_{--}(t) e^{-\beta^* \Gamma}.
\end{equation} Para todos los casos, $C_{\pm \pm}(t)$ son funciones dependientes del tiempo pero no contienen operadores. Se puede encontrar $U^\dagger$ de forma completamente equivalente. Con esto, se tiene que

\begin{align}
\dot{U}_\Gamma =&(\dot{\beta}\Gamma^\dagger +\beta \dot{\Gamma}^\dagger)U_\Gamma - U_\Gamma\dot{\beta}^*\Gamma-U_\Gamma\beta^* \dot{\Gamma} + (\beta^*)^2 C_{--}(t)U_\Gamma-\frac{1}{2}(\dot{\beta} \beta^*+\dot{\beta}^* \beta)U_\Gamma, \\
\dot{U}^\dagger_\Gamma=&-(\dot{\beta}\Gamma^\dagger +\beta \dot{\Gamma}^\dagger)U_\Gamma^\dagger + U_\Gamma^\dagger\dot{\beta}^*\Gamma+U_\Gamma^\dagger\beta^* \dot{\Gamma} + (\beta^*)^2 C_{--}(t)U_\Gamma^\dagger-\frac{1}{2}(\dot{\beta} \beta^*+\dot{\beta}^* \beta)U_\Gamma^\dagger.
\end{align} Únicamente falta calcular las transformaciones de las derivadas temporales de los operadores $\Gamma$. Estas se calculan de la misma forma y resultan

\begin{align}
U^{\dagger}\dot{\Gamma}U =& \dot{\Gamma} + \beta C_{-+}(t) +\beta^* C_{--}(t),\\
U^{\dagger}\dot{\Gamma}^\dagger U =& \dot{\Gamma}^\dagger + \beta^* C_{+-}(t) +\beta C_{++}(t). 
\end{align} Con esto, tomando en cuenta que la expresión total es

\begin{equation}
U^\dagger \dot{U} \rho + \rho \dot{U}^\dagger U = U^\dagger(U_a \dot{U}_\Gamma+\dot{U}_a U_\Gamma)\rho + \rho(U_a^\dagger \dot{U}_\Gamma^\dagger+\dot{U}_a^\dagger U_\Gamma^\dagger)U,
\end{equation} simplemente se debe sustituir, ya que todos los términos son conocidos. Se llega a la expresión final que puede expresarse en términos de conmutadores con $\rho$

\begin{align}
U^\dagger \dot{U} \rho + \rho \dot{U}^\dagger U =& [\dot{\alpha}a^\dagger - \dot{\alpha}^* a,\rho] +[\dot{\beta}\Gamma^\dagger - \dot{\beta}^* \Gamma,\rho]+  [\beta\dot{\Gamma}^\dagger - \beta^* \dot{\Gamma},\rho]\\
&-\nonumber\beta^2C_{++}(t)\rho + |\beta|^2(C_{+-}(t) - C_{-+}(t))\rho +3(\beta^*)^2C_{--}(t)\rho
\end{align}




\bibliographystyle{unsrt}
\bibliography{Bib}


\end{document}