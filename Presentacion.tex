\documentclass[11pt]{beamer}
\usetheme{Berkeley}
\usepackage[utf8]{inputenc}
\usepackage[spanish]{babel}
\selectlanguage{spanish}
\usepackage{amsmath}
\usepackage{amsfonts}
\usepackage{amssymb}
\author{Pablo Enrique Yanes Thomas}
\title{Enfriamiento Optomecánico con Parámetros Dependientes del Tiempo}
%\setbeamercovered{transparent} 
%\setbeamertemplate{navigation symbols}{} 
%\logo{} 
%\institute{} 
%\date{} 
%\subject{} 
\begin{document}

\begin{frame}
\titlepage
\end{frame}

%\begin{frame}
%\tableofcontents
%\end{frame}

\begin{frame}{Resumen}

\begin{itemize}
\item Se modeló el enfriamiento de un tambor mecánico con frecuencia natural dependiente del timpo acoplado al campo electromagnético de una cavidad óptica

\item Se encontró un mejor modelo de disipación.

\item Se dedujo una mejor ecuación maestra para modelar el sistema bajo la aproximación adiabática.

\item Se hizo una predicción para el número promedio de excitaciones del tambor.
\end{itemize}

\end{frame}

\begin{frame}{Aproximaciones}
\begin{itemize}
\item Acoplamiento débil

\item Proceso Markoviano

\item Frecuencia natural del tambor oscila poco en torno a una frecuencia central

\item Interacción mucho más lenta que evolución libre
\end{itemize}

\end{frame}

\begin{frame}{Sistemas Optomecánicos}
Interacción entre un sistema mecánico y un sistema óptico, hay varios tipos posibles:

\begin{itemize}
\item Espejos suspendidos
\item Microresonadores
\item Objetos Suspendidos
\end{itemize}

Se trató con un sistema del primer tipo.

\end{frame}

\begin{frame}{El Sistema}

Se estudió un sistema equivalente a una cavidad de Fabry-Perót con uno de los espejos sujetos a un resorte. Este se acopla de manera paramétrica al campo electromagnético de la cavidad. Se estudió un régimen donde el acoplamiento es débil.

\end{frame}

\begin{frame}{Estudios Anteriores}

El sitema se ha estudiado antes \cite{BarberisLC}, sin embargo la dependencia temporal no se tomó en cuenta durante la aproximación de Markov. Existe un procedimiento que sí toma esto en cuenta y lleva a una mejor descripción \cite{HanngiFM}. Se siguió este procedimeitno y se llegó a una mejor ecuación maestra para describir este sistema particular.

\end{frame}

\begin{frame}{Disipación para Oscilador Dependiente del Tiempo}

\textbf{Base de Decaimiento}

Base de vectores propios del operador de Lindblad que se obtiene en la ecuación:

\begin{align}\label{OsciladorLindblad}
\textbf{L}_a \rho =& - \frac{A}{2}(\nu + 1)[a^\dagger a\rho + \rho a^\dagger a -2a\rho a^\dagger] \nonumber \\
 &- \frac{A}{2}(\nu)[ aa^\dagger\rho + \rho  aa^\dagger -2a^\dagger\rho a].
\end{align} Estos son de la forma:
\begin{align}\label{DampingBasis}
&a^{\dagger l}\frac{(-1)^n}{(\nu+1)^{l+1}}:L_n^l[\frac{a^\dagger a}{\nu+1}]e^{-[\frac{a^\dagger a}{\nu+1}]}:\quad l \geq 0, \\
&\frac{(-1)^n}{(\nu+1)^{|l|+1}}:L_n^{|l|}[\frac{a^\dagger a}{\nu+1}]e^{-[\frac{a^\dagger a}{\nu+1}]}:a^{|l|}\quad l \leq 0,
\end{align}
\end{frame}

\begin{frame}
\textbf{Oscilador Armónico Dependiente del Tiempo}

Se resuelve clásicamente mediante teoría de Floquet\cite{WardFT}. En este trabajo se trató el caso donde la frecuencia natural es de la forma:
\begin{equation}
k(t) = \nu_0 + \epsilon cos(2\omega t).
\end{equation} La solución clásica es
\begin{equation}
f(t)= e^{i\omega t} + \frac{\epsilon}{16} e^{3i\omega t}
\end{equation}

\end{frame}


\begin{frame}
\textbf{Operadores de Floquet}

Equivalentes a los operadores de escalera del oscilador armónico para el caso donde la frecuencia natural del oscilador es una función periódica del tiempo. Dependen de la solución clásica del oscilador mecánico dependiente del tiempo. Estos son

\begin{equation}
\Gamma(t) = \frac{1}{2i}(\dot{f}(t)\sqrt{\frac{2}{\hbar m}}\hat{x}-f(t)\sqrt{\frac{\hbar}{2m}}\hat{p}),
\end{equation} y su complejo conjugado

\end{frame}

\begin{frame}
Se tiene un mejor modelo de disipación al sustituir operadores de Floquet en la ecuación \eqref{OsciladorLindblad} y por tener la misma algebra esta queda resuelta por la base \eqref{DampingBasis}. La ecuación resultante es

\begin{equation} \label{GammaLindblat}
\dot{\rho}=\frac{\gamma}{2}(N+1)(2\Gamma\rho\Gamma^\dagger - \Gamma^\dagger\Gamma \rho - \rho \Gamma^\dagger \Gamma )
 + \frac{\gamma}{2}N(2\Gamma^\dagger \rho \Gamma - \Gamma \Gamma^\dagger \rho - \rho \Gamma\Gamma^\dagger)
\end{equation}
\end{frame}

\begin{frame}{Hamiltoniano de Enfriamiento Laser}
Se expresó el Hamiltoniano utilizado en \cite{BarberisLC} para modelar un tambor mecánico interactuando con el campo electromagnético de una cavidad óptica mediante presión de radiación

\begin{equation}
H(t) = -\hbar \delta a^\dagger a + \frac{p^2}{2m} + \frac{1}{2}m \nu (t) x^2 -\hbar g a^\dagger a x + \hbar\frac{\Omega}{2}(a^\dagger + a)
\end{equation}

Y se pasó al formalismo de los operadores de Floquet

\end{frame}

\begin{frame}
\begin{align}
H(t) &= \nonumber -\hbar \delta a^\dagger a + \frac{W}{|f(t)|^2}(\Gamma^\dagger \Gamma + \frac{1}{2}) \\
&+  g'a^\dagger a[\gamma_+(t)\Gamma (t) +\gamma_-(t)\Gamma^\dagger (t)] + \hbar\frac{\Omega}{2}(a^\dagger + a)
\end{align} y se obtuvo así una nueva ecuación maestra de enfriamiento laser

\begin{equation}
\dot{\rho} = \frac{1}{i\hbar}[H,\rho] + L_a\rho + L_\Gamma \rho
\end{equation}
\end{frame}

\begin{frame}{Transformación al Marco Desplazado}
Se utilizó una transformación unitaria a un marco de referencia desplzado para eliminar los términos de tercer orden en operadores ya que estos son no lineales

\begin{equation}
U_{a,\Gamma} = e^{(\alpha(t) a^\dagger - \alpha(t)^*a)}e^{(\beta(t) \Gamma^\dagger - \beta(t)^*\Gamma)},
\end{equation} para aplicar la transformación se utiliza

\begin{equation}
e^{A} B e^{-A} = B + [A,B] + \frac{1}{2}[A,[A,B]] + ... .
\end{equation}
\end{frame}

\begin{frame}
Con esto se obtuvieron las formas transformadas de los operadores

\begin{align}
U^{\dagger} a U =& a + \alpha, \\
U^{\dagger} a^{\dagger} U =& a^{\dagger} + \alpha^*, \\
U^{\dagger} \Gamma U =& \Gamma + \beta, \\
U^{\dagger} \Gamma^{\dagger} U =& \Gamma^{\dagger} + \beta^*, 
\end{align} 

sin embargo la ecuación trasnformada es

\begin{equation}
\dot{\rho} = U_{a,\Gamma}^\dagger L[U_{a,\Gamma} \rho' U_{a,\Gamma}^\dagger]U_{a,\Gamma}-U_{a,\Gamma}^\dagger\dot{U}_{a,\Gamma}\rho'-\rho'\dot{U}_{a,\Gamma}^\dagger U_{a,\Gamma}
\end{equation}

\end{frame}

\begin{frame}
Se prestó especial atención a los términos que involucraban derivadas temporales ya que en este caso algunos operadores dependen explícitamente del tiempo. Esto generó términos involucrando los conmutadores entre los operadores de Floquet y sus derivadas temporales

\begin{align*}
U^{\dagger}\dot{U}\rho + \rho \dot{U}^\dagger U =& -(\dot{\alpha}a \rho + \rho\dot{\alpha}a^{\dagger}) + \dot{\alpha}a^{\dagger}\rho + \rho \dot{\alpha}^*a,\\
&+ \dot{\beta}\Gamma^{\dagger}\rho + \rho\dot{\beta}^*\Gamma-(\dot{\beta}^*\Gamma + \beta^*\dot{\Gamma})\rho - \rho(\dot{\beta} \Gamma^{\dagger} + \beta \dot{\Gamma}^{\dagger}) +\beta \dot{\Gamma}^{\dagger} + \rho\beta^* \dot{\Gamma},\\
&+3(\beta^*)^2C_{--}(t)\rho + |\beta|^2(C_{+-}(t) - C_{-+}(t))\rho \\
&-  \beta^2 C_{++}(t)\rho,
\end{align*}
\end{frame}

\begin{frame}
Se agruparon los términos con un solo operador y se factorizaron para obtener ecuaciones diferenciales para $\alpha(t)$ y $\beta(t)$

\begin{align}
\dot{\alpha} =& \alpha(-\frac{A}{2}+i(\delta+g'(\gamma_-(t) \beta^* + \gamma_+(t) \beta))-i\frac{\Omega}{2},\\
\dot{\beta} =& \beta(-\frac{\gamma}{2}-i\frac{W}{|f(t)|^2})+ig'|\alpha|^2\gamma_+(t),
\end{align}


\end{frame}

\begin{frame}
Se pidió que se cumplieran estas ecuaciones y se obtuvo un nuevo Hamiltoniano en el Marco Desplazado

\begin{align}
H'=& \nonumber -\hbar \delta' a^\dagger a + \frac{W}{|f(t)|^2}\Gamma \Gamma^\dagger + i\hbar(\beta^*\dot{\Gamma} - \beta \dot{\Gamma}^\dagger) \\
&-\hbar g'[(a^{\dagger}a +\alpha a^{\dagger}+\alpha^* a)(\gamma_-(t)\Gamma^{\dagger}+\gamma_+(t)\Gamma)]
\end{align}
\end{frame}

\begin{frame}{Cálculo de Factores Dependientes del Tiempo}
Se utilizó la solución obtenida para el oscilador dependiente del tiempo para obtener expresiones explícitas para los distintos factores con dependencia temporal que aparecen en la ecuación maestra. Comenzando con los conmutadores

\begin{align}
C_{++}(t) =& -\epsilon\frac{i}{8}e^{-4i\omega t},\\
C_{--}(t) =& -\epsilon\frac{i}{8}e^{4i\omega t},\\
C_{+-}(t) =& i [1 -\frac{\epsilon}{16}e^{2i\omega t}-\frac{6\epsilon}{16}e^{-2i\omega t}],\\
C_{-+}(t) =& i [1 -\frac{\epsilon}{16}e^{-2i\omega t}-\frac{6\epsilon}{16}e^{2i\omega t}].
\end{align}

\end{frame}

\begin{frame}
Los coeficientes $\gamma_\pm$

\begin{equation}
\gamma_\pm= \frac{1}{\omega}e^{\mp i\omega t},
\end{equation}

Los operadores $\dot{\Gamma}$

\begin{align}
\dot{\Gamma}(t) =& i\omega \Gamma(t),\\
\dot{\Gamma}^\dagger(t) =& -i\omega \Gamma^\dagger(t),
\end{align}

Y el factor del Hamiltoniano de Oscilador Armónico Dependiente del Tiempo

\begin{equation}
\frac{W}{|f|^2} = \omega.
\end{equation}

\end{frame}

\begin{frame}
Y finalmente soluciones para $\alpha(t)$ y $\beta(t)$

\begin{align}
\alpha_0 =& \frac{\Omega}{2\delta-iA},\\
\beta_0 =& 0.
\end{align}

Lo cual lleva al Hamiltoniano final para el caso de pequeñas oscilaciones en torno a una frecuencia central, a primer orden de perturbación

\begin{equation}
H = -\hbar \delta a^{\dagger}a +\hbar\omega\Gamma^{\dagger}\Gamma +\frac{\hbar g'}{\omega}(\alpha_0 a^{\dagger}+\alpha^*_0 a)(e^{i\omega t}\Gamma^{\dagger}+e^{-i\omega t}\Gamma)
\end{equation}
\end{frame}

\begin{frame}{Escalas Temporales}

Debido a los parámetros involucrados, la ecuación se parte en dos escalas temporales. 

\begin{align*}
L_0 =& L_c + L_m,\\
L_c =& \frac{1}{i\hbar}[H_c,\cdot] + L_A,\\
L_m =& \frac{1}{i\hbar}[H_m,\cdot] + L_\gamma,\\
L_1 =& \frac{1}{i\hbar}[H_{int},\cdot].
\end{align*}

El movimiento optomecánico es mucho más rápido que la interacción.


\end{frame}

\begin{frame}
 Se tomó como parámetro perturbativo $\eta=\frac{g'}{\omega}$.Se proyectó la ecuación al subespacio correspondiente al estado estacionario y se obtuvo

\begin{equation}
P\dot{\rho} = PL_1 \frac{Q}{\lambda_m-L_0}L_1\rho_{st}\mu,
\end{equation}

$\mu$ representa a $Tr_c[P\rho]$
\end{frame}

\begin{frame}

Al desarrollar los conmutadores involucrados en $L_1$ se obtiene una ecuación con la misma estructura que en \cite{BarberisLC} y se llega a

\begin{equation}
\dot{\mu} = \frac{1}{i\hbar}[\hat{H},\mu] + \frac{A_-}{2}D[\Gamma]\mu + \frac{A_+}{2}D[\Gamma^\dagger]\mu.
\end{equation}

En este caso, $\hat{H}$ es una pequeña corrección al Hamiltoniana proporcional a $\Gamma^\dagger \Gamma$ y $A_\pm$ dependen únicamente del valor de las trazas.

\end{frame}

\begin{frame}{Excitaciones Promedio}
Por el trabajo citado previamente, el número de excitaciones promedio del tambor mecánico es

\begin{equation}
<m> = Tr[\Gamma^\dagger \Gamma \mu_{st}] = \frac{A_+}{A_+ - A_-},
\end{equation}

y regresando a los operadores usuales de oscilador da una nueva predicción

\begin{equation}
<m> = Tr[(\frac{(\omega + \nu )^2}{4\nu})b^\dagger b +\frac{\omega^2 - \nu^2}{4\nu}(b^\dagger b^\dagger +bb) +\frac{(\omega - \nu )^2}{4\nu}) b b^\dagger)\mu_st].
\end{equation}

\end{frame}

\begin{frame}{Conclusiones}



\end{frame}

\bibliographystyle{unsrt}
\bibliography{Bib}

\end{document}