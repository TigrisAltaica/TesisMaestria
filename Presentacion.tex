\documentclass[10pt]{beamer}
\usetheme{Berkeley}
\usepackage[utf8]{inputenc}
\usepackage[spanish]{babel}
\selectlanguage{spanish}
\usepackage{amsmath}
\usepackage{amsfonts}
\usepackage{amssymb}
\author{Pablo Enrique Yanes Thomas}
\title{Enfriamiento Optomecánico con Parámetros Dependientes del Tiempo}
%\setbeamercovered{transparent} 
%\setbeamertemplate{navigation symbols}{} 
%\logo{} 
\institute{IIMAS} 
%\date{} 
%\subject{} 
\begin{document}

\begin{frame}
\titlepage
\end{frame}

%\begin{frame}
%\tableofcontents
%\end{frame}

\begin{frame}{Resumen}

\begin{itemize}
\item Se modeló el enfriamiento de un tambor mecánico con frecuencia natural dependiente del tiempo acoplado al campo electromagnético de una cavidad óptica.

\item Se encontró una  ecuación maestra que incorpora un mejor modelo de disipación.

\item Se encontró una expresión analítica para el número promedio de excitaciones del tambor bajo la aproximación adiabática.
\end{itemize}

\end{frame}

\begin{frame}{Sistemas Optomecánicos}
Interacción entre un sistema mecánico cuántico y el campo electromagnético. Hay varios tipos posibles

\begin{itemize}
\item Espejos suspendidos
\item Microresonadores
\item Objetos Suspendidos
\end{itemize}

Se trató con un sistema del primer tipo.

\end{frame}

\begin{frame}{¿Por qué Sistemas Optomecánicos?}

\begin{itemize}
\item Estados mecánicos no clásicos
\item Detección extremadamente sensible
\item Interfaz para información cuántica
\item Preparación de estados mecánicos puros en régimen cuántico 
\end{itemize}


\end{frame}

\begin{frame}{El Sistema}

\includegraphics[scale=1.3]{Sistema_Espejo_Laser.png} 

\end{frame}

\begin{frame}{Estudios Anteriores}

\begin{itemize}
\item Otros trabajos anteriores (Barberis) trataron el sistema sin realmente tomar en cuenta la dependencia temporal del oscilador.

\item Existe un artículo (Hanggi) que establece un procedimiento que lleva a una mejor descripción de la disipación al tomar en cuenta esta dependencia.

\item En este trabajo se aplicó este procedimiento para encontrar una nueva ecuación maestra.

\end{itemize}

\end{frame}

\begin{frame}{Disipación para Oscilador Armónico Dependiente del Tiempo}

Modelo usual para la disipación en un oscilador armónico con frecuencia natural independiente del tiempo

\begin{align*}
\textbf{L}_b \rho =& \frac{\gamma}{2}(N + 1)[2b\rho b^\dagger -b^\dagger b\rho - \rho b^\dagger b ] \nonumber \\
 &+ \frac{\gamma}{2}(N)[2b^\dagger\rho b -bb^\dagger\rho - \rho  bb^\dagger ].
\end{align*} 

\end{frame}

\begin{frame}

\textbf{Operadores de Floquet}

\begin{itemize}
\item Misma regla de conmutación que operadores usuales del oscilador

\item Dependen de la solución clásica
\end{itemize}

\begin{equation*}
\Gamma(t) = \frac{1}{2i}(\dot{f}(t)\sqrt{\frac{2}{\hbar m}}\hat{x}-f(t)\sqrt{\frac{\hbar}{2m}}\hat{p}),
\end{equation*} y su complejo conjugado

\begin{equation*}
m\ddot{f}(t) + \nu(t)f(t)=0
\end{equation*}

\end{frame}

\begin{frame}
Se tiene un mejor modelo de disipación al sustituir operadores de Floquet en el modelo usual para la disipación

\begin{align*} 
L_\Gamma \dot{\rho}=&\frac{\gamma}{2}(N+1)(2\Gamma\rho\Gamma^\dagger - \Gamma^\dagger\Gamma \rho - \rho \Gamma^\dagger \Gamma )\\
 &+ \frac{\gamma}{2}N(2\Gamma^\dagger \rho \Gamma - \Gamma \Gamma^\dagger \rho - \rho \Gamma\Gamma^\dagger)
\end{align*}
\end{frame}


\begin{frame}{Hamiltoniano de Enfriamiento Laser}
Se utilizó el Hamitloniano empleado en el estudio anterior del problema
\begin{equation*}
H(t) = -\hbar \delta a^\dagger a + \frac{p^2}{2m} + \frac{1}{2}m \nu (t) x^2 -\hbar g a^\dagger a x + \hbar\frac{\Omega}{2}(a^\dagger + a)
\end{equation*}

Y se pasó a operadores de Floquet

\end{frame}

\begin{frame}
\begin{align*}
H(t) &= -\hbar \delta a^\dagger a + \frac{W}{|f(t)|^2}(\Gamma^\dagger \Gamma + \frac{1}{2}) \\
&+  g'a^\dagger a[\gamma_+(t)\Gamma (t) +\gamma_-(t)\Gamma^\dagger (t)] + \hbar\frac{\Omega}{2}(a^\dagger + a)
\end{align*} y se obtuvo así una nueva ecuación maestra de enfriamiento laser. Se asumió acoplamiento débil entre la cavidad y el tambor y se realizó aproximación de Markov.

\begin{equation*}
\dot{\rho} = \frac{1}{i\hbar}[H,\rho] + L_a\rho + L_\Gamma \rho
\end{equation*}
\end{frame}

\begin{frame}{Transformación al Marco Desplazado}
Se utilizó una transformación unitaria a un marco de referencia desplazado para eliminar los términos de tercer orden en operadores ya que estos son no lineales

\begin{equation*}
U_{a,\Gamma} = e^{(\alpha(t) a^\dagger - \alpha(t)^*a)}e^{(\beta(t) \Gamma^\dagger - \beta(t)^*\Gamma)},
\end{equation*} para aplicar la transformación se utiliza

\begin{equation*}
e^{A} B e^{-A} = B + [A,B] + \frac{1}{2}[A,[A,B]] + ... .
\end{equation*}
\end{frame}

\begin{frame}
Con esto se obtuvieron las formas transformadas de los operadores

\begin{align*}
U^{\dagger} a U =& a + \alpha, \\
U^{\dagger} a^{\dagger} U =& a^{\dagger} + \alpha^*, \\
U^{\dagger} \Gamma U =& \Gamma + \beta, \\
U^{\dagger} \Gamma^{\dagger} U =& \Gamma^{\dagger} + \beta^*, 
\end{align*} 

sin embargo la ecuación trasnformada es

\begin{equation*}
\dot{\rho} = U_{a,\Gamma}^\dagger L[U_{a,\Gamma} \rho' U_{a,\Gamma}^\dagger]U_{a,\Gamma}-U_{a,\Gamma}^\dagger\dot{U}_{a,\Gamma}\rho'-\rho'\dot{U}_{a,\Gamma}^\dagger U_{a,\Gamma}
\end{equation*}

\end{frame}

\begin{frame}
La dependencia temporal de los operadores generó términos involucrando los conmutadores entre los operadores de Floquet y sus derivadas temporales

\begin{align*}
U^{\dagger}\dot{U}\rho + \rho \dot{U}^\dagger U =& -(\dot{\alpha}a \rho + \rho\dot{\alpha}a^{\dagger}) + \dot{\alpha}a^{\dagger}\rho + \rho \dot{\alpha}^*a,\\
&+ \dot{\beta}\Gamma^{\dagger}\rho + \rho\dot{\beta}^*\Gamma-(\dot{\beta}^*\Gamma + \beta^*\dot{\Gamma})\rho  \\
&- \rho(\dot{\beta} \Gamma^{\dagger}+ \beta \dot{\Gamma}^{\dagger}) +\beta \dot{\Gamma}^{\dagger} + \rho\beta^* \dot{\Gamma},\\
&+3(\beta^*)^2C_{--}(t)\rho + |\beta|^2(C_{+-}(t) - C_{-+}(t))\rho \\
&-  \beta^2 C_{++}(t)\rho,
\end{align*}
\end{frame}

\begin{frame}
Se agruparon los términos con un solo operador y se factorizaron para obtener ecuaciones diferenciales para $\alpha(t)$ y $\beta(t)$

\begin{align*}
\dot{\alpha} =& \alpha(-\frac{A}{2}+i(\delta+g'(\gamma_-(t) \beta^* + \gamma_+(t) \beta))-i\frac{\Omega}{2},\\
\dot{\beta} =& \beta(-\frac{\gamma}{2}-i\frac{W}{|f(t)|^2})+ig'|\alpha|^2\gamma_+(t),
\end{align*}
\end{frame}

\begin{frame}
Se pidió que se cumplieran estas ecuaciones y se obtuvo un nuevo Hamiltoniano en el Marco Desplazado

\begin{align*}
H'=& \nonumber -\hbar \delta' a^\dagger a + \frac{W}{|f(t)|^2}\Gamma \Gamma^\dagger + i\hbar(\beta^*\dot{\Gamma} - \beta \dot{\Gamma}^\dagger) \\
&-\hbar g'[(a^{\dagger}a +\alpha a^{\dagger}+\alpha^* a)(\gamma_-(t)\Gamma^{\dagger}+\gamma_+(t)\Gamma)]
\end{align*}
\end{frame}

\begin{frame}{Caso de Oscilaciones Pequeñas}

Se resuelve clásicamente mediante teoría de Floquet. En este trabajo se trató el caso donde la frecuencia natural es de la forma:

\begin{equation*}
\nu(t) = \nu_0 + \epsilon cos(2\omega t).
\end{equation*} La solución a primer orden en $\epsilon$ es
\begin{equation*}
f(t)= e^{i\omega t} + \frac{\epsilon}{16} e^{3i\omega t}
\end{equation*}

\end{frame}


\begin{frame}
Para los conmutadores

\begin{align*}
C_{++}(t) =& -\epsilon\frac{i}{8}e^{-4i\omega t},\\
C_{--}(t) =& -\epsilon\frac{i}{8}e^{4i\omega t},\\
C_{+-}(t) =& i [1 -\frac{\epsilon}{16}e^{2i\omega t}-\frac{6\epsilon}{16}e^{-2i\omega t}],\\
C_{-+}(t) =& i [1 -\frac{\epsilon}{16}e^{-2i\omega t}-\frac{6\epsilon}{16}e^{2i\omega t}].
\end{align*}

\end{frame}

\begin{frame}
Los coeficientes $\gamma_\pm$

\begin{equation*}
\gamma_\pm= \frac{1}{\omega}e^{\mp i\omega t},
\end{equation*}

Los operadores $\dot{\Gamma}$

\begin{align*}
\dot{\Gamma}(t) =& i\omega \Gamma(t),\\
\dot{\Gamma}^\dagger(t) =& -i\omega \Gamma^\dagger(t),
\end{align*}

Y el factor del Hamiltoniano de Oscilador Armónico Dependiente del Tiempo

\begin{equation*}
\frac{W}{|f|^2} = \omega.
\end{equation*}

\end{frame}

\begin{frame}
Y finalmente soluciones para $\alpha(t)$ y $\beta(t)$

\begin{align*}
\alpha_0 =& \frac{\Omega}{2\delta-iA},\\
\beta_0 =& 0.
\end{align*}

Lo cual lleva al Hamiltoniano final para el caso de pequeñas oscilaciones en torno a una frecuencia central, a primer orden de perturbación. Se asumió $\Omega$ grande para despreciar el término sin $\alpha$ 

\begin{equation*}
H = -\hbar \delta a^{\dagger}a +\hbar\omega\Gamma^{\dagger}\Gamma +\frac{\hbar g'}{\omega}(\alpha_0 a^{\dagger}+\alpha^*_0 a)(e^{i\omega t}\Gamma^{\dagger}+e^{-i\omega t}\Gamma)
\end{equation*}
\end{frame}

\begin{frame}{Escalas Temporales}

Debido a los parámetros involucrados, la ecuación se parte en dos escalas temporales. 

\begin{align*}
L_0 =& L_c + L_m,\\
L_c =& \frac{1}{i\hbar}[H_c,\cdot] + L_A,\\
L_m =& \frac{1}{i\hbar}[H_m,\cdot] + L_\gamma,\\
L_1 =& \frac{1}{i\hbar}[H_{int},\cdot].
\end{align*}

El movimiento optomecánico es mucho más rápido que la interacción, por lo que se utilizó la aproximación adiabática.


\end{frame}

\begin{frame}
Se tomó como parámetro perturbativo $\eta=\frac{g'}{\omega}$.Se proyectó la ecuación al subespacio correspondiente al estado estacionario

\begin{align*}
P = P_c^{\lambda_c = 0}P_m^{\lambda_m = 0},\\
Q = 1 - P.
\end{align*}

Se obtuvieron dos ecuaciones acopladas

\begin{align*}
P\dot{\rho}=& PL_1Q\rho, \\
Q\dot{\rho}=& QLQ\rho + QLP\rho.
\end{align*}
\end{frame}

\begin{frame}

El proceso consistió en

\begin{itemize}
\item Pasar a cuadro de decaimiento.
\item Resolver para $Q\rho$.
\item Sustituir en la ecuación para $P$.
\item Desarrollar los proyectores en la base de $L_0$.
\item Hacer las integrales temporales.
\end{itemize}

\end{frame}

\begin{frame}
Para poder aplicar los proyectores estos se desarrollaron en la base que el oscilador armónico con intercambios de energía, la base de decaimiento

\begin{align*}
&a^{\dagger l}\frac{(-1)^n}{(\nu+1)^{l+1}}:L_n^l[\frac{a^\dagger a}{\nu+1}]e^{-[\frac{a^\dagger a}{\nu+1}]}:\quad l \geq 0, \\
&\frac{(-1)^n}{(\nu+1)^{|l|+1}}:L_n^{|l|}[\frac{a^\dagger a}{\nu+1}]e^{-[\frac{a^\dagger a}{\nu+1}]}:a^{|l|}\quad l \leq 0,
\end{align*}
\end{frame}

\begin{frame}
Y se llegó, despues de trazar sobre los estaos de la cavidad a

\begin{equation*}
\dot{\mu} = PL_1 \frac{Q}{\lambda_m-L_0}L_1\rho_{st}\mu,
\end{equation*}

$\mu$ representa la solución en el espacio $P$ después de trazar sobre los estados de la cavidad. 
\end{frame}

\begin{frame}

Al desarrollar los conmutadores involucrados en $L_1$ se obtuvo una ecuación con la misma estructura algebráica que en trabajos anteriores, por lo que se conserva la estructura de la solución

\begin{equation*}
\dot{\mu} = \frac{1}{i\hbar}[\hat{H},\mu] + \frac{A_-}{2}D[\Gamma]\mu + \frac{A_+}{2}D[\Gamma^\dagger]\mu.
\end{equation*}

En este caso, $\hat{H}$ es una pequeña corrección al Hamiltoniana proporcional a $\Gamma^\dagger \Gamma$ y $A_\pm$ dependen únicamente del valor de las trazas.

\end{frame}

\begin{frame}{Excitaciones Promedio}
En los estudios anteriores, el número de excitaciones promedio del tambor mecánico es

\begin{equation*}
<m> = Tr[\Gamma^\dagger \Gamma \mu_{st}] = \frac{A_+}{A_+ - A_-},
\end{equation*}

y regresando a los operadores usuales de oscilador da una nueva expresión

\begin{equation*}
<m> = Tr[(\frac{(\omega + \nu )^2}{4\nu})b^\dagger b +\frac{\omega^2 - \nu^2}{4\nu}(b^\dagger b^\dagger +bb) +\frac{(\omega - \nu )^2}{4\nu}) b b^\dagger)\mu_st].
\end{equation*}

\end{frame}

\begin{frame}{Conclusiones}

\begin{itemize}
\item Se obtuvo un mejor modelo para la disipación para un oscilador armónico con frecuencia natural dependiente del tiempo.

\item A futuro se estudiará la solución numericamente, así como estudiar los cambios en la disipación de la cavidad.

\item La solución obtenida depende fuertemente de la solución clásica del oscilador armónico con frecuencia natural dependiente del tiempo.
\end{itemize}


\end{frame}


\end{document}